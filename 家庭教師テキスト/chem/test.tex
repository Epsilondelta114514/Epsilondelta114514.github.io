\documentclass[a4paper,14pt]{ltjsarticle}
\usepackage{base}
\usepackage[top=25truemm,bottom=20truemm,left=10truemm,right=10truemm]{geometry}
\title{}
\author{}
\date{}
\begin{document}
\pagestyle{empty}
\begin{itemize}
    \item アルコールはヒドロキシ基[     ]を持つ化合物で,単体の[      ]と反応して,水素と[                  ]を生成する.例えば,エタノールの反応は次の反応式で表される:\\
    \noindent [                                     ]\\
    \item メタノールと単体の[      ]を反応させると[             ]が,エタノールと反応させると[                ]が生じる.\\
    \item アルコールに濃硫酸を加えて160C$^\circ$程度に加熱すると[     ]で脱水反応が起こり,[       ]が生じる.一方で,130C$^\circ$程度に加熱すると[      ]で脱水反応が起こり,[      ]が生じる.\\
    \item アルコールはヒドロキシキが結合する炭素に結合する水素の数で1級,2級,3級アルコールに分類される.第1級アルコールを酸化すると[         ],[         ]の順に変化する.第2級アルコールは酸化すると[     ]になる.第3級アルコールは酸化されにくい.\\
    \item アルデヒドは[      ]基をもつ化合物で,第[  ]級アルコールを酸化して得られる.[       ]性を持ち,次の2つの検出法が使われる.\begin{itemize}
        \item アンモニア性硝酸銀水溶液にアルデヒドを加えて加熱すると,単体の銀が析出する([       ]反応)
              \item フェーリング液にアルデヒドを加えて加熱すると,[         ]色の[         ]が沈殿する.(フェーリング反応)
    \end{itemize}
     \\
    \item ケトンは[      ]基をもつ化合物で,アルデヒドと異なり,[       ]性を持たない.メチル基を2つ持つケトンは[      ]と呼ばれ,有機溶媒として用いられる.\\
  
\end{itemize}
\newpage
\begin{itemize}
      \item ~[       ]基をもつ化合物はヨードホルム反応を示し,ヨウ素と水酸化ナトリウムを混ぜて加熱すると[        ]の[        ]色沈殿を生じる.ただし,この反応は\ce{O}と二重結合している炭素に[      ]原子または[      ]原子が結合している場合に限って起こる.よって,酢酸とエステルはヨードホルム反応を[       ].\\
    \item カルボン酸は[         ]基を持つ化合物である.液性は[      ]性だが,炭酸よりは[       ].よって,炭酸水素ナトリウムにカルボン酸を加えると[           ]反応が起き,[            ]が発生する.\\
    \item 2つのカルボキシ基の間で脱水反応が起こると,[              ]が生じる.例えば,2価カルボン酸のフマル酸とマレイン酸のうち,[        ]は分子[   ]で脱水反応をおこし,[             ]を生じる.分子内脱水を起こす他の例としては,ベンゼン環に2つのカルボキシ基が結合した[           ]などがある.\\
    \item カルボン酸とアルコールで脱水反応を起こすと,[          ]が生じる.このとき,[          ]から\ce{-OH}が脱離するのであった.この化合物はアセチル基を持つが,ヨードホルム反応を[         ].\\
    \item エステルに酸や塩基を入れると[           ]が起こり,カルボン酸とアルコールが再生する.特に,塩基を使う場合は[         ]化と呼ばれ,カルボン酸はナトリウム塩の形で生じる.\\
    \item 4種類の異なる原子または原子団と結合している炭素原子を[          ]という.このような炭素原子を持つ化合物には[           ]異性体が存在する.
\end{itemize}
\end{document}