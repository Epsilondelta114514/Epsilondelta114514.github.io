\documentclass[a4paper,11pt]{ltjarticle}
\usepackage{base}
\title{}
\author{}
\date{}
\usepackage[top=15mm, bottom=15mm, left=25mm, right=25mm]{geometry} 
\title{6/4 追加問題の解答}
\author{}
\date{}
\begin{document}
\begin{que}
素焼き版で仕切った容器に硫酸亜鉛水溶液と硫酸銅(II)水溶液を入れ,亜鉛板と銅板をそれぞれの水溶液に浸し,電池とした.
\begin{itemize}
\item[(1)]電流は導線中をどの向きで流れるか説明せよ.
\item[(2)]負極と正極で起こる反応を,それぞれ電子\ce{e-}を用いた反応式で示せ.
\item[(3)] この電池の起電力は1.10Vであった.代わりに銅電極と銀電極で電池を作ると起電力は0.46Vであった.亜鉛電極と銀電極で電池を作った場合の負極はどちらか.また,その電池の起電力を求めよ. 
\end{itemize}
\end{que}
\ans
\newpage
\begin{que}
代表的な二次電池である鉛蓄電池は,正極に\ce{PbO2},負極に\ce{Pb},電解液に質量パーセント濃度が38.0$\%$の希硫酸(密度1.28[g/cm$^3$])を用いていおり,放電によって両電極の表面に\ce{PbSO4}が形成される.\\
 \ce{H}$=1.00$,~\ce{O}$=16.0$,~\ce{S}$=32.0$,~\ce{Pb}$=207$,~ファラデー定数を$F=9.65\times 10^{4}$~[C/mol]として以下の問いに答えよ.
\begin{itemize}
\item[(1)]正極および負極で起きる放電時の反応を電子\ce{e-}を含むイオン反応式でそれぞれ示せ.
\item[(2)]電流5.00Aで5時間21分40秒の放電を行ったとき,正極および負極の質量はそれぞれどれだけ増減するかを計算せよ.
\item[(3)] 放電前の希硫酸が1.00kgであった場合,上記の放電後の希硫酸の質量パーセント濃度を求めよ.
\end{itemize}
\end{que}
\ans
\newpage
\begin{que}
水素1.75mol,ヨウ素1.50molを容器に入れて加熱した.圧力・温度を一定に保ったところ,ヨウ化水素が生じて平衡状態に達した.このとき,水素は0.50molに減少していた.
\begin{itemize}
\item[(1)]平衡状態とはどのような状態か40字程度で説明せよ.
\item[(2)]平衡時のヨウ素とヨウ化水素はそれぞれ何molか.
\item[(3)]平衡定数$K$を求めよ.
\item[(4)]この平衡状態において,圧力・温度を一定に保ったまま水素を注入すると平衡は移動するか.また,このときの$K$の値はどうなるか.
\end{itemize}
\end{que}
\ans
\newpage
\begin{que}
括弧内のように条件を変化させると,次の平衡はどちらに移動するか.ただし,(s)は固体,(g)は気体であることを表す.
\begin{itemize}
\item[(1)]\ce{N2O4 <=> 2NO2}~~~~$\varDelta H=57$kj~~(加熱する)
\item[(2)]\ce{CO2 + H2O(g) <=> CO2 + H2}~~(圧力を高くする)
\item[(3)]\ce{NaCl(s) +aq <=> Na+ aq + Cl- aq}~~(塩化水素を通じる)
\item[(4)]\ce{N2 + 3H2 <=> 2NH3}~~(全圧を一定に保ち,アルゴンを加える)
\item[(5)]\ce{N2 + 3H2 <=> 2NH3}~~(体積を一定に保ち,アルゴンを加える)
\item[(6)]\ce{C(s) + H2O(g) <=> CO + H2}~~(圧力を高くする)
\end{itemize}
\end{que}
\ans
\newpage

\begin{que}
酢酸とエタノールを3.0molずつ混ぜ,少量の濃硫酸の下で一定温度に保ったところ,化合物Xと水が2.0molずつ生じたところで平衡に達した.
\begin{itemize}
\item[(1)]酢酸とエタノールの構造式を記せ.
\item[(2)]化合物Xの名称と構造式を記せ.
\item[(3)]この反応の化学反応式を示せ.
\item[(4)]この反応の平衡定数を求めよ.
\item[(5)]平衡状態に達したのち,酢酸とXを1molずつ追加で加えると,平衡はどちらに移動するか. 

\end{itemize}
\end{que}
\ans
\newpage
\begin{que}
\ce{N2O4}は\ce{N2O4 <=> 2NO2}のように反応する.容器に\ce{N2O4}を入れて圧力を$1.0\times10^5$Paに保ったところ,$40\%$の\ce{N2O4}が反応して平衡状態に達した.
\begin{itemize}
\item[(1)]\ce{N2O4},~\ce{NO2}の分圧[Pa]を求めよ.
\item[(2)]圧平衡定数$K_p$を求めよ.
\end{itemize}
\end{que}
\ans
\newpage
\begin{que}
    \begin{minipage}{0.6\linewidth}
    5種類の気体と,それらを発生させるために用いる試薬を表に示す.
    \begin{itemize}
    \item [(1)]表中の\fbox{ア}〜\fbox{エ}に当てはまる試薬として最も適切なものをそれぞれ選び,組成式で答えよ.\\[3pt]
    蛍石,塩化アンモニウム,塩化ナトリウム,硫化鉄(II),塩化カルシウム,酸化マンガン(IV),亜硫酸ナトリウム,硫黄,石灰石
    \item [(2)]表の示した5つの反応を化学反応式で示せ.
\end{itemize}
    \end{minipage}
    \begin{minipage}{0.1\linewidth}
    
    \end{minipage}
    \begin{minipage}{0.3\linewidth}
    \begin{tabular}{cc}
\toprule
気体 & 試薬 \\
\midrule
水素 & 亜鉛と希硫酸 \\
硫化水素 & \fbox{ア}と希硫酸 \\
塩化水素 & \fbox{イ}と濃硫酸\\
二酸化硫黄 & \fbox{ウ}と希硫酸\\
塩素 & \fbox{エ}の濃塩酸 \\
\bottomrule
\end{tabular}
    \end{minipage}
\begin{itemize}
    \item [(3)]各気体の特徴を表す記述を,次からそれぞれ選べ.
    \begin{itemize}
   \item [(a)]無色で刺激臭がある.水に溶けやすく,水溶液は強酸性を示す.
\item [(b)]無色で水に溶けにくい.空気に触れると赤褐色になる.
\item [(c)]無色・無臭である.酸素との混合気体は,添加により爆発的に反応する.
\item [(d)]黄緑色で刺激臭がある.水にいくらか溶ける.
\item [(e)]無色で腐乱臭がある.多くの金属イオンと反応し,沈殿を生じる.
\item [(f)]赤褐色で刺激臭がある.水に溶けやすく,水溶液は酸性を示す.
\item [(g)]無色で刺激臭がある.硫酸の原料として工業的に用いられている.
    \end{itemize}
    \item[(4)]亜鉛と希硫酸を混合することで,標準状態で1.12Lの水素を得るためには,濃度2.0mol/Lの希硫酸が何ml必要か.ただし,亜鉛は十分に用意されていると仮定してよい.
\end{itemize}
\end{que}
\ans
\newpage
\begin{que}
分子式\ce{C9H10O2}の3種類の芳香族エステルA,B,Cがある.次の記述を読み,化合物A〜Gの構造式と名称を示せ.
\begin{itemize}
   \item [(a)]Aを加水分解すると,化合物Dとエタノールが生じた.
\item [(b)]Bを加水分解すると,化合物EとFが生じた.
\item [(c)]Eはエタノールを十分に酸化したときの生成物と同一物であった.
\item [(d)]Fを十分に酸化すると,Dが生じた.
\item [(e)]Cを加水分解すると,ベンゼン一置換体であるGとメタノールが生じた.
\end{itemize}
\end{que}
\ans

\end{document}

