\documentclass[a4paper,11pt]{ltjsarticle}
\usepackage{base}
\title{}
\author{}
\date{}
\begin{document}
\begin{que}%重要問題週97
\begin{itemize}
    \item [(1)]エンタルピー変化を付した反応式(a)〜(d)で表される反応が,それぞれある温度,圧力の下で平衡状態にある.反応条件を高温高圧にしたとき,平衡はどのように移動するか,(ア)〜(ウ)から選べ.
    \begin{itemize}
        \item [(a)]\ce{N2 + O2 -> 2NO}~~~~~$\varDelta H=181$kJ
        \item [(b)]\ce{C(s) + CO2 -> 2CO}~~~~~$\varDelta H=172$kJ
        \item [(c)]\ce{N2 + 3H2 -> 2NH3}~~~~~$\varDelta H=-92$kJ
        \item [(d)]\ce{CO + H2O(g) -> CO2 + H2}~~~~~$\varDelta H=-42$kJ
    \end{itemize}
     \\
    (ア)右に移動する (イ)左に移動する (ウ)この条件からは判断できない
    \item [(2)]塩化ナトリウムの飽和水溶液に塩化水素を通じるとどのような変化が起こるか.また,その減少を何効果というか.さらに,その理由も述べよ.
\end{itemize}
\end{que}
\ans
\newpage

\begin{que}
    電離平衡に関する次の問に答えよ.
\begin{itemize}%需要問題週120,リードアルファ197
    \item [(1)]酢酸を水に溶解したときの平衡反応の化学反応式を示せ.
    \item [(2)]溶解した酢酸の初濃度を$c$~mol/Lとして,電離度$\alpha$と電離定数$K_a$の関係を表す近似式を記せ.
    \item [(3)](2)のとき,水溶液の濃度を$\dfrac{1}{10}$倍にすると$\alpha$は何倍になるか.また,pHはどれだけ変化するか.
    \item [(4)]酢酸の電離定数が$K_a=2.0\times 10^{-5}$~mol/Lであるとき,$0.20$~mol/Lの酢酸水溶液のpHを計算せよ.ただし,$\log_{10}2=0.30,~\log_{10}3=0.48$とする.
\end{itemize}
\end{que}
\ans
\newpage
\begin{que}%新演習92くらい
次の反応で窒素と水素からアンモニアを合成した:\\
\centerline{\ce{N2 + 3H2 <=> 2NH3},~~$\varDelta H=-92$kJ}
反応容器に3.0molの窒素と9molの水素を入れ,触媒の存在下で400$^\circ$Cに保って反応させたところ平衡状態に達し,全圧が$1.0\times 10^7$Paとなり,体積で50$\%$のアンモニアを含むようになった.気体はすべて理想気体として振る舞いうと仮定して,次の各問いに答えよ.ただし,気体定数を$R=8.3\times 10^3$~(Pa$\cdot$L/(K$\cdot$mol))とする.
\begin{itemize}
    \item [(1)]平衡時の窒素,水素,アンモニアの物質量はそれぞれ何molか.
    \item [(2)]反応により発生した熱量は何kJか.
    \item [(3)]平衡時の混合気体($400^\circ$C,~$1.0\times 10^7$Pa)の全体積は何Lか.
    \item [(4)]この反応の平衡定数$K$を表す式を書け.
    \item [(5)]この条件下での平衡定数$K$の値を求めよ.
\end{itemize}
\end{que}
\newpage
\setcounter{page}{1}
\setcounter{quecounter}{0}
\section*{解答}
\begin{que}%重要問題週97
\begin{itemize}
    \item [(1)]エンタルピー変化を付した反応式(a)〜(d)で表される反応が,それぞれある温度,圧力の下で平衡状態にある.反応条件を高温高圧にしたとき,平衡はどのように移動するか,(ア)〜(ウ)から選べ.
    \begin{itemize}
        \item [(a)]\ce{N2 + O2 -> 2NO}~~~~~$\varDelta H=181$kJ
        \item [(b)]\ce{C(s) + CO2 -> 2CO}~~~~~$\varDelta H=172$kJ
        \item [(c)]\ce{N2 + 3H2 -> 2NH3}~~~~~$\varDelta H=-92$kJ
        \item [(d)]\ce{CO + H2O(g) -> CO2 + H2}~~~~~$\varDelta H=-42$kJ
    \end{itemize}
     \\
    (ア)右に移動する (イ)左に移動する (ウ)この条件からは判断できない
    \item [(2)]塩化ナトリウムの飽和水溶液に塩化水素を通じるとどのような変化が起こるか.また,その減少を何効果というか.さらに,その理由も述べよ.
\end{itemize}
\end{que}
\noindent\textbf{解答.}
\begin{itemize}
    \item [(1)]ア (b)ウ (c)ウ (d)イ
    \item [(2)]\textbf{(変化)}塩化ナトリウムの結晶が析出する.\\
    (\textbf{現象})共通イオン効果\\
    \textbf{(理由)}塩化ナトリウムの飽和水溶液中では,\ce{NaCl <=> Na+ + Cl-}なる平衡反応が起きている.塩化水素を通じると右辺の\ce{Cl-}濃度が大きくなるため,ルシャトリエの原理から平衡は左に移動する.よって,塩化ナトリウムの結晶が析出する.
\end{itemize}
\newpage

\begin{que}
    電離平衡に関する次の問に答えよ.
\begin{itemize}%需要問題週120,リードアルファ197
    \item [(1)]酢酸を水に溶解したときの平衡反応の化学反応式を示せ.
    \item [(2)]溶解した酢酸の初濃度を$c$~mol/Lとして,電離度$\alpha$と電離定数$K_a$の関係を表す近似式を記せ.
    \item [(3)](2)のとき,水溶液の濃度を$\dfrac{1}{10}$倍にすると$\alpha$は何倍になるか.また,pHはどれだけ変化するか.
    \item [(4)]酢酸の電離定数が$K_a=2.0\times 10^{-5}$~mol/Lであるとき,$0.20$~mol/Lの酢酸水溶液のpHを計算せよ.ただし,$\log_{10}2=0.30,~\log_{10}3=0.48$とする.
\end{itemize}
\end{que}
\noindent\textbf{解答.}
\begin{itemize}
    \item [(1)]\ce{CH3COOH  <=> CH3COO- + H+}
\item[(2)]電離度が$\alpha$なので,酢酸イオンと水素イオンの濃度は$c\alpha$~mol/Lである.よって,電離定数は
\[K_a=\frac{c^2\alpha^2}{c(1-\alpha)}~\text{[mol/L]}\]
である.ここで,酢酸は弱酸であるから,その電離度は1より十分に小さく,$1-\alpha \fallingdotseq 1$と近似してよい.よって,電離定数の近似式は
\[K_a\fallingdotseq \frac{c^2\alpha^2}{c}=c\alpha^2~\text{[mol/L]}\]
である.
\item[(3)]$c$を$\dfrac{1}{10}c$に置きかえたときの電離度を$\alpha'$とする.電離定数は温度にのみ依存するので,水溶液の濃度を変えても変化しない.したがって,
\[\frac1{10}\alpha'^2=c\alpha^2\]
が成り立つ.これを解いて$\alpha'=\sqrt{10}\alpha$となるので,電離度は$\sqrt{10}=3.1622\cdots\fallingdotseq 3.2$倍になる.\\
 また,水素イオン濃度は$[\ce{H+}]=c\alpha'$なので,
\[\ce{pH}=-\log_{10}[\ce{H+}]=-\log_{10}\frac{c}{10}\alpha'=-\log_{10}\left(\frac c{10}\cdot\sqrt{10}\alpha\right)=\frac12-\log_{10}c\alpha\]
となり,\ce{pH}は0.50大きくなることがわかる.
\item[(4)]水素イオン濃度は$[\ce{H+}]=c\alpha=\sqrt{cK_a}$であることに注意すると,
\[\ce{pH}=-\frac12\log_{10}cK_a=-\frac12\log_{10}4.0\times 10^{-6}=3.0-
\log_{10}2=2.7\]
\end{itemize}
\newpage
\begin{que}%新演習92くらい
次の反応で窒素と水素からアンモニアを合成した:\\
\centerline{\ce{N2 + 3H2 <=> 2NH3},~~$\varDelta H=-92$kJ}
反応容器に3.0molの窒素と9molの水素を入れ,触媒の存在下で400$^\circ$Cに保って反応させたところ平衡状態に達し,全圧が$1.0\times 10^7$Paとなり,体積で50$\%$のアンモニアを含むようになった.気体はすべて理想気体として振る舞いうと仮定して,次の各問いに答えよ.ただし,気体定数を$R=8.3\times 10^3$~(Pa$\cdot$L/(K$\cdot$mol))とする.
\begin{itemize}
    \item [(1)]平衡時の窒素,水素,アンモニアの物質量はそれぞれ何molか.
    \item [(2)]反応により発生した熱量は何kJか.
    \item [(3)]平衡時の混合気体($400^\circ$C,~$1.0\times 10^7$Pa)の全体積は何Lか.
    \item [(4)]この反応の平衡定数$K$を表す式を書け.
    \item [(5)]この条件下での平衡定数$K$の値を求めよ.
\end{itemize}
\end{que}
\noindent \textbf{解答.}
\begin{itemize}
    \item [(1)]アンモニアが$2x$~mol生成して平衡に達したとすると,窒素,水素の物質量はそれぞれ$3.0-x$~mol,~$9.0-3x$~molである.よって,全物質量は
    $12-2x$~molである.アンモニアが50$\%$の体積を占めたことから,
    \[\frac{2x}{12-2x}=\frac12\]
    が成り立つので,これを解いて,$x=2.0$~molを得る.したがって,窒素,水素,アンモニアの物質量はそれぞれ1.0mol,~3.0mol,~4.0molである.
    \item[(2)]アンモニアが4.0mol生成したので,エンタルピーの変化は$-184=1.8\times 10^2$kJである.よって,$1.8\times 10^2$kJの発熱があったことがわかる.
    \item[(3)]平衡混合気体の体積を$V$~Lとすると,気体の状態方程式から
    \[V=\frac{nRT}{P}=\frac{8.0\times 8.3\times 10^3\times(273+400)}{1.0\times10^7}=4.46\cdots\fallingdotseq 4.5(\text{L})\]
    を得る.
    \item[(5)]今回は両辺の係数の和が異なるので,体積はキャンセルされずに残ることに注意する.
    \[K=\frac{[\ce{NH3}]^2}{[\ce{N2}][\ce{H2}]^3}=\frac{16}{1.0\times 27}\times V^2\fallingdotseq 12~(\text{L/mol})^2\]
\end{itemize}
\end{document}