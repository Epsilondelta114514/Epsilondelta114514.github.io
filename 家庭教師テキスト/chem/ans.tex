\documentclass[a4paper,12pt]{ltjsreport}
\usepackage{base}
\setchemfig{atom sep=2em}
\title{}
\author{}
\date{}
\begin{document}

\subsection*{練習問題}
\begin{que}
次のアルコールの名称を答えよ.また,級数でアルコールを分類せよ.\\[5pt]
\begin{minipage}{0.5\linewidth}
    \begin{itemize}
        \item [(1)]\chemfig{CH3-CH2-CH2-CH2-OH}\\
        \item [(3)]\chemfig{CH3-CH(-[6]OH)-CH3}\\[5pt]
        \item [(5)]\chemfig{CH3-C(-[2]CH3)(-[6]OH)-CH3}
    \end{itemize}
\end{minipage}
\begin{minipage}{0.5\linewidth}
\begin{itemize}
    \item [(2)]\chemfig{CH3-CH2-OH}\\
    \item [(4)]\chemfig{CH3-CH(-[6]OH)-CH(-[6]OH)-CH3}\\[5pt]
    \item [(6)]\chemfig{CH3-C(-[2]H)(-[6]OH)-C(-[2]CH3)(-[6]H)-CH2-OH}
\end{itemize}
\end{minipage}
\end{que}
\ans
\begin{itemize}
    \item [(1)]1-ブタノール \\
    \item [(2)]エタノール\\
     \item [(3)]2-プロパノール\\
      \item [(4)]2,3-ブタンジオール\\
       \item [(5)]2-メチル-2-プロパノール\\
        \item [(6)]2-メチル-1,3-プロパノール\\
\end{itemize}
    第1級アルコール:(1),(2),および(6)の1-ヒドロキシ基\\[7pt]

   \noindent  第2級アルコール:(3),および(4)の2つのヒドロキシ基と(6)の3-ヒドロキシ基\\[7pt]

    \noindent 第3級アルコール:(5)\newpage
    \begin{que}
        この問題では,\ce{H}$=$1.0,\ce{C}$=$12,\ce{O}$=$16,\ce{Na}$=$23とする.
    \begin{itemize}
        \item [(1)]エタノールと単体ナトリウムの反応を化学反応式で示せ.また,ナトリウムを含む生成物の名称を答えよ.
    \end{itemize}
            あるアルコール\ce{C}$_{x}$\ce{H}$_{y}$\ce{OH}1.5gを十分な量のナトリウムと反応させたところ,標準状態で280mLの気体が発生した.
            \begin{itemize}
                \item [(2)]$x,y$を決定せよ.
                \item [(3)]アルコールの構造の候補をすべて構造式で記せ.
            \end{itemize}
    \end{que}
    \ans
    \begin{itemize}
        \item [(1)]反応式:\ce{2C2H5OH + 2Na -> 2C2H5ONa + H2}\\[15pt]
         名称:ナトリウムエトキシド\\[10pt]
        \item[(2)]計算:生成した水素は$280\times 10^{-3}/22.4$molなので,アルコールの分子量を$M$とすると,化学反応式の係数比から
        \[\frac{1.5}{M}=2\times\frac{280\times 10^{-3}}{22.4}~\Longleftrightarrow~M=60\]
        $M=12x+y+17$であるから$(x,y)=(0,43),(1,31),(2,19),(3,7)$であるが,炭素に結合できる水素の数を考えると,どう考えても$x=3,7$である.\\[10pt]$                        \underline{x=3~~~~~~~~~~~~~~y=7~~~~~~~~~~~}$\\[10pt]
        \item[(3)] \chemfig{CH_3-CH_2-CH_2-OH}      \chemfig{CH_3-CH(-[6]OH)-CH_3}
    \end{itemize}
    \newpage
    \begin{que}
      2-ペンタノールを濃硫酸で160C$^\circ$まで加熱し,脱水させた.
      \begin{itemize}
        \item [(1)]2-ペンタノールの構造式を示せ.
        \item [(2)]この脱水反応の反応式を示し,生成した炭化水素の構造式と名称を答えよ.
      \end{itemize}
    \end{que}
    \ans
    \begin{itemize}
        \item [(1)] \chemfig{CH3-CH(-[6]OH)-CH2-CH2-CH3}\\[70pt]
        \item [(2)]反応式:\ce{C5H11OH -> C5H10 + H2O}\\[30pt]
         名称:2-ペンテン\\[30pt]
        構造式:\chemfig{CH_3-CH=CH-CH_2-CH_3}
        \ascboxF{注意}
        今回の脱水ではヒドロキシ炭素の左右どちらの炭素から水素を奪うかで生成物が2種類考えられるが,経験的に水素が少ない方の炭素から水素が脱離することが知られている(ザイツェフ則).よって,この場合は1-ペンテンではなく2-ペンテンが主生成物となる.
    \end{itemize}
  \newpage
        \begin{que}
        \begin{itemize}
            \item [(1)]次のエーテルの名称を答えよ.\\
            
            \noindent(a)~\chemfig{([:-30]*6(-=-(-O-(*6(-=-=-=)))=-=))}~~~~~~~(b)~\chemfig{H-C(-[2]H)(-[6]H)-O-C(-[2]H)(-[6]H)-H}\\[5pt]
            (c)~\chemfig{H-C(-[2]H)(-[6]H)-C(-[2]H)(-[6]H)-C(-[2]H)(-[6]H)-C(-[2]H)(-[6]H)-O-C(-[2]H)(-[6]H)-H}\\
            \item [(2)]次の化合物の構造式を描け.\\
            (a)~エチルブチルエーテル~~~~~~~~~~(b)~ジプロピルエーテル~~\\
            (c)~エチルヘキシルエーテル
        \end{itemize}
        \end{que}
        \ans
        \begin{itemize}
            \item [(1)](a)ジフェニルエーテル~~~~~~~~~~~~~~~~~~~~~~~~~~~~~~~(b)ジメチルエーテル~~~~~~~~~~~~~~~~~\\[15pt]
            (c)ブチルメチルエーテル\\[10pt]
            \item[(2)](a)\chemfig{CH_3-CH_2-O-CH_2-CH_2-CH_2-CH_3}\\[70pt]
            (b)\chemfig{CH_3-CH_2-CH_2-O-CH_2-CH_2-CH_3}\\[70pt]

            (c)\chemfig{CH_3-CH_2-O-CH_2-CH_2-CH_2-CH_2-CH_2-CH_3}
        \end{itemize}
        \newpage
        \begin{que}
        枝分かれを持たない第1級アルコール\ce{C}$_{x}$\ce{H}$_{2x+1}$\ce{OH}を54.76g用意し,濃硫酸で130C$^\circ$程度に加熱して脱水させたところ,$6.66$gの水が生じた.
      \begin{itemize}
        \item [(1)]この脱水反応の反応式を示せ.
        \item [(2)]$x$を求めよ.
        \item [(3)]反応したアルコールおよび生成した有機化合物の名称を答えよ.
      \end{itemize}
        \end{que}
        \ans
        \begin{itemize}
            \item [(1)]2\ce{C}$_{x}$\ce{H}$_{2x+1}$\ce{OH}$~\longrightarrow~$\ce{C}$_{x}$\ce{H}$_{2x+1}$\ce{-O - }\ce{C}$_{x}$\ce{H}$_{2x+1}$\ce{ + H2O}\\[15pt]
            \item [(2)]計算:生成した水は$6.66/18$molなので,反応したアルコールは$6.66/18\times 2$molである.よって,分子量を$M$とすると,
            \[\frac{54.76}{M}=\frac{6.66}{18}\times2\]
            である.よって,$M=54.76/0.74=74$であり,$x=~4$を得る.\\
            \rightline{\underline{$x=4$       }}\\
            \item [(3)]アルコールの名称:1-ブタノール\\[35pt]
        生成した有機化合物の名称:ジブチルエーテル
        \end{itemize}
\newpage
\begin{que}
\begin{itemize}
    \item [(1)]ホルムアルデヒドとアセトンの構造式を描け.
    \item [(2)]次の記述のうち,ホルムアルデヒドのみに当てはまる性質には◯,アセトンのみに当てはまる性質には△,両方に当てはまる性質には☆を,どちらにも当てはまらない性質には$\times$をつけよ.
    \begin{align*}
    &(\text{a})酸化するとカルボン酸になる&&(\text{b})常温で液体である.\\
    &(\text{c})水によく溶ける.&&(\text{d})酸性を示す.\\
    &(\text{e})フェーリング液を還元する.&&(\text{f})銀鏡反応を示す.\\
    &(\text{g})結合~\chemfig{-C(=[6]O)-}~を持つ.&&(\text{h})還元するとアルコールになる.
    \end{align*}
\end{itemize}
\end{que}
\ans 
(1)略\\
(2)\\[5pt]
\noindent (a)◯        (b)☆        (c)☆        (d)$\times$        \\
(e)△        (f)◯        (g)☆        (h)☆        
\newpage
\begin{que}
磨いた銅線をらせん状に巻いてガスバーナーで熱した.動線を炎から出し,冷却したあとに観察すると,\underline{銅線は変色していた.}$_{(\mathrm{a})}$\\
 この銅線を再びガスバーナーで熱したあと,すぐに試験官に入れてメタノールの液面に近づけたところ,\underline{銅線は元の色に戻った.}$_{(\mathrm{b})}$この操作を繰り返して,\underline{刺激臭のある化合物Aを得た.}$_{(\mathrm{c})}$\\
 Aは\fbox{ア}性を示し,Aを含む水溶液をフェーリング液に加えて加熱すると,\fbox{イ}色の\fbox{ウ}が沈殿する.また,アンモニア性硝酸銀水溶液に加えて加熱すると,\fbox{エ}反応がみられる.
\begin{itemize}
    \item [(1)]文中の\fbox{ }に適切な語句,物質名を入れよ.
    \item [(2)]下線部(a)で,銅線は何色に変色したか.また,このときに銅線の表面に生成した物質はなにか.
    \item [(3)]下線部(c)で生じた化合物の構造式と名称を示せ.
    \item [(4)]下線部(b),(c)の変化を1つの化学反応式で示せ.
\end{itemize}
\end{que}
\ans 
\begin{itemize}
    \item[(1)]\fbox{ア}:還元  \fbox{イ}:赤  \fbox{ウ}:酸化銅(I) \fbox{エ}:銀鏡\\[20pt]
    \item [(2)]色:黒色           物質名:酸化銅(II)\\
    \item [(3)]構造式:\chemfig{H-C(=[6]O)-H}          名称:ホルムアルデヒド\\[10pt]
    \item [(4)]まずは半反応式を書くと,\begin{itemize}
        \item [(b)]\ce{2CuO +2H+ + 2e^- -> Cu2O + H2O}
    \item[(c)]\ce{CH3OH  -> HCHO + + 2H+ + 2e^-}
    \end{itemize}
   なので,これらを合わせて\ce{2CuO + CH3OH -> Cu2O + H2O}を得る.
\end{itemize}

\newpage
\begin{que}
次の(1)〜(3)それぞれに当てはまるものを全て選び,記号で答えよ.
\begin{itemize}
    \item [(1)]分子内脱水反応を起こすもの\\
    (a)フタル酸 (b)テレフタル酸 (c)酢酸 (d)マレイン酸 \\(e)フマル酸 (f)エタノール
    \item[(2)]ヨードホルム反応を示すもの\\
     (a)メタノール (b)エタノール (c)ホルムアルデヒド (d)アセトン\\
     (e)アセトアルデヒド (f)2-プロパノール
\end{itemize}
\end{que}
\ans 
\begin{itemize}
    \item [(1)]~(a),(d),(f) \\[20pt]
    \item [(2)]\chemfig{CH3-C(=[6]O)-H}または$\chemfig{CH3-C(=[6]O)-C-}$,あるいは酸化によりこれを生じるものを選べばよいので,(b),(d),(e),(f).
\end{itemize}
\newpage
\begin{que}
次の文章の\fbox{ }に適切な物質名,語句を入れよ.
\begin{itemize}
    \item[(1)] \fbox{ア}は食酢の主成分で,アセトアルデヒドを\fbox{イ}して得られる無色・刺激臭の液体である.水溶液は\fbox{ウ}性を示し,その強さは炭酸と比べて\fbox{エ}.そのため,炭酸水素ナトリウム水溶液に加えると\fbox{オ}を発生する.純度の高い\fbox{ア}は室温が下がると凝固するので,\fbox{カ}と呼ばれる.また,\fbox{ア}を強い脱水剤で脱水すると,\fbox{キ}を生じる.
    \item [(2)]ギ酸はカルボキシ基とともに\fbox{ク}基を含むため\fbox{ケ}性質を示し,アンモニア性硝酸銀水溶液から\fbox{コ}を析出させる.この反応を\fbox{サ}という.
\end{itemize}
\end{que}
\ans 
\begin{minipage}{0.5\linewidth}
\noindent (1)\begin{itemize}
   \item [\fbox{ア}]:酢酸\\
  \item [\fbox{イ}]:酸化\\
  \item [\fbox{ウ}]:弱酸\\
  \item [\fbox{エ}]:強い\\
  \item [\fbox{オ}]:二酸化炭素\\
  \item [\fbox{カ}]:氷酢酸\\
  \item [\fbox{キ}]:無水酢酸
\end{itemize}
\end{minipage}
\begin{minipage}{0.5\linewidth}
\noindent (2)
\begin{itemize}
    \item[\fbox{ク}]:ホルミル基\\
      \item [\fbox{ケ}]:還元\\
  \item [\fbox{コ}]:銀\\
  \item [\fbox{サ}]:銀鏡\\
    \item [ ] \\
  \item [ ] \\
  \item [ ] 
\end{itemize}
\end{minipage}
\newpage
\begin{que}
分子式\ce{C3H8O}で表される化合物A,B,Cがある.AとBはナトリウムと反応して気体を発生するが,Cは反応しない.また,AとBを穏やかに酸化すると,Aからは化合物Dが,Bからは化合物Eが得られた.DとEに銀鏡反応を試みたところ,Eだけが銀鏡を生成した.
\begin{itemize}
    \item [(1)]化合物A〜Eの構造式を示せ.
    \item [(2)]化合物A〜Cのうち,濃硫酸と加熱すると脱水してプロピレンを生じるものはどれか.
    \item [(3)]化合物A〜Eのうち,ヨードホルム反応を示すものはどれか.
    \item [(4)]化合物A〜Eのうち,フェーリング反応を示すものはどれか.
\end{itemize}
\end{que}
\ans 
\begin{itemize}
    \item[(1)](A)\chemfig{C-C(-[6]OH)-C}           (B)\chemfig{C-C-C-OH}\\[70pt]
         (C)\chemfig{C-C-O-C}          (D)\chemfig{C-C(=[6]O)-C}\\[70pt]
            (E)\chemfig{C-C-C(=[6]O)-H}\\
    \item[(2)]A,B\\[10pt]
    \item[(3)]A,D \\[10pt]
    \item[(4)]E
\end{itemize}
\begin{que}
分子式\ce{C4H10O2}のXは2価アルコール,つまりヒドロキシ基を2つ持つアルコールである.Xを穏やかに二クロム酸カリウムの希硫酸溶液で酸化すると,分子式\ce{C4H8O2}のYが生成する.Yにフェーリング液を加えて加熱すると,赤色の沈殿が生じる.Yをさらに酸化すると,分子式\ce{C4H8O3}の化合物が生じる.\underline{Zを炭酸水素ナトリウム水溶液に加えると,発泡して溶解する.}
\begin{itemize}
    \item [(1)]XとYの構造式を示せ.
    \item [(2)]下線部で発生した気体は何か.
\end{itemize}
\end{que}
\ans
最初の酸化で\ce{H}が2つだけ減ったので,酸化されたヒドロキシ基は1つである.したがって,3級ヒドロキシ基を持つことに注意する.よってXの構造式は
\begin{figure}[H]
    \centering
    \chemfig{C-C(-[2]C)(-[6]OH)-C-OH}
\end{figure}
に決まる.これを穏やかに酸化するとアルデヒドで止まるので,Yの構造式は
\begin{figure}[H]
    \centering
    \chemfig{C-C(-[2]C)(-[6]OH)-C(=[6]O)-H}
\end{figure}
と決まる.Zはカルボン酸なので,炭酸より強い.よって,弱酸の遊離反応で二酸化炭素が発生する.
\newpage
    \begin{que}
    酢酸とエタノールの混合物に少量の濃硫酸を加えて温めると,\fbox{ア}が生じる:\\
\centerline{\ce{\fbox{イ} + \fbox{ウ} <=> \fbox{エ} + H2O}}
この反応を\fbox{オ}といい,反応で生じる水の酸素原子は\fbox{イ}から脱離したものである.\fbox{ア}は水よりも軽く,水に溶け\fbox{カ}い液体で芳香がある.主に溶剤として用いられる.
\begin{itemize}
    \item [(1)]文中の\fbox{ }を埋めよ.ただし,\fbox{イ},\fbox{ウ},\fbox{エ}には示性式を記せ.
    \item [(2)]\fbox{ア}に塩酸を加えて加熱したときの反応を化学反応式で示せ.
    \item [(3)]\fbox{ア}に水酸化ナトリウム水溶液を加えて加熱したときの反応を化学反応式で示せ.
    \item [(4)](2),(3)の反応を何というか.
\end{itemize}
\end{que}
\ans 
\begin{itemize}
    \item [(1)]\fbox{ア}:酢酸エチル  \fbox{イ}:\ce{CH3COOH}  \fbox{ウ}:\ce{C2H5OH}  \fbox{エ}:\ce{CH3COOC2H5}  \\\fbox{オ}:エステル化  \fbox{カ}:にく\\
    \item[(2)]\ce{CH3COOC2H5 + H2O -> CH3COOH + C2H5OH}\\[10pt]
    \item[(3)]\ce{CH3COOC2H5 + NaOH -> CH3COOH + C2H5ONa}\\[10pt]
    \item[(4)](1)加水分解   (2)けん化  
\end{itemize}
\newpage
\begin{que}
元素の質量百分率が炭素54.5$\%$,水素9.1$\%$で,分子量が88.0のエステルAがある.Aを加水分解するとカルボン酸とアルコールが生じた.
\begin{itemize}
    \item[(1)]Aの分子式を求めよ.
    \item [(2)]加水分解により生じたカルボン酸が銀鏡反応を示した.このとき考えられるAの構造異性体は何種類か.
    \item [(3)]加水分解に生じたアルコールを酸化したところ,その生成物は銀鏡反応を示した.Aの構造式を描け.
\end{itemize}
\end{que}
\ans \noindent (1)\ce{C4H8O2}    (2)2種類    (3)\chemfig{H-C(=[6]O)-O-C-C-C}
\newpage
\begin{que}
分子式\ce{C3H6O2}で表される化合物A,B,Cがある.Aは水によく溶け,水溶液は酸性であった.BとCはエステル結合を持ち,それぞれを加水分解したところ,Bからは化合物Dと水溶液が酸性を示す化合物Eが,Cからは化合物Fと銀鏡反応を示す化合物Gが得られた.
\begin{itemize}
    \item [(1)]A,B,C,E,Gの構造式を示せ.
    \item [(2)]A〜Fのうち,酸化されるとアルデヒドになるものをすべて答えよ.
    \item [(3)]A〜Fのうち,ヨードホルム反応を示すものをすべて答えよ.
\end{itemize}
\end{que}
\ans 
\begin{itemize}
    \item [(1)]Aはカルボン酸なので,プロピオン酸\ce{CH3CH2COOH}である.B,Cはエステルなので加水分解でカルボン酸とアルコールが生じている.まずはCから決めるといい.
\begin{itemize}
    \item Cの観察\\
    Gは銀鏡反応を示すのでホルミル基を持つ.カルボン酸アルコールでホルミル基を持ちうるのは,ギ酸のみである.よって,\underline{Gはギ酸}に決まる.炭素数を考えて\underline{Fはエタノール}.
    \item Bの観察\\
残る候補は酢酸のエステルのみである.よって,\underline{Eは酢酸,Dはメタノール}である.  
\end{itemize}
面倒なので構造式は省略する.
    \item [(2)]酸化でアルデヒドになるのは第一級アルコールのメタノールとエタノール.よって,\underline{D,F}.
    \item [(3)]酢酸エステルはヨードホルム反応を示さないことを注意する.それぞれ構造式を確認して,やはりヨードホルム反応を示すのは\underline{Fのエタノールのみ.}
\end{itemize}

\newpage
\begin{que}
(リードアルファ329)
\begin{itemize}
    \item [(1)]エタノールに当てはまり,フェノールに当てはまらない性質を次から選べ.
\item [(2)]フェノールに当てはまり,エタノールに当てはまらない性質を次から選べ.
\item [(3)]フェノールとエタノールの両方に当てはまる性質を次から選べ.
\end{itemize}
\begin{align*}
&(あ)水によく溶ける&&(い)酸化するとアルデヒドを生じる\\
&(う)ヒドロキシ基を持っている&&(え)水溶液は塩基性である\\
&(お)水溶液は酸性である&&(か)塩基と反応して塩を作る\\
&(き)エステルを作る&&(く)酸化鉄(\text{III})で呈色する.
\end{align*}
\end{que}
\ans 
リードアルファで確認してください.
\newpage
\begin{que}  フェノールはベンゼン環に\fbox{あ}基がついた\fbox{い}酸で,水酸化ナトリウム水溶液に溶けて\fbox{う}となる.この水溶液に二酸化炭素を吹き込むと,炭酸はフェノールよりも\fbox{え}い酸なので,\fbox{お}反応によりフェノールが得られる.\\
  ベンゼン環に直接結合したヒドロキシ基は\fbox{か}と呼ばれ,アルコールとは異なる性質を示す.これを検出するには,\fbox{き}水溶液に加えて色が\fbox{く}〜\fbox{け}に変化することを確認すればよい.\\
  フェノールの代表的な製法である\fbox{こ}法では,プロピレンへのベンゼンの付加反応により生じる\fbox{さ}を酸化して得られる\fbox{し}を硫酸で分解してフェノールを得る.このとき,副産物として\fbox{す}も得られる.\\
  また,ベンゼンと濃硫酸を加熱することで得られる\fbox{せ}を中和した後,水酸化ナトリウムと融解することで\fbox{そ}が生じる.これを酸性にすることで,\fbox{た}反応によりフェノールが得られる.\\
 フェノールはベンゼンと比べて\fbox{ち}反応を受けやすい.例えば,フェノールに十分量の臭素水を加えると\fbox{つ}の白色沈殿が生じる.
 \begin{itemize}
    \item [(1)]文中に当てはまる語句などを答えよ.
    \item [(2)]\fbox{し}と\fbox{つ}の構造式を記せ.
 \end{itemize}
\end{que}
\ans 
\noindent (1)\begin{itemize}
   \item [\fbox{あ}]:ヒドロキシ\\
  \item [\fbox{い}]:弱酸\\
  \item [\fbox{う}]:ナトリウムフェノキシド\\
  \item [\fbox{え}]:強\\
  \item [\fbox{お}]:弱酸遊離\\
  \item [\fbox{か}]:フェノール性ヒドロキシ基\\
  \item [\fbox{き}]:塩化鉄(III)
  \item [\fbox{く}]:赤紫
\item [\fbox{け}]:青紫\\
  \item [\fbox{こ}]:クメン\\
  \item [\fbox{さ}]:クメン\\
  \item [\fbox{し}]:クメンヒドロペルオキシド\\
  \item [\fbox{す}]:アセトン\\
\item [\fbox{せ}]:ベンゼンスルホン酸\\
  \item [\fbox{そ}]:ナトリウムフェノキシド\\
  \item [\fbox{た}]:弱酸遊離\\
  \item [\fbox{ち}]:置換\\
  \item [\fbox{つ}]:2,4,6-トリブロモフェノール 
\end{itemize}
(2)略.リードアルファとかに載ってます.
\end{document}