\documentclass[a4paper,12pt]{ltjsreport}
\usepackage{base}
\setchemfig{atom sep=2em}
\title{}
\author{}
\date{}
\begin{document}

\subsection*{練習問題}
\begin{que}
次のアルコールの名称を答えよ.また,級数でアルコールを分類せよ.\\[5pt]
\begin{minipage}{0.5\linewidth}
    \begin{itemize}
        \item [(1)]\chemfig{CH3-CH2-CH2-CH2-OH}\\
        \item [(3)]\chemfig{CH3-CH(-[6]OH)-CH3}\\[5pt]
        \item [(5)]\chemfig{CH3-C(-[2]CH3)(-[6]OH)-CH3}
    \end{itemize}
\end{minipage}
\begin{minipage}{0.5\linewidth}
\begin{itemize}
    \item [(2)]\chemfig{CH3-CH2-OH}\\
    \item [(4)]\chemfig{CH3-CH(-[6]OH)-CH(-[6]OH)-CH3}\\[5pt]
    \item [(6)]\chemfig{CH3-C(-[2]H)(-[6]OH)-C(-[2]CH3)(-[6]H)-CH2-OH}
\end{itemize}
\end{minipage}
\end{que}
\ans
\begin{itemize}
    \item [(1)] \\
    \item [(2)] \\
     \item [(3)] \\
      \item [(4)] \\
       \item [(5)] \\
        \item [(6)] \\
\end{itemize}
    第1級アルコール: \\[7pt]

   \noindent  第2級アルコール: \\[7pt]

    \noindent 第3級アルコール:\newpage
    \begin{que}
        この問題では,\ce{H}$=$1.0,\ce{C}$=$12,\ce{O}$=$16,\ce{Na}$=$23とする.
    \begin{itemize}
        \item [(1)]エタノールと単体ナトリウムの反応を化学反応式で示せ.また,ナトリウムを含む生成物の名称を答えよ.
    \end{itemize}
            あるアルコール\ce{C}$_{x}$\ce{H}$_{y}$\ce{OH}1.5gを十分な量のナトリウムと反応させたところ,標準状態で280mLの気体が発生した.
            \begin{itemize}
                \item [(2)]$x,y$を決定せよ.
                \item [(3)]アルコールの構造の候補をすべて構造式で記せ.
            \end{itemize}
    \end{que}
    \ans
    \begin{itemize}
        \item [(1)]反応式:\\[15pt]
         名称:\\[10pt]
        \item[(2)]計算:\\[70pt]$                        \underline{x=~~~~~~~~~~~~~~~y=~~~~~~~~~~~~}$\\[10pt]
        \item[(3)] 
    \end{itemize}
    \newpage
    \begin{que}
      2-ペンタノールを濃硫酸で160C$^\circ$まで加熱し,脱水させた.
      \begin{itemize}
        \item [(1)]2-ペンタノールの構造式を示せ.
        \item [(2)]この脱水反応の反応式を示し,生成した炭化水素の構造式と名称を答えよ.
      \end{itemize}
    \end{que}
    \ans
    \begin{itemize}
        \item [(1)] \chemfig{CH3-CH(-[6]OH)-CH2-CH2-CH3}\\[70pt]
        \item [(2)]反応式:\ce{C5H11OH -> C5H10 + H2O}\\[30pt]
         名称:ペンテン\\[30pt]
        構造式:\chemfig{C=C-C-C-C}
    \end{itemize}
  \newpage
        \begin{que}
        \begin{itemize}
            \item [(1)]次のエーテルの名称を答えよ.\\
            
            \noindent(a)~\chemfig{([:-30]*6(-=-(-O-(*6(-=-=-=)))=-=))}~~~~~~~(b)~\chemfig{H-C(-[2]H)(-[6]H)-O-C(-[2]H)(-[6]H)-H}\\[5pt]
            (c)~\chemfig{H-C(-[2]H)(-[6]H)-C(-[2]H)(-[6]H)-C(-[2]H)(-[6]H)-C(-[2]H)(-[6]H)-O-C(-[2]H)(-[6]H)-H}\\
            \item [(2)]次の化合物の構造式を描け.\\
            (a)~エチルブチルエーテル~~~~~~~~~~(b)~ジプロピルエーテル~~\\
            (c)~エチルヘキシルエーテル
        \end{itemize}
        \end{que}
        \ans
        \begin{itemize}
            \item [(1)](a)ジフェニルエーテル~~~~~~~~~~~~~~~~~~~~~~~~~~~~~~~(b)ジメチルエーテル~~~~~~~~~~~~~~~~~\\[15pt]
            (c)ブチルメチルエーテル\\[10pt]
            \item[(2)](a)\chemfig{CH3-CH2-O-CH2-CH2-CH2-CH3}\\[70pt]
            (b)\chemfig{CH3-CH2-CH2-O-CH2-CH2-CH3}\\[70pt]

            (c)\chemfig{CH3-CH2-O-CH2-CH2-CH2-CH2-CH2-CH3}
        \end{itemize}
        \newpage
        \begin{que}
        アルコール\ce{C}$_{x}$\ce{H}$_{2x+1}$\ce{OH}を54.76g用意し,濃硫酸で130C$^\circ$程度に加熱して脱水させたところ,$6.66$gの水が生じた.
      \begin{itemize}
        \item [(1)]この脱水反応の反応式を示せ.
        \item [(2)]$x$を求めよ.
        \item [(3)]\ce{C}$_{x}$\ce{H}$_{2x+2}$および生成した有機化合物の名称を答えよ.
      \end{itemize}
        \end{que}
        \ans
        \begin{itemize}
            \item [(1)]2\ce{C}$_{x}$\ce{H}$_{2x+1}$\ce{OH}$~\longrightarrow~$\ce{C}$_{x}$\ce{H}$_{2x+1}$\ce{-O - }\ce{C}$_{x}$\ce{H}$_{2x+1}$\ce{ + H2O}\\[15pt]
            \item [(2)]計算:生成した水は$6.66/18$molなので,反応したアルコールは$6.66/18\times 2$molである.よって,分子量を$M$とすると,
            \[\frac{54.76}{M}=\frac{6.66}{18}\times2\]
            である.よって,$M=54.76/0.74=74$であり,$x=~4$を得る.\\
            \rightline{\underline{$x=4$       }}\\
            \item [(3)] \ce{C}$_{x}$\ce{H}$_{2x+2}$ の 名 称:ブタン\\[35pt]
        生成した有機化合物の名称:ジブチルエーテル
        \end{itemize}
\newpage
\begin{que}
\begin{itemize}
    \item [(1)]ホルムアルデヒドとアセトンの構造式を描け.
    \item [(2)]次の記述のうち,ホルムアルデヒドのみに当てはまる性質には◯,アセトンのみに当てはまる性質には△,両方に当てはまる性質には☆を,どちらにも当てはまらない性質には$\times$をつけよ.
    \begin{align*}
    &(\text{a})酸化するとカルボン酸になる&&(\text{b})常温で液体である.\\
    &(\text{c})水によく溶ける.&&(\text{d})酸性を示す.\\
    &(\text{e})フェーリング液を還元する.&&(\text{f})銀鏡反応を示す.\\
    &(\text{g})結合~\chemfig{-C(=[6]O)-}~を持つ.&&(\text{h})還元するとアルコールになる.
    \end{align*}
\end{itemize}
\end{que}
\ans 
\noindent (a)         (b)         (c)         (d)         \\
(e)         (f)         (g)         (h)         
\newpage
\begin{que}
磨いた銅線をらせん状に巻いてガスバーナーで熱した.動線を炎から出し,冷却したあとに観察すると,\underline{銅線は変色していた.}$_{(\mathrm{a})}$\\
 この銅線を再びガスバーナーで熱したあと,すぐに試験官に入れてメタノールの液面に近づけたところ,\underline{銅線は元の色に戻った.}$_{(\mathrm{b})}$この操作を繰り返して,\underline{刺激臭のある化合物Aを得た.}$_{(\mathrm{c})}$\\
 Aは\fbox{ア}性を示し,Aを含む水溶液をフェーリング液に加えて加熱すると,\fbox{イ}色の\fbox{ウ}が沈殿する.また,アンモニア性硝酸銀水溶液に加えて加熱すると,\fbox{エ}反応がみられる.
\begin{itemize}
    \item [(1)]文中の\fbox{ }に適切な語句,物質名を入れよ.
    \item [(2)]下線部(a)で,銅線は何色に変色したか.また,このときに銅線の表面に生成した物質はなにか.
    \item [(3)]下線部(c)で生じた化合物の構造式と名称を示せ.
    \item [(4)]下線部(b),(c)の変化を1つの化学反応式で示せ.
\end{itemize}
\end{que}
\ans 
\begin{itemize}
    \item[(1)] \\[20pt]
    \item [(2)]色:             物質名:\\
    \item [(3)]構造式:                    名称:\\[70pt]
    \item [(4)](b)\\[20pt]
    (c)
\end{itemize}

\newpage
\begin{que}
次の(1)〜(3)それぞれに当てはまるものを全て選び,記号で答えよ.
\begin{itemize}
    \item [(1)]分子内脱水反応を起こすもの\\
    (a)フタル酸 (b)テレフタル酸 (c)酢酸 (d)マレイン酸 \\(e)フマル酸 (f)エタノール
    \item[(2)]ヨードホルム反応を示すもの\\
     (a)メタノール (b)エタノール (c)ホルムアルデヒド (d)アセトン\\
     (e)アセトアルデヒド (f)2-プロパノール
\end{itemize}
\end{que}
\ans 
\begin{itemize}
    \item [(1)]~(a),(d),(f) \\[20pt]
    \item [(2)]\chemfig{CH3-C(=[6]O)-}または$\chemfig{CH3-CH(-[6]OH)-}$を持つものを選べばよいので,(b),(d),(e),(f).
\end{itemize}
\newpage
\begin{que}
次の文章の\fbox{ }に適切な物質名,語句を入れよ.
\begin{itemize}
    \item[(1)] \fbox{ア}は食酢の主成分で,アセトアルデヒドを\fbox{イ}して得られる無色・刺激臭の液体である.水溶液は\fbox{ウ}性を示し,その強さは炭酸と比べて\fbox{エ}.そのため,炭酸水素ナトリウム水溶液に加えると\fbox{オ}を発生する.純度の高い\fbox{カ}は室温が下がると凝固するので,\fbox{キ}と呼ばれる.また,\fbox{ク}を強い脱水剤で脱水すると,\fbox{ケ}を生じる.
    \item [(2)]ギ酸はカルボキシ基とともに\fbox{コ}基を含むため\fbox{サ}性質を示し,アンモニア性硝酸銀水溶液から\fbox{シ}を析出させる.この反応を\fbox{ス}という.
\end{itemize}
\end{que}
\ans 
\begin{minipage}{0.5\linewidth}
\noindent (1)\begin{itemize}
   \item [\fbox{ア}]:酢酸\\
  \item [\fbox{イ}]:酸化\\
  \item [\fbox{ウ}]:弱酸\\
  \item [\fbox{エ}]:強い\\
  \item [\fbox{オ}]:二酸化炭素\\
  \item [\fbox{カ}]:酢酸\\
  \item [\fbox{キ}]:氷酢酸\\
  \item [\fbox{ク}]:酢酸\\
  \item [\fbox{ケ}]:無水酢酸
\end{itemize}
\end{minipage}
\begin{minipage}{0.5\linewidth}
\noindent (2)
\begin{itemize}
    \item[\fbox{コ}]:ホルミル基\\
      \item [\fbox{サ}]:還元\\
  \item [\fbox{シ}]:銀\\
  \item [\fbox{ス}]:銀鏡\\
    \item [ ]:\\
  \item [ ] \\
  \item [ ] \\
  \item [ ] \\
  \item [ ] 
\end{itemize}
\end{minipage}
\newpage
\begin{que}
分子式\ce{C3H8O}で表される化合物A,B,Cがある.AとBはナトリウムと反応して気体を発生するが,Cは反応しない.また,AとBを穏やかに酸化すると,Aからは化合物Dが,Bからは化合物Eが得られた.DとEに銀鏡反応を試みたところ,Eだけが銀鏡を生成した.
\begin{itemize}
    \item [(1)]化合物A〜Eの構造式を示せ.
    \item [(2)]化合物A〜Cのうち,濃硫酸と加熱すると脱水してプロピレンを生じるものはどれか.
    \item [(3)]化合物A〜Eのうち,ヨードホルム反応を示すものはどれか.
    \item [(4)]化合物A〜Eのうち,フェーリング反応を示すものはどれか.
\end{itemize}
\end{que}
\ans 
\begin{itemize}
    \item[(1)](a)\hspace{180pt}(b)\\[70pt]
         (c)\hspace{180pt}(d)\\[70pt]
            (e)\\
    \item[(2)] \\[20pt]
    \item[(3)] \\[20pt]
    \item[(4)] \\[20pt]
\end{itemize}
\begin{que}
分子式\ce{C4H10O2}のXは2価アルコール,つまりヒドロキシ基を2つ持つアルコールである.Xを穏やかに二クロム酸カリウムの希硫酸溶液で酸化すると,分子式\ce{C4H8O2}のYが生成する.Yにフェーリング液を加えて加熱すると,赤色の沈殿が生じる.Yをさらに酸化すると,分子式\ce{C4H8O3}の化合物が生じる.\underline{Zを炭酸水素ナトリウム水溶液に加えると,発泡して溶解する.}
\begin{itemize}
    \item [(1)]XとYの構造式を示せ.
    \item [(2)]下線部で発生した気体は何か.
\end{itemize}
\end{que}
\ans 
\begin{itemize}
    \item [(1)]X:\hspace{200pt}Y:\\[100pt]
    \item [(2)]
\end{itemize}
\newpage
\section{エステル}
エステル結合\ce{-COO-R-}を持つ化合物をエステルという.
\subsubsection*{エステルの例と命名法}
エステル\ce{R1-COO-R2}は,「(カルボン酸\ce{R1-COOH})$+$(炭化水素基R$_2$)$+$エステル」と命名される.
\begin{table}[H]
    \centering
  \begin{tabular}{|c|c|c|}
   \hline
    構造と名称&\ce{R1COOH}&R$_2$\\
\hline     \chemfig{CH3-C(=[6]O)-O-CH2CH3}
    &\ce{CH3CH2-COOH}
    &\ce{CH3CH2 -}\\
    酢酸エチル&酢酸&エチル(ethyl)基\\
    \hline
     & & \\
\chemfig{H-C(=[6]O)-O-CH3}
    &\ce{H-COOH}
    &\ce{CH3 -}\\
ギ酸メチル &ギ酸&メチル (methyl )基\\
    \hline
       \chemfig{*6(=-(-[::-30]OH)=(-[::-90]COOCH3)-=-)}
    &\ce{   \chemfig{*6(=-(-[::-30]OH)=(-[::-90]COOH)-=-)}}
    &\ce{CH3 -}\\
サリチル酸メチル &サリチル酸&メチル(methyl)基\\
    \hline
    \end{tabular}
\end{table}
\subsubsection*{エステルの製法と加水分解}
カルボン酸とアルコールの脱水反応で得られる.この反応を{\color{red}\textbf{エステル化}}という.このとき,{\color{red}\textbf{カルボン酸から\ce{-OH}が,アルコールから\ce{-H}が脱離している.}}また,エステルに希塩酸や希硫酸を加えて加熱すると逆向きの反応({\color{red}加水分解})が起こり,カルボン酸とアルコールが生じる.\\ 
{\centerline{\ce{R1 -COOH  + HO -R2 <=> R1 -COO -R2 + H2O}}}
\newpage
\subsubsection*{けん化}
エステルを強塩基で加水分解するとカルボン酸のナトリウム塩とアルコールが生成する.この反応を{\color{red}けん化}という.\\
 {\centerline{\ce{R1 -COO -R2 + NaOH -> R1 -COONa + HO -R2}}}
 特に,脂肪酸とグリセリンのエステル(油脂)をけん化するとセッケンが得られる(技術的な問題で反応式は準備中).
% \centerline{\chemfig{H-CH(-[6,1.5]CH(-[4]H)(-[6,1.5]CH(-[4]H)-O-C(=[6]O)-R3)-O-C(=[6]O)-R2)-O-C(=[2]O)-R1}~+~3\ce{NaOH}~$\longrightarrow$~\chemfig{H-CH(-[6,1.5]CH(-[4]H)(-[6,1.5]CH(-[4]H)-O-C(=[6]O)-R3)-O-C(=[6]O)-R2)-O-C(=[2]O)-R1}}
\subsubsection*{エステルの性質}
\begin{itemize}
    \item [(1)]\textbf{加水分解とけん化}\\
さっきやった.
    \item[(2)]\textbf{難溶性}\\
    水に溶けない.
    \item[(3)]\textbf{芳香}\\
    果物のようないい匂いがするらしい.
    \newpage
\end{itemize}
    \begin{que}
    酢酸とエタノールの混合物に少量の濃硫酸を加えて温めると,\fbox{ア}が生じる:\\
\centerline{\ce{\fbox{イ} + \fbox{ウ} <=> \fbox{エ} + H2O}}
この反応を\fbox{オ}といい,反応で生じる水の酸素原子は\fbox{イ}から脱離したものである.\fbox{ア}は水よりも軽く,水に\fbox{カ}い液体で芳香がある.主に溶剤として用いられる.
\begin{itemize}
    \item [(1)]文中の\fbox{ }を埋めよ.ただし,\fbox{イ},\fbox{ウ},\fbox{エ}には構造式を記せ.
    \item [(2)]\fbox{ア}に塩酸を加えて加熱したときの反応を化学反応式で示せ.
    \item [(3)]\fbox{ア}に水酸化ナトリウム水溶液を加えて加熱したときの反応を化学反応式で示せ.
    \item [(4)](2),(3)の反応を何というか.
\end{itemize}
\end{que}
\ans 
\begin{itemize}
    \item [(1)]\fbox{ア}\hspace{200pt}\fbox{イ}\\[50pt]
    \fbox{ウ}\hspace{200pt}\fbox{エ}\\[50pt]
    \fbox{オ}\hspace{200pt}\fbox{カ}\\[50pt]
    \item[(2)] \\[10pt]
    \item[(3)] \\[10pt]
    \item[(4)](1)\hspace{200pt}(2)  
\end{itemize}
\newpage
\begin{que}
元素の質量百分率が炭素54.5$\%$,水素9.1$\%$で,分子量が88.0のエステルAがある.Aを加水分解するとカルボン酸とアルコールが生じた.
\begin{itemize}
    \item[(1)]Aの分子式を求めよ.
    \item [(2)]加水分解により生じたカルボン酸が銀鏡反応を示した.このとき考えられるAの構造異性体は何種類か.
    \item [(3)]加水分解に生じたアルコールを酸化したところ,その生成物は銀鏡反応を示した.このとき考えられるAの構造異性体は何種類か.
\end{itemize}
\end{que}
\ans \noindent (1)\hspace{150pt}(2)\hspace{150pt}(3)
\newpage
\begin{que}
分子式\ce{C3H6O2}で表される化合物A,B,Cがある.Aは水によく溶け,水溶液は酸性であった.BとCはエステル結合を持ち,それぞれを加水分解したところ,Bからは化合物Dと水溶液が酸性を示す化合物Eが,Cからは化合物Fと銀鏡反応を示す化合物Gが得られた.
\begin{itemize}
    \item [(1)]A,B,C,E,Gの構造式を示せ.
    \item [(2)]A〜Fのうち,酸化されるとアルデヒドになるものをすべて答えよ.
    \item [(3)]A〜Fのうち,ヨードホルム反応を示すものをすべて答えよ.
\end{itemize}
\end{que}
\begin{itemize}
    \item [(1)]A:\hspace{200pt}B:\\[80pt]
    C:\hspace{200pt}E:\\[80pt]
    G:\\[80pt]
    \item [(2)] \\[15pt]
    \item [(3)]
\end{itemize}
\end{document}