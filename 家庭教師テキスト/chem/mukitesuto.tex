\documentclass[a4paper,twocolumn,11pt]{ltjsarticle}
\setlength{\columnseprule}{0.4pt}
\usepackage{base}
\title{}
\author{}
\date{}
\begin{document}
\noindent

\begin{itemize} 
    \item [(1)] 銅に希硝酸を加えたときの反応を化学反応式で記せ.\\[2cm]
    \item [(2)] ガラスの主成分はなにか.\\[2cm]
     \item [(3)] 塩化アンモニウムと水酸化カルシウムを混ぜて加熱したときの反応を化学反応式で記せ.\\[2cm] 
     \item [(4)] 硫酸の工業的製法をなんと言うか.\\[2cm] 
     \item [(5)] ブリキとトタンのうち,傷がつくと内部の鉄が錆びやすいのはどちらか?また,その理由を述べよ.\newpage
     \item [(6)] 銀と水酸化ナトリウムを反応させると生成する沈殿の名称と色を答えよ.\\[2cm] \item [(7)] ハロゲン元素を,原子番号の小さい順に4つあげよ.また,それぞれの常温での状態を述べよ.\\[5cm] 
     \item [(8)] アンモニアソーダ法では,
     \begin{itemize} \item [①]食塩水にアンモニアを吸収させ,二酸化炭素を通す. \item[②]反応①で得られる沈殿を加熱する. \end{itemize} 
     の反応が起きている.それぞれの反応を化学反応式で表せ.また,2つの反応式をまとめた反応式も作れ.\newpage
      \item [(9)] 蛍石に濃硫酸を加えたときの反応を化学反応式で記せ.\\[2cm] 
      \item [(10)] \ce{Sr},~\ce{Li},~\ce{Ca},~\ce{Na},~\ce{Ba},~\ce{Cu},~\ce{K}の炎色反応は何色かを順番に答えよ\\[2cm] 
       \item [(11)] \ce{Cu(OH)2}は何色か.\\[2cm] 
       \item [(12)] 石灰水に少量の二酸化炭素を吹き込んだときの反応を化学反応式で記せ.\\[2cm]
       \item [(13)] ハロゲン元素を,反応性が高い順に並べよ.\newpage
       \item [(14)] \ce{Cu(OH)2}の沈殿を溶解させるためには,どのような試薬を入れればいいか.また,溶解した際に生じる錯イオンの化学式と名称を答えよ.\\[4cm] 
       \item [(15)] 硫化鉄(II)の希塩酸を加えたときの反応を化学反応式で記せ.\\[2cm] 
       \item [(16)] 鉄にスズをメッキしたものをなんというか.\\[2cm]
        \item [(17)] 濃硫酸が弱酸性である理由を説明せよ.\newpage
        \item [(18)] 炭酸カルシウムに過剰の二酸化炭素を吹き込んだときの反応を化学反応式で記せ.\\[2cm]
         \item [(19)] フッ化水素酸をガラス容器で保存できない理由を述べよ.\\[2cm]
          \item [(20)] \ce{Al(OH)3}は何色か.また,これを錯イオンにして溶かすためには,どんな試薬を加えればよいか.\\[2cm] 
           \item [(21)] ハロゲン化水素の水溶液のうち,弱酸であるものを答えよ.\\[2cm]
            \item [(22)] 酸化マンガン(IV)に濃塩酸を加えて加熱したときの反応を化学反応式で記せ.\\[2cm] 
            \item [(23)] アンモニアソーダ法(ソルベー法)で合成される物質を答えよ.\newpage
            \item [(24)] 炭酸カルシウムに塩酸を加えたときの反応を化学反応式で記せ.\\[2cm] \item [(25)] アンモニアの工業的製法をなんと言うか.また,その反応式と必要な触媒を答えよ.\\[4cm] 
            \item [(26)] \ce{Zn(OH)2}は何色か.また,これを錯イオンにして溶かすためには,どんな試薬を加えればよいか.\\[2cm] 
          \item [(27)] 両性金属と呼ばれる元素をすべてあげよ.\\[1cm] 
            \item [(28)] 硝酸の工業的製法をなんと言うか.また,この製法によりアンモニアから硝酸を製造する過程を,3本の化学反応式で示せ.\newpage 
            \item [(29)] 銅に濃硫酸を加えて加熱したときの反応を化学反応式で記せ.\\[2cm] 
            \item [(30)] 鉄に亜鉛をメッキしたものをなんというか.\\[2cm] 
            \item [(31)] 接触法により硫黄から硫酸を製造する過程を,3本の化学反応式で示せ.\\[4cm]
             \item [(32)] 銅に濃硝酸を加えたときの反応を化学反応式で記せ.\\[2cm] \end{itemize}

\end{document}