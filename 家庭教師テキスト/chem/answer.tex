\documentclass[a4paper,11pt]{ltjsarticle}
\usepackage{base}
\title{}
\author{}
\date{}
\begin{document}
\pagestyle{empty}
\begin{itemize}
    \item アルコールはヒドロキシ基[  \textcolor{red}{\ce{-OH}}  ]を持つ化合物で,単体の[ \textcolor{red}{ナトリウム} ]と反応して,水素と[ \textcolor{red}{ナトリウムアルコキシド} ]を生成する.例えば,エタノールの反応は次の反応式で表される:\\
    \noindent [         \textcolor{red}{\ce{2C2H5OH + 2Na -> C2H5ONa + H2}}            ]\\
    \item メタノールと単体の[\textcolor{red}{ナトリウム}]を反応させると[\textcolor{red}{ナトリウムメトキシド}]が,エタノールと反応させると[\textcolor{red}{ナトリウムエトキシド}]が生じる.\\
    \item アルコールに濃硫酸を加えて160C$^\circ$程度に加熱すると[ \textcolor{red}{分子内} ]で脱水反応が起こり,[ \textcolor{red}{アルケン} ]が生じる.一方で,130C$^\circ$程度に加熱すると[ \textcolor{red}{分子間} ]で脱水反応が起こり,[ \textcolor{red}{エーテル} ]が生じる.\\
    \item アルコールはヒドロキシキが結合する炭素に結合する水素の数で1級,2級,3級アルコールに分類される.第1級アルコールを酸化すると[ \textcolor{red}{アルデヒド} ],[ \textcolor{red}{カルボン酸} ]の順に変化する.第2級アルコールは酸化すると[ \textcolor{red}{ケトン} ]になる.第3級アルコールは酸化されにくい.\\
    \item アルデヒドは[  \textcolor{red}{ホルミル}  ]基をもつ化合物で,第[ \textcolor{red}{1} ]級アルコールを酸化して得られる.[  \textcolor{red}{還元}  ]性を持ち,次の2つの検出法が使われる.\begin{itemize}
        \item アンモニア性硝酸銀水溶液にアルデヒドを加えて加熱すると,単体の銀が析出する\\([ \textcolor{red}{銀鏡} ]反応)
              \item フェーリング液にアルデヒドを加えて加熱すると,[ \textcolor{red}{赤} ]色の[ \textcolor{red}{酸化銅(I)} ]が沈殿する.(フェーリング反応)
    \end{itemize}
     \\
    \item ケトンは[ \textcolor{red}{ケトン} ]基をもつ化合物で,アルデヒドと異なり,[ \textcolor{red}{還元} ]性を持たない.メチル基を2つ持つケトンは[ \textcolor{red}{アセトン} ]と呼ばれ,有機溶媒として用いられる.\\
  
\end{itemize}
\newpage
\begin{itemize}
      \item ~[ \textcolor{red}{アセチル} ]基をもつ化合物はヨードホルム反応を示し,ヨウ素と水酸化ナトリウムを混ぜて加熱すると[ \textcolor{red}{ヨードホルム} ]の[ \textcolor{red}{黄} ]色沈殿を生じる.ただし,この反応は\ce{O}と二重結合している炭素に[ \textcolor{red}{炭素} ]原子または[ \textcolor{red}{水素} ]原子が結合している場合に限って起こる.よって,酢酸とエステルはヨードホルム反応を[ \textcolor{red}{示さない} ].\\
    \item カルボン酸は[ \textcolor{red}{カルボキシ} ]基を持つ化合物である.液性は[ \textcolor{red}{弱酸} ]性だが,炭酸よりは[ \textcolor{red}{強い} ].よって,炭酸水素ナトリウムにカルボン酸を加えると[ \textcolor{red}{弱酸の遊離} ]反応が起き,[ \textcolor{red}{二酸化炭素} ]が発生する.\\
    \item 2つのカルボキシ基の間で脱水反応が起こると,[ \textcolor{red}{酸無水物} ]が生じる.例えば,2価カルボン酸のフマル酸とマレイン酸のうち,[ \textcolor{red}{マレイン} ]は分子[ \textcolor{red}{内} ]で脱水反応をおこし,[ \textcolor{red}{無水マレイン酸} ]を生じる.分子内脱水を起こす他の例としては,ベンゼン環に2つのカルボキシ基が結合した[ \textcolor{red}{フタル酸} ]などがある.\\
    \item カルボン酸とアルコールで脱水反応を起こすと,[ \textcolor{red}{エステル} ]が生じる.このとき,[ \textcolor{red}{カルボン} ]から\ce{-OH}が脱離するのであった.この化合物はアセチル基を持つが,ヨードホルム反応を[ \textcolor{red}{示さない} ].\\
    \item エステルに酸や塩基を入れると[ \textcolor{red}{加水分解} ]が起こり,カルボン酸とアルコールが再生する.特に,塩基を使う場合は[ \textcolor{red}{けん} ]化と呼ばれ,カルボン酸はナトリウム塩の形で生じる.\\
    \item 4種類の異なる原子または原子団と結合している炭素原子を[ \textcolor{red}{不斉炭素原子} ]という.このような炭素原子を持つ化合物には[ \textcolor{red}{鏡像(光学)} ]異性体が存在する.
\end{itemize}
\end{document}
