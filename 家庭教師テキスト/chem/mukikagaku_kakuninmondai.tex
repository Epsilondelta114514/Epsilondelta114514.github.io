\documentclass[a4paper,twocolumn,11pt]{ltjsarticle}
\setlength{\columnseprule}{0.4pt}
\usepackage{base}
\title{}
\author{}
\date{}
\begin{document}
\twocolumn[\section{無機化学の確認問題(ざっくり!)}]
\noindent
\begin{itemize}
    \item [(1)]ハロゲン元素を,原子番号の小さい順に4つあげよ.\\[2cm]
    \item [(2)]ハロゲン元素で,最も反応性が強い元素はどれか.\\[2cm]
    \item [(3)]臭素は常温でどんな状態か.\\[2cm]
    \item [(4)]単体の塩素の実験室的製法を化学反応式で示せ.\\[2cm]
    \item [(5)]\ce{AgCl}は何色か.\\[2cm]
           \item [(6)]ハロゲン化水素の水溶液のうち,弱酸であるものを答えよ.\\[2cm]
\end{itemize}
\newpage
        \begin{itemize}
    \item [(7)]硫化水素により硫化物の沈殿を生じる金属イオンを3つ以上あげよ.\\[2cm]
            \item [(8)]硫酸の工業的製法をなんと言うか.\\[2cm]
    \item [(9)]接触法により硫黄から硫酸を製造する過程を,3本の化学反応式で示せ.\\[4cm]
    \item [(10)]濃硫酸が弱酸性である理由を説明せよ.\\[3cm]
    \item[(11)]アンモニアの工業的製法をなんと言うか.\\[2cm]
    \newpage
    \item[(12)]ハーバー・ボッシュ法の化学反応式を示せ.\\[2cm]
    \item[(13)]ハーバー・ボッシュ法で用いられる触媒を答えよ.\\[2cm]
    \item[(14)]硝酸の工業的製法をなんと言うか.\\[2cm]
    \item[(15)]オストワルト法によりアンモニアから硝酸を製造する過程を,3本の化学反応式で示せ.\\ [4cm]
    \item[(16)]塩酸,硝酸,硫酸のうち,酸化力が特に高いものをすべてあげよ.\\[2cm]
    \item[(17)]ガラスの主成分はなにか.
    \newpage
    \item[(18)]フッ化水素酸をガラス容器で保存できない理由を述べよ.\\[2cm]
    \item[(19)]\ce{Sr},~\ce{Li},~\ce{Ca},~\ce{Na},~\ce{Ba},~\ce{Cu},~\ce{K}の炎色反応は何色かを順番に答えよ\\[2cm]
    \item[(20)]アンモニアソーダ法(ソルベー法)で合成される物質を答えよ.\\[2cm]
    \item[(21)]アンモニアソーダ法では,\begin{itemize}
        \item [①]食塩水にアンモニアを吸収させ,二酸化炭素を通す.
        \item[②]反応①で得られる沈殿を加熱する.
    \end{itemize}
    の反応が起きている.それぞれの反応を化学反応式で表せ.また,2つの反応式をまとめた反応式も作れ.\newpage
\item[(22)]アルカリ土類金属と呼ばれる元素をすべてあげよ.\\[2cm]
\item[(23)]両性金属と呼ばれる元素を5つあげよ.\\[2cm]
\item[(24)]鉄にスズをメッキしたものをなんというか.\\[2cm]
\item[(25)]鉄に亜鉛をメッキしたものをなんというか.\\  [2cm]
\item[(27)]ブリキとトタンのうち,傷がつくと内部の鉄が錆びやすいのはどちらか?\\[2cm]
\item[(28)]\ce{Cu(OH)2}は何色か.\newpage
\item[(29)]  \ce{Cu(OH)2}に過剰のアンモニア水を加えると生成するイオンの名称と化学式を示せ.\\[3cm]
\item[(30)]\ce{Zn(OH)2}は何色か.また,これを錯イオンにして溶かすためには,どんな試薬を加えればよいか.\\[2cm]
\item[(31)]\ce{Al(OH)3}は何色か.また,これを錯イオンにして溶かすためには,どんな試薬を加えればよいか.\\[2cm]
\item[(32)]銀と水酸化ナトリウムを反応させると生成する沈殿の名称と色を答えよ.\\[2cm]
\item[(33)]銅に濃硝酸を加えたときの反応を化学反応式で記せ.\\[2cm]
\newpage 
\item[(34)]銅に希硝酸を加えたときの反応を化学反応式で記せ.\\[2cm]
\item[(35)]酸化マンガン(IV)に濃塩酸を加えて加熱したときの反応を化学反応式で記せ.\\[2cm]
\item[(36)]硫化鉄(II)の希塩酸を加えたときの反応を化学反応式で記せ.\\[2cm]
\item[(37)]蛍石に濃硫酸を加えたときの反応を化学反応式で記せ.\\[2cm]
\item[(38)]銅に濃硫酸を加えて加熱したときの反応を化学反応式で記せ.\\[2cm]
\newpage
\item[(39)] 塩化アンモニウムと水酸化カルシウムを混ぜて加熱したときの反応を化学反応式で記せ.\\[2cm]
\item[(40)]炭酸カルシウムに塩酸を加えたときの反応を化学反応式で記せ.\\[2cm]
\item[(41)]酢酸ナトリウムに塩酸を加えたときの反応を化学反応式で記せ.\\[2cm]       
 \item[(42)]石灰水に二酸化炭素を吹き込んだときの反応を化学反応式で記せ.\\[2cm]
 \item[(43)]炭酸カルシウムに過剰の二酸化炭素を吹き込んだときの反応を化学反応式で記せ.      
    \end{itemize}

\end{document}