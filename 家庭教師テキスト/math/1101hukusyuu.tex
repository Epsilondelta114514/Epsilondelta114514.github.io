\documentclass[a4paper,11pt]{ltjsarticle}
\usepackage{base}
\title{}
\author{}
\date{}
\newcommand{\printheader}[2]{
\begin{tikzpicture}[remember picture, overlay]
\node[yshift=-2.5cm, anchor=north] at (current page.north) {
\begin{tikzpicture}
\fill[gray!20] (0,0) rectangle (\textwidth, 2cm);
\fill[gray!80] (0,0) rectangle (0.2cm, 2cm);
\draw[gray!80, thick] (0,0) -- (	\textwidth, 0);
\node[anchor=west, text width=\textwidth-1cm, inner xsep=1cm] at (0, 1.25cm) {
\parbox[b]{\linewidth}{
{\color{gray!50!black}\bfseries #1} \par
\vspace{0.2em}
{\huge\bfseries #2}
}
};
\end{tikzpicture}
};
\end{tikzpicture}
\vspace{0.5cm}
}
\begin{document}
\printheader{11月1日}{前回の復習}
\begin{toi}
$a>0$を定数とし,曲線$C_1:y=x^2,~C_2:y=x^2-2ax+a^2+2a$を考える.
\begin{itemize}
    \item [(1)]$C_1,C_2$の共通接線$l$の方程式を求めよ.
    \item [(2)]$C_1,C_2,l$および$y$軸とで囲まれた部分の面積を求めよ.
\end{itemize}
\end{toi}
\begin{toi}
曲線$C:y=x^3-2x$を考える.
\begin{itemize}
    \item [(1)]点$(0,2)$を通り,曲線$C$に接する直線$l$の方程式を求めよ.
    \item [(2)]$C$と$l$とで囲まれた部分の面積を求めよ.
\end{itemize}
\end{toi}


\end{document}