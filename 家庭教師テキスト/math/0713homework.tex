\documentclass[a4paper,11pt]{ltjarticle}
\usepackage{base}
\title{}
\author{}
\date{}
\usepackage[top=15mm, bottom=15mm, left=25mm, right=25mm]{geometry} 
\title{7/13日の宿題}
\author{}
\date{}
\begin{document}
\pagestyle{empty}
\maketitle
\centerline{\large{\textbf{注意事項}}}
\begin{itemize}
    \item 問1〜問3に全て解答してください.
    \item わからない問題は調べても構いませんが,どの問題も必ず10分は考えるようにしてください.どの参考書にも類題は載っています.
    \item 解いた問題は,来週の授業で\textbf{\underline{私に向かって解説してもらいます.}} 要するに,先生役をやってもらいます.別の紙を使って構いませんから,他人から見やすいような答案を心がけてみてください.
    \item 答案の書き方はいつものように直していくので,最初はうまく書けなくてもよいです.しっかりとした答案を作ろうとすると,勝手に理解も深まります.
    \item 問3では図を描いたほうがいいと思います.
    \item がんば.
\end{itemize}
\newpage
\begin{que}[:ベクトルの内積]
\begin{itemize}
    \item [(1)]$\overrightarrow{a}=(\sqrt5,-2)$と直交する単位ベクトル$\overrightarrow{e}$を求めよ.
     \item [(2)] $\overrightarrow{a}=(1,\sqrt3),~\overrightarrow{b}=(-x,\sqrt6)$のなす角が$60^\circ$となる$x$を求めよ.
     \item [(3)]$\overrightarrow{\mathrm{OA}}=\overrightarrow{a},~\overrightarrow{\mathrm{OB}}=\overrightarrow{b}$とするとき,$\triangle $OABの面積$S$は
     \[S=\frac12\sqrt{|\overrightarrow{a}|^2|\overrightarrow{b}|^2-(\overrightarrow{a}\cdot\overrightarrow{b})^2}\]
     であることを証明せよ.
\end{itemize}
\end{que}
\ascboxF{\textbf{Hint.}}(3)まずは$S$を$|\overrightarrow{a}|,|\overrightarrow{b}|$と$\sin\theta$で表す.
\ans
\newpage
\begin{que}[:点の存在範囲]
$\triangle$OABで,辺OAを$2:1$に内分する点をM,辺OBを$3:2$に内分する点をN,2直線AN,~BMの交点をPとする.$\overrightarrow{\text{OA}}=\overrightarrow{a},~\overrightarrow{\text{OB}}=\overrightarrow{b}$として,$\overrightarrow{\text{OP}}=\overrightarrow{p}$を$\overrightarrow{a}$と$\overrightarrow{b}$で表せ.
\end{que}
\ascboxF{\textbf{Hint.}}$\mathrm{AP}:\mathrm{PN}=s:(1-s)$,$\mathrm{BP}:\mathrm{PM}=t:(1-t)$とおき,$\overrightarrow{p}$を具体的に表す.
\ans\newpage
\begin{que}
    $\triangle \mathrm{OAB}$において,$\mathrm{A}(\overrightarrow{a}),\mathrm{B}(\overrightarrow{b})$とする.実数$s,t$が$s+t=1,~s\geqq0~,t\geqq0$を満たしながら動くとき,
    \[\overrightarrow{p}=s\overrightarrow{a}+t\overrightarrow{b}\]
    を満たす点$\mathrm{P}(\overrightarrow{p})$の存在範囲を図示せよ.
\end{que}
\ascboxF{\textbf{Hint.}}$s+t=1$を用いて,$\overrightarrow{p}$を内分点の公式の形にできないか...?
\ans
\end{document}

