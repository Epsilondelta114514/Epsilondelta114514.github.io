\documentclass[a4paper]{ltjsarticle}
\usepackage{base}
\usepackage[top=15mm,bottom=15truemm,left=20truemm,right=20truemm]{geometry}
\title{}
\author{}
\date{}
\begin{document}
\pagestyle{empty}
\noindent \textbf{問題1}\\
$\triangle $ABCにおいて,$A=\dfrac23\pi$とする.
\begin{itemize}
    \item [(1)]$\sin B+\sin C$の取りうる値の範囲を求めよ.
    \item [(2)]$\sin B\sin C$の取りうる値の範囲を求めよ.
\end{itemize}

\noindent \textbf{問題2.}\\
$\triangle$ABCにおいて,
\[\frac{\sin C+\sin (A-B)}{\tan(B+C)}=-2\cos A\cos B\]
を示せ.

\noindent \textbf{問題3.}\\
$\displaystyle{0\leqq\theta\leqq\frac\pi2}$に対し,$f(\theta)=2\sin^2\theta -4\cos^2\theta +6\sqrt3\sin\theta\cos\theta$とする.
\begin{itemize}
    \item [(1)]$f(\theta)$を$\sin2\theta,\cos2\theta$の式で表せ.
    \item [(2)]$f(\theta)=a+r\sin(2\theta-\alpha)$を満たす実数$a,r,\alpha$を求めよ.ただし,$0\leqq \alpha\leqq\pi$とする.
    \item [(3)]$f(\theta)$の最大値と最小値を求めよ.
\end{itemize}

\noindent \textbf{問題4.}\\
$f(x)=4^x+4^{-x}-8(2^x+2^{-x})+10$とする.
\begin{itemize}
    \item [(1)]$t=2^x+2^{-x}$とおいて,$f(x)$を$t$の式で表せ.
    \item [(2)]$f(x)$の最小値と,そのときの$x$の値を求めよ.
\end{itemize}

\noindent \textbf{問題5.}\\
$f(x)=x^2+\displaystyle{\int_0^2xf(t)dt}$をみたす$f(x)$を求めよ.\\

\noindent \textbf{問題6.}\\
曲線$C_1:y=x^3$と曲線$C_2:y=x^2+x-1$を考える.
\begin{itemize}
    \item[(1)]$x^3>x^2+x-1$を解け.
    \item[(2)]$C_1$と$C_2$で囲まれた部分の面積を求めよ. 
\end{itemize}


\noindent \textbf{問題6.}\\
自然数$n$に対し,
$\displaystyle{\sum_{k=1}^n\frac{1}{\sqrt{k}}<2\sqrt n}$
であることを示せ.

\noindent \textbf{問題8.}\\
正の整数$k$に対して,$a_k$を$\sqrt{k}$に最も近い整数とする.例えば,$a_5=2,~a_8=3$である.
\begin{itemize}
    \item [(1)]$\displaystyle{\sum_{k=1}^{12}a_k}$を求めよ.
    \item [(2)]$\displaystyle{\sum_{k=1}^{2020}a_k}$を求めよ.
\end{itemize}
\noindent \textbf{問題9.}\\
袋の中に1から$n$までの番号がついた合計$n$個の玉が入っている.この袋から玉を1個取り出し,番号を調べてもとに戻す操作を$r$回行うとき,取り出された玉の番号の最大値を$X$とし,$X$の期待値を$E_n$とおく.
\begin{itemize}
    \item [(1)]$k=1,2,\ldots,n$に対して,$X=k$をとなる確率を求めよ.
    \item [(2)]$r=2$のとき,$E_n$を求めよ.
    \item [(3)]$\displaystyle{\lim_{n\to\infty}\frac{E_n}{n}}$を求めよ.
\end{itemize}
ヒント:区分求積法を用いる.\\



\end{document}