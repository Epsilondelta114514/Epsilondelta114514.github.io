\documentclass[a4paper,12pt]{ltjsarticle}
\usepackage{base}
\title{}
\author{}
\date{}
\begin{document}
\pagestyle{empty}
\ascboxC{\textbf{アセタール化の割合}}
ポリビニルアルコール中のヒドロキシ基$x\%$がアセタール化されたときの分子量を求めよう.\\[5cm]まず,100$\%$がアセタール化されたとすると,生成するビニロンの構造は\\[5cm]
である.よって,分子量の増加は$12\times n/2=6n$であるから,ビニルアルコールの単位構造1つにつき,分子量が6増加したことがわかる.\\
 ビニルアルコールの単位構造の数とヒドロキシ基の数は同じなので,ヒドロキシ基1個がアセタール化に使われると,「ビニルアルコール→ビニロン」で分子量は6増える.つまり,$x\%$のヒドロキシ式がアセタール化されたとき,生成するビニロンの分子量は
\[44n~~~+~~~\left(n\times \frac{x}{100}\right)~~~\times~~~ 6=\left(44+\frac{6x}{100}\right)n\]
\end{document}