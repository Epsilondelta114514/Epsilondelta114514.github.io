\documentclass[a4paper,11pt]{ltjsarticle}
\usepackage{base}
\title{}
\author{}
\date{}
\newtcolorbox{rembox}[1][]{enhanced,
    before skip=2mm,after skip=3mm,fontupper=\gtfamily\sffamily,
    boxrule=0.4pt,left=5mm,right=2mm,top=1mm,bottom=1mm,
    colback=yellow!50,
    colframe=yellow!20!black,
    sharp corners,rounded corners=southeast,arc is angular,arc=3mm,
    underlay={
        \path[fill=tcbcolback!80!black] ([yshift=3mm]interior.south east)--++(-0.4,-0.1)--++(0.1,-0.2);
        \path[draw=tcbcolframe,shorten <=-0.05mm,shorten >=-0.05mm] ([yshift=3mm]interior.south east)--++(-0.4,-0.1)--++(0.1,-0.2);
        \path[fill=yellow!50!black,draw=none] (interior.south west) rectangle node[white]{\Huge\bfseries !} ([xshift=4mm]interior.north west);
    },
drop fuzzy shadow,#1}
\newcommand{\printheader}[2]{
\begin{tikzpicture}[remember picture, overlay]
\node[yshift=-2.5cm, anchor=north] at (current page.north) {
\begin{tikzpicture}
\fill[gray!20] (0,0) rectangle (\textwidth, 2cm);
\fill[gray!80] (0,0) rectangle (0.2cm, 2cm);
\draw[gray!80, thick] (0,0) -- (	\textwidth, 0);
\node[anchor=west, text width=\textwidth-1cm, inner xsep=1cm] at (0, 1.25cm) {
\parbox[b]{\linewidth}{
{\color{gray!50!black}\bfseries #1} \par
\vspace{0.2em}
{\huge\bfseries #2}
}
};
\end{tikzpicture}
};
\end{tikzpicture}
\vspace{0.5cm}
}
\begin{document}
\printheader{単元別演習 数列②}{漸化式(応用①)}
\ascboxA{\textbf{連立漸化式}}
\begin{exque}
    次の漸化式で定まる数列$\{a_n\},\{b_n\}$の一般項を求めよ.
 \[a_1=1,~b_1=3,~\begin{cases}
       a_{n+1}&=3a_n+b_n\\
       b_{n+1}&=2a_n+4b_n
    \end{cases}\]  
\end{exque}
\ascboxG{\textbf{Point.}}連立漸化式は$\boldsymbol{a_{n+1}+\alpha b_{n+1}=\beta(a_n+\alpha b_n)}$を満たす$\alpha,\beta$を求めて等比数列の形にする.
\ans 
$a_{n+1}+\alpha b_{n+1}=\beta(a_n+\alpha b_n)\cdots(\ast)$に与えられた漸化式を代入して$a_{n+1},b_{n+1}$を消去すると,\[(3a_n+b_n)+\alpha (2a_n+4b_n)=\beta(a_n+\alpha b_n)\]
であるから,これを整理すると\[(2\alpha+3)a_n+(4\alpha+1)b_n=\beta(a_n+\alpha b_n)\]
である.両辺の係数を比較することにより,
\[\begin{cases}
  2\alpha+3&=\beta\\
  4\alpha+1&=\alpha\beta
\end{cases}\]
これを解くと,$\displaystyle{(\alpha,\beta)=\left(-\frac12,~2\right),~(1,~5)}$である.
\begin{itemize}
    \item[(1)]$\displaystyle{(\alpha,\beta)=\left(-\frac12,~2\right)}$を$(\ast)$に代入すると,
\[a_{n+1}-\frac12 b_{n+1}=2\left(a_n-\frac12 b_n\right)\]
なので,$\displaystyle{\left\{a_n-\frac12 b_n\right\}}$は初項$a_1-\dfrac12 b_1=-\dfrac12$,公比2の等比数列である.したがって,
\[a_n-\frac12b_n=-\frac12\cdot2^{n-1}\]
\item[(2)]$\displaystyle{(\alpha,\beta)=(1,~5)}$を$(\ast)$に代入すると,
\[a_{n+1}+ b_{n+1}=5\left(a_n+ b_n\right)\]
なので,$\displaystyle{\left\{a_n+ b_n\right\}}$は初項$a_1+ b_1=4$,公比5の等比数列である.したがって,
\[a_n+b_n=4\cdot5^{n-1}\]
\end{itemize}
以上より,\[\boldsymbol{a_{n}=\frac43\cdot5^{n-1}-\frac13\cdot2^{n-1},~b_n=\frac83\cdot5^{n-1}+\frac13\cdot2^{n-1}}\]
\begin{toi}
    次の漸化式で定まる数列$\{a_n\},\{b_n\}$の一般項を求めよ.\\[5pt]
    \begin{minipage}{0.5\linewidth}
        \begin{itemize}
            \item [(1)]$\displaystyle{a_1=3,~b_1=2,~\begin{cases}
      a_{n+1}&=2a_n+b_n\\
      b_{n+1}&=3a_n+4b_n
   \end{cases}}$   
        \end{itemize}
    \end{minipage}
    \begin{minipage}{0.5\linewidth}
         \begin{itemize}
            \item [(2)]$\displaystyle{a_1=1,~b_1=2,~\begin{cases}
      a_{n+1}&=2a_n-b_n\\
      b_{n+1}&=a_n+4b_n
   \end{cases}}$   
        \end{itemize}
    \end{minipage}
\end{toi}
 \\
\ascboxA{\textbf{分数型の漸化式}}
$\displaystyle{a_{n+1}=\frac{\gamma a_n+\delta}{\alpha a_n+\beta}}$の形の漸化式を,分数型の漸化式という.$\delta=0$の場合は簡単だが,そうでない場合は入試応用レベルに難しくなる.
\begin{exque}
    $a_1=1,~a_{n+1}=\dfrac{2a_n}{a+n+4}$で定まる数列$\{a_n\}$の一般項を求めよ.
\end{exque}
\ascboxG{\textbf{Point.}}$\boldsymbol{a_n\neq0}$を適当にチェックしてから両辺の逆数をとり,$b_n=\dfrac{1}{a_n}$とおくとうまくいく.
\ans 
初項と漸化式の形から,任意の$n$に対して$a_n\neq0$は明らか\footnote{厳密には数学的帰納法で示すべきだが,面倒だしそこまで求められてないと思うので適当でいい.}.そこで,両辺の逆数をとると,
\[\frac1{a_{n+1}}=\frac{a_n+4}{2a_n}=\frac{2}{a_n}+\frac12\]
となるから,$b_n=\dfrac{1}{a_n}$とおくと,数列$\{b_n\}$は漸化式
\[b_1=2,~b_{n+1}=2b_n+\frac12\]
で定まる数列である.これを解くと$b_n=3\cdot 2^{n-2}-\dfrac12=\dfrac{3\cdot 2^{n-1}-1}{2}$であるから,
\[\boldsymbol{a_n=\frac{1}{b_n}=\dfrac2{3\cdot 2^{n-1}-1}}\]
\newpage
\begin{toi}
  $a_1=2$,~$a_{n+1}=\dfrac{3a_n}{1-5a_n}$で定まる数列$\{a_n\}$の一般項を求めよ.
\end{toi}
\begin{exque}
     $a_1=7$,~$a_{n+1}=\dfrac{7a_n+3}{a_n+5}$で定まる数列$\{a_n\}$の一般項を求めよ.
\end{exque}
\ascboxG{\textbf{Point.}}特性方程式$\boldsymbol{x=\dfrac{7x+3}{x+5}}$の解$\alpha$を1つ選び,$b_n=a_n-\alpha$とおく.
\ans 
特性方程式$x=\dfrac{7x+3}{x+5}$を解くと,$x=-1,3$である.今回は$\alpha=3$とする.$a_n-3$を作るために,与えられた漸化式の両辺から3を引くと,
\[a_{n+1}-3=\dfrac{7a_n+3}{a_n+5}-3=\dfrac{4a_n-12}{a_n+5}=\frac{4(a_n-3)}{(a_n-3)+8}\]
と変形できる.$b_n=a_n-3$とおくと,数列$\{b_n\}$は漸化式
\[b_1=4,b_{n+1}=\frac{4b_n}{b_n+8}\]
で定まる数列である.例題2と同じようにして解くと,$b_n=\dfrac{4}{2^{n+1}-1}$なので,
\[\boldsymbol{a_n=b_n+3=\frac{3\cdot 2^{n+1}+1}{2^{n+1}-1}}\]
\begin{toi}
次の漸化式で定まる数列$\{a_n\}$の一般項を求めよ.\\[5pt]
\begin{minipage}{0.5\linewidth}
\begin{itemize}
    \item [(1)]$a_1=2,~a_{n+1}=\dfrac{3a_n+2}{a_n+4}$
\end{itemize}
\end{minipage}
\begin{minipage}{0.5\linewidth}
\begin{itemize}
    \item [(2)]$a_1=5,~a_{n+1}=\dfrac{4a_n-9}{a_n-2}$
\end{itemize}
\end{minipage}
\end{toi}
\newpage
\ascboxA{\textbf{復習用問題}}
\begin{toi}
    次の漸化式で定まる数列$\{a_n\},\{b_n\}$の一般項を求めよ.\\[5pt]
    \begin{minipage}{0.5\linewidth}
        \begin{itemize}
            \item [(1)]$\displaystyle{a_1=2,~b_1=-1,~\begin{cases}
      a_{n+1}&=3a_n+b_n\\
      b_{n+1}&=2a_n+2b_n
   \end{cases}}$   
        \end{itemize}
    \end{minipage}
    \begin{minipage}{0.5\linewidth}
         \begin{itemize}
            \item [(2)]$\displaystyle{a_1=3,~b_1=2,~\begin{cases}
      a_{n+1}&=3a_n+4b_n\\
      b_{n+1}&=2a_n+3b_n
   \end{cases}}$   
        \end{itemize}
    \end{minipage}
\end{toi}
\begin{toi}
  $a_1=\dfrac12$,~$a_{n+1}=\dfrac{a_n}{4a_n-3}$で定まる数列$\{a_n\}$の一般項を求めよ.
\end{toi}
\begin{toi}
  $a_1=\dfrac13$,~$a_{n+1}=\dfrac{1}{3-2a_n}$で定まる数列$\{a_n\}$の一般項を求めよ.
\end{toi}
\end{document}