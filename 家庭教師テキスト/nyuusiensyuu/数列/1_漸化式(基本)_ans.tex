\documentclass[a4paper,11pt]{ltjsarticle}
\usepackage{base}
\title{}
\author{}
\date{}
\newtcolorbox{rembox}[1][]{enhanced,
    before skip=2mm,after skip=3mm,fontupper=\gtfamily\sffamily,
    boxrule=0.4pt,left=5mm,right=2mm,top=1mm,bottom=1mm,
    colback=yellow!50,
    colframe=yellow!20!black,
    sharp corners,rounded corners=southeast,arc is angular,arc=3mm,
    underlay={
        \path[fill=tcbcolback!80!black] ([yshift=3mm]interior.south east)--++(-0.4,-0.1)--++(0.1,-0.2);
        \path[draw=tcbcolframe,shorten <=-0.05mm,shorten >=-0.05mm] ([yshift=3mm]interior.south east)--++(-0.4,-0.1)--++(0.1,-0.2);
        \path[fill=yellow!50!black,draw=none] (interior.south west) rectangle node[white]{\Huge\bfseries !} ([xshift=4mm]interior.north west);
    },
drop fuzzy shadow,#1}
\newcommand{\printheader}[2]{
\begin{tikzpicture}[remember picture, overlay]
\node[yshift=-2.5cm, anchor=north] at (current page.north) {
\begin{tikzpicture}
\fill[gray!20] (0,0) rectangle (\textwidth, 2cm);
\fill[gray!80] (0,0) rectangle (0.2cm, 2cm);
\draw[gray!80, thick] (0,0) -- (	\textwidth, 0);
\node[anchor=west, text width=\textwidth-1cm, inner xsep=1cm] at (0, 1.25cm) {
\parbox[b]{\linewidth}{
{\color{gray!50!black}\bfseries #1} \par
\vspace{0.2em}
{\huge\bfseries #2}
}
};
\end{tikzpicture}
};
\end{tikzpicture}
\vspace{0.5cm}
}
\begin{document}
\printheader{単元別演習 数列①}{漸化式(基本) (解答)}
\begin{toi}
次の漸化式で定まる数列$\{a_n\}$の一般項を求めよ.\\
\begin{minipage}{0.5\linewidth}
\begin{itemize}
    \item [(1)]$a_1=1,~a_{n+1}=\dfrac13a_n+3$
\end{itemize}
\end{minipage}
\begin{minipage}{0.5\linewidth}
\begin{itemize}
    \item [(2)]$a_1=\dfrac32,~2a_{n+1}=5a_n+3$
\end{itemize}
\end{minipage}
\end{toi}
\ans 
\begin{itemize}
    \item [(1)] 特性方程式$\alpha=\dfrac13\alpha+3$の解は$\alpha=\dfrac92$である.よって,与えられた漸化式は
    \[a_{n+1}-\frac92=\frac{1}{3}\left(a_n-\frac92\right)\]
    と変形できる.これより,数列$\left\{a_n-\dfrac92\right\}$は初項$a_1-\dfrac92=1-\dfrac92=-\dfrac72$,公比$\dfrac13$の等比数列であるから,
    \[\boldsymbol{a_n=-\frac{7}{2}\cdot\left(\frac13\right)^{n-1}+\frac92}\]
\item[(2)] 与式は $a_{n+1}=\dfrac52a_n+\dfrac32$ と変形できる.特性方程式$\alpha=\dfrac52\alpha+\dfrac32$の解は$\alpha=-1$である.よって,与えられた漸化式は
    \[a_{n+1}-(-1)=\frac{5}{2}(a_n-(-1))\Longleftrightarrow a_{n+1}+1=\frac{5}{2}(a_n+1)\]
    と変形できる.これより,数列$\{a_n+1\}$は初項$a_1+1=\dfrac32+1=\dfrac52$,公比$\dfrac52$の等比数列であるから,
    \[ \boldsymbol{a_n=\left(\frac52\right)^n-1}\]
\end{itemize}



 \begin{toi}
次の漸化式で定まる数列$\{a_n\}$の一般項を求めよ.\\
\begin{minipage}{0.5\linewidth}
\begin{itemize}
    \item [(1)]$a_1=2,~a_{n+1}=3a_n+2^n$
\end{itemize}
\end{minipage}
\begin{minipage}{0.5\linewidth}
\begin{itemize}
    \item [(2)]$a_1=5,~a_{n+1}=3a_n+5\cdot3^n$
\end{itemize}
\end{minipage}
\end{toi}
\ans 
\begin{itemize}
    \item [(1)]両辺を$2^{n+1}$で割ると,
 \[\frac{a_{n+1}}{2^{n+1}}=\frac32\cdot \frac{a_n}{2^n}+\frac12\]
 $b_n=\dfrac{a_n}{2^n}$とおくと,数列$\{b_n\}$は$b_1=\dfrac{a_1}{2^1}=\dfrac22=1,~b_{n+1}=\dfrac32 b_n+\dfrac12$で定まる数列である.
 特性方程式$\alpha=\dfrac32\alpha+\dfrac12$の解は$\alpha=-1$なので,
 \[b_{n+1}+1=\frac32(b_n+1)\]
 と変形できる.数列$\{b_n+1\}$は初項$2$,公比$\dfrac32$の等比数列であるから,$b_n+1=2\cdot\left(\dfrac32\right)^{n-1}$となり,
 $b_n=2\cdot\left(\dfrac32\right)^{n-1}-1$.よって,
 \[\boldsymbol{a_n=2^nb_n=4\cdot3^{n-1}-2^n}\]

\item[(2)] 両辺を$3^{n+1}$で割ると,
 \[\frac{a_{n+1}}{3^{n+1}}=\frac{a_n}{3^n}+\frac53\]
 $b_n=\dfrac{a_n}{3^n}$とおくと,数列$\{b_n\}$は$b_1=\dfrac53,~b_{n+1}=b_n+\dfrac53$で定まる数列である.
 これは初項$\dfrac53$,公差$\dfrac53$の等差数列であるから,$b_n=\dfrac53+(n-1)\dfrac53=\dfrac53n$.よって,
 \[\boldsymbol{a_n=3^nb_n=5n\cdot3^{n-1}}\]\end{itemize}


   \begin{toi}
次の漸化式で定まる数列$\{a_n\}$の一般項を求めよ.\\
\begin{minipage}{0.5\linewidth}
\begin{itemize}
    \item [(1)]    $a_1=2,~a_2=1,~a_{n+2}=a_{n+1}+2a_n$
\end{itemize}
\end{minipage}
\begin{minipage}{0.5\linewidth}
\begin{itemize}
    \item [(2)]$a_1=0,~a_2=2,~a_{n+2}=4a_{n+1}-4a_n$
\end{itemize}
\end{minipage}
\end{toi}
\ans 
\begin{itemize}
    \item [(1)] 特性方程式$x^2=x+2$の解は$x=2,-1$であるから,
  \[a_{n+2}+a_{n+1}=2(a_{n+1}+a_n)\]
  と変形できるので,数列$\{a_{n+1}+a_n\}$は初項$3$,公比2の等比数列なので,$a_{n+1}+a_n=3\cdot2^{n-1}$である.よって,指数関数型の漸化式$a_{n+1}=-a_n+3\cdot2^{n-1}$が得られたので,これを解いて,
  \[\boldsymbol{a_n=2^{n-1}+(-1)^{n-1}}\]

\item[(2)] 特性方程式$x^2=4x-4$の解は$x=2$(重解)であるから,
  \[a_{n+2}-2a_{n+1}=2(a_{n+1}-2a_n)\]
  と変形できるので,数列$\{a_{n+1}-2a_n\}$は初項$2$,公比2の等比数列なので,$a_{n+1}-2a_n=2^n$.
つまり,$a_{n+1}=2a_n+2^n$.これを解いて,
  \[\boldsymbol{a_n=(n-1)2^{n-1}}\]
\end{itemize}


\newpage
\ascboxA{\textbf{復習用問題}}
\begin{toi}
次の漸化式で定まる数列$\{a_n\}$の一般項を求めよ.\\
\begin{minipage}{0.5\linewidth}
\begin{itemize}
    \item [(1)]$a_1=6,~a_{n+1}=3a_n-8$
\end{itemize}
\end{minipage}
\begin{minipage}{0.5\linewidth}
\begin{itemize}
    \item [(2)]$a_1=2,~a_{n+1}=6a_n-15$
\end{itemize}
\end{minipage}
\end{toi}
\ans 
\begin{itemize}
    \item [(1)]特性方程式$\alpha=3\alpha-8$の解は$\alpha=4$である.よって,与えられた漸化式は
    \[a_{n+1}-4=3(a_n-4)\]
    と変形できる.これより,数列$\{a_n-4\}$は初項$2$,公比$3$の等比数列であるから,\\$a_n-4=2\cdot3^{n-1}$である.よって,
    \[\boldsymbol{a_n=2\cdot3^{n-1}+4}\]

\item[(2)] 特性方程式$\alpha=6\alpha-15$の解は$\alpha=3$である.よって,与えられた漸化式は
    \[a_{n+1}-3=6(a_n-3)\]
    と変形できる.これより,数列$\{a_n-3\}$は初項$-1$,公比$6$の等比数列であるから,\\$a_n-3=(-1)\cdot6^{n-1}$である.よって,
    \[\boldsymbol{a_n=-6^{n-1}+3}\]
\end{itemize}


\begin{toi}
次の漸化式で定まる数列$\{a_n\}$の一般項を求めよ.\\
\begin{minipage}{0.5\linewidth}
\begin{itemize}
    \item [(1)]$a_1=2,~a_{n+1}=3a_n+2^n$
\end{itemize}
\end{minipage}
\begin{minipage}{0.5\linewidth}
\begin{itemize}
    \item [(2)]$a_1=-10,~a_{n+1}=a_n+\dfrac{4}{3^n}$
\end{itemize}
\end{minipage}
\end{toi}
\ans 
\begin{itemize}
    \item [(1)]問2(1)と同じである. $\boldsymbol{a_n=4\cdot3^{n-1}-2^n}$.

\item[(2)]両辺に$3^{n+1}$をかけて,$3^{n+1}a_{n+1}=3(3^na_n)+12$となるので,$3^na_n=-8\cdot 3^n-6$.よって,
\[\boldsymbol{a_n=-2\cdot\left(\frac13\right)^{n-1}-8}\]

\end{itemize}

    \newpage
   \begin{toi}
次の漸化式で定まる数列$\{a_n\}$の一般項を求めよ.\\
\begin{minipage}{0.5\linewidth}
\begin{itemize}
    \item [(1)]    $a_1=1,~a_2=2,~a_{n+2}=a_{n+1}+6a_n$
\end{itemize}
\end{minipage}
\begin{minipage}{0.5\linewidth}
\begin{itemize}
    \item [(2)]$a_1=1,~a_2=1,~a_{n+2}=a_{n+1}+a_n$
\end{itemize}
\end{minipage}
\end{toi}
\ans 
\begin{itemize}
\item[(1)] 特性方程式$x^2=x+6$の解は$x=3,-2$であるから,
\begin{align*}
    a_{n+2}-3a_{n+1}=-2(a_{n+1}-3a_n) \end{align*}
と変形できる.
数列$\{a_{n+1}-3a_n\}$は初項$-1$,公比$-2$の等比数列なので,\\
$a_{n+1}-3a_n=-(-2)^{n-1}$.よって, $\boldsymbol{a_n=\dfrac15\left(4\cdot3^{n-1}+(-2)^{n-1}\right)}$.

\item[(2)] 特性方程式$x^2=x+1$の解は$x=\dfrac{1\pm\sqrt5}{2}$である.
$\alpha=\dfrac{1+\sqrt5}{2}, \beta=\dfrac{1-\sqrt5}{2}$とおくと,
\[a_{n+2}-\alpha a_{n+1}=\beta(a_{n+1}-\alpha a_n)\]
と変形できる.これより,数列$\{a_{n+1}-\alpha a_n\}$は初項$1-\alpha=\beta$,公比$\beta$の等比数列なので,$a_{n+1}-\alpha a_n=\beta^n$.
同様に,$a_{n+1}-\beta a_n=\alpha^n$.
辺々を引くと,$(\alpha-\beta)a_n=\alpha^n-\beta^n$.
ここで,$\alpha-\beta=\sqrt5$であるから,
\[\boldsymbol{a_n=\frac{1}{\sqrt5}\left\{\left(\frac{1+\sqrt5}{2}\right)^n-\left(\frac{1-\sqrt5}{2}\right)^n\right\}}\]
\end{itemize}


\end{document}