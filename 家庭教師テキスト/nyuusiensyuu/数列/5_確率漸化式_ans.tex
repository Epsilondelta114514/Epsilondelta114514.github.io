\documentclass[a4paper,11pt]{ltjsarticle}
\usepackage{base}
\title{}
\author{}
\date{}
\newtcolorbox{rembox}[1][]{enhanced, 
    before skip=2mm,after skip=3mm,fontupper=\gtfamily\sffamily,
    boxrule=0.4pt,left=5mm,right=2mm,top=1mm,bottom=1mm,
    colback=yellow!50,
    colframe=yellow!20!black,
    sharp corners,rounded corners=southeast,arc is angular,arc=3mm,
    underlay={
        \path[fill=tcbcolback!80!black] ([yshift=3mm]interior.south east)--++(-0.4,-0.1)--++(0.1,-0.2);
        \path[draw=tcbcolframe,shorten <=-0.05mm,shorten >=-0.05mm] ([yshift=3mm]interior.south east)--++(-0.4,-0.1)--++(0.1,-0.2);
        \path[fill=yellow!50!black,draw=none] (interior.south west) rectangle node[white]{\Huge\bfseries !} ([xshift=4mm]interior.north west);
    },
drop fuzzy shadow,#1}
\newcommand{\printheader}[2]{
\begin{tikzpicture}[remember picture, overlay]
\node[yshift=-2.5cm, anchor=north] at (current page.north) {
\begin{tikzpicture}
\fill[gray!20] (0,0) rectangle (\textwidth, 2cm);
\fill[gray!80] (0,0) rectangle (0.2cm, 2cm);
\draw[gray!80, thick] (0,0) -- (	\textwidth, 0);
\node[anchor=west, text width=\textwidth-1cm, inner xsep=1cm] at (0, 1.25cm) {
\parbox[b]{\linewidth}{
{\color{gray!50!black}\bfseries #1} \par
\vspace{0.2em}
{\huge\bfseries #2}
}
};
\end{tikzpicture}
};
\end{tikzpicture}
\vspace{0.5cm}
}
\begin{document}
\printheader{単元別演習 数列⑤}{確率漸化式 (解答)}
  \\[0.6cm]
\begin{toi}
正三角形ABCの頂点を点Pが次のルールに従って移動する:
\begin{itemize}
    \item 時刻0にPはAにいる.
    \item 1秒ごとにPは$\dfrac15$の確率で今いる頂点にとどまり,それぞれ$\dfrac25$の確率で他の2頂点のいずれかに移動する.
\end{itemize}
このとき,$n$秒後にPがAにいる確率を$p_n$を求めよ.
\end{toi}
\ans 
$n$秒後にPがBにいる確率を$q_n$,Cにいる確率を$r_n$とおく.$n+1$秒後にAにいるのは,
\begin{itemize}
    \item $n$秒後にAにいて,その場にとどまる.
    \item $n$秒後にBにいて,$n+1$秒後にAに移る.
    \item $n$秒後にCにいて,$n+1$秒後にAに移る.
\end{itemize}
のいずれかである.これらの確率は順に$\dfrac{p_n}{5},~\dfrac{2}{5}q_n,~\dfrac25r_n$であるから,
\[p_{n+1}=\dfrac{p_n}{5}+\dfrac{2}{5}(q_n+r_n)\]
ここで,$p_n+q_n+r_n=1$であるから,
\[p_{n+1}=\dfrac{p_n}{5}+\dfrac{2}{5}(1-p_n)=-\frac{p_n}5+\frac25\]
$p_1=\dfrac15$に注意してこの漸化式を解くと,
\[\boldsymbol{p_n=\frac{2}{3}\left(-\frac15\right)^n+\frac{1}{3}}\]
\newpage
\begin{toi}
正四面体ABCDの頂点を移動する点Pがある.点Pは1秒ごとに隣の3頂点のいずれかに等しい確率$\dfrac a3$で移るか,もとの頂点に確率$1-a$でとどまる.はじめ頂点Aにいた点Pが,$n$秒後に頂点Aにいる確率を$p_n$とする.ただし,$0<a<1$とし,$n$は自然数とする.
\begin{itemize}
    \item [(1)]数列$\{p_n\}$の漸化式を求めよ.
    \item [(2)]確率$p_n$を求めよ.
\end{itemize}
\hfill(北海道大)
\end{toi}
\ans 
\begin{itemize}
    \item [(1)]図形の対称性から,$n$秒後にPがB,C,Dにいる確率はそれぞれ等しいので,これを$q_n$とおく.このとき,$p_n+3q_n=1$に注意しておく.$n+1$秒後にPがAにいるのは,
\begin{itemize}
    \item $n$秒後にAにいて,その場にとどまる.
    \item $n$秒後にB,C,Dのいずれかにいて,$n+1$秒後にAに移る.
\end{itemize}
のいずれかである.確率は順に$p_n(1-a_n),~3q_n\cdot \dfrac a3=aq_n$であるから,求める漸化式は
\[\boldsymbol{p_{n+1}=p_n(1-a_n)+aq_n=\left(1-\frac{4a}3\right)p_n+\frac a3.}\]
    \item [(2)]上で導いた漸化式を解けば,\[\boldsymbol{p_n=\frac14+\frac34\left(1-\frac{4a}{3}\right)^n}\]
\end{itemize}


\end{document}