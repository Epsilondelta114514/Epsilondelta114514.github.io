\documentclass[a4paper,11pt]{ltjsarticle}
\usepackage{base}
\title{}
\author{}
\date{}
\newtcolorbox{rembox}[1][]{enhanced,
    before skip=2mm,after skip=3mm,fontupper=\gtfamily\sffamily,
    boxrule=0.4pt,left=5mm,right=2mm,top=1mm,bottom=1mm,
    colback=yellow!50,
    colframe=yellow!20!black,
    sharp corners,rounded corners=southeast,arc is angular,arc=3mm,
    underlay={
        \path[fill=tcbcolback!80!black] ([yshift=3mm]interior.south east)--++(-0.4,-0.1)--++(0.1,-0.2);
        \path[draw=tcbcolframe,shorten <=-0.05mm,shorten >=-0.05mm] ([yshift=3mm]interior.south east)--++(-0.4,-0.1)--++(0.1,-0.2);
        \path[fill=yellow!50!black,draw=none] (interior.south west) rectangle node[white]{\Huge\bfseries !} ([xshift=4mm]interior.north west);
    },
drop fuzzy shadow,#1}
\newcommand{\printheader}[2]{
\begin{tikzpicture}[remember picture, overlay]
\node[yshift=-2.5cm, anchor=north] at (current page.north) {
\begin{tikzpicture}
\fill[gray!20] (0,0) rectangle (\textwidth, 2cm);
\fill[gray!80] (0,0) rectangle (0.2cm, 2cm);
\draw[gray!80, thick] (0,0) -- (	\textwidth, 0);
\node[anchor=west, text width=\textwidth-1cm, inner xsep=1cm] at (0, 1.25cm) {
\parbox[b]{\linewidth}{
{\color{gray!50!black}\bfseries #1} \par
\vspace{0.2em}
{\huge\bfseries #2}
}
};
\end{tikzpicture}
};
\end{tikzpicture}
\vspace{0.5cm}
}
\begin{document}
\printheader{単元別演習}{10月25日の問題}

\begin{toi}
     $a_1=10,~a_{n+1}=a_n^2$で定まる数列$\{a_n\}$の一般項を求めよ.
\end{toi}
\begin{toi}
    数列$\{a_n\}$に対して,$S_n=\displaystyle{\sum_{k=1}^n a_k}$とすると,
    $S_n=\dfrac{3}{2}a_n+3-4n$
    が成り立つとする.
    \begin{itemize}
        \item [(1)]$a_1$を求めよ.
        \item [(2)]$a_{n+1}$と$a_n$の漸化式を作れ.
        \item [(3)]$a_n$を求めよ.
    \end{itemize}
\end{toi}
\begin{toi}
\noindent 正の実数$a$についての関数$f(a)$を$\displaystyle{f(a)=\int_{-1}^1 |x^2-a^2|dx}$により定める.
\begin{itemize}
    \item [(1)]$f(a)$を求めよ.
    \item [(2)]$f(a)$の最小値を求めよ.
\end{itemize}
\end{toi}
\begin{toi}
\noindent 1次式$f_n(x)=a_nx+b_n~(n=1,2,\ldots)$が
\[f_1(x)=x+1,~~~~x^2f_{n+1}(x)=x^3+x^2+\int_0^{x}tf_n(t)dt~(n=1,2,\ldots)\]
を満たすとする.
\begin{itemize}
    \item [(1)]数列$\{a_n\},~\{b_n\}$が満たす漸化式を求めよ.
        \item [(2)]$f_n(x)$を求めよ.
\end{itemize}
\end{toi}

\begin{toi}
\noindent 放物線$\mathrm{C}~:~y=x^2$とその上の点$(a,a^2)$~$(0<a\leqq 1)$における接線を$l$とするとき,次の問いに答えよ.
\begin{itemize}
    \item[(1)]$l$の方程式を求めよ.
    \item[(2)]直線$x=0,~x=1,~$放物線Cと接線$l$とで囲まれた部分で,$y\geqq0$を満たす部分の面積$S(a)$を求めよ.
    \item[(3)]$S(a)$の最小値を求めよ.   
\end{itemize}
\end{toi}

\newpage
\printheader{単元別演習}{10月25日の宿題}
\setcounter{toicounter}{0}
\begin{toi}
整式 $P(x)$ を $(x-1)^2$ で割ったときの余りが $4x-5$ で,$x+2$ で割ったときの余りが $-4$ である.
\begin{enumerate}
  \item[(1)] $P(x)$ を $x-1$ で割ったときの余りを求めよ.
  \item [(2)]$P(x)$ を $(x-1)(x+2)$ で割ったときの余りを求めよ.
  \item [(3)]$P(x)$ を $(x-1)^2(x+2)$ で割ったときの余りを求めよ.
\end{enumerate}
\hfill[山形大]
\end{toi}

\begin{toi}
正四面体ABCDの頂点を移動する点Pがある.点Pは1秒ごとに隣の3頂点のいずれかに等しい確率$\dfrac a3$で移るか,もとの頂点に確率$1-a$でとどまる.はじめ頂点Aにいた点Pが,$n$秒後に頂点Aにいる確率を$p_n$とする.ただし,$0<a<1$とし,$n$は自然数とする.
\begin{itemize}
    \item [(1)]数列$\{p_n\}$の漸化式を求めよ.
    \item [(2)]確率$p_n$を求めよ.
\end{itemize}
\hfill(北海道大)
\end{toi}

\newpage
\noindent \textbf{問1の解答例}\\[5pt]
$P(x)=(x-1)^2q_1(x)+4x-5 \cdots$①,$P(x)=(x+2)q_2(x)-4 \cdots$②とおく.
\begin{enumerate}
  \item[(1)] 因数定理より$x-1$で割った余りは$P(1)=4\cdot1-5=\boldsymbol{-1}$.
  \item [(2)] $P(x)$を$(x-1)(x+2)$で割った余りを$ax+b$とおく.
  $P(1)=-1, P(-2)=-4$なので,
  \[\begin{cases}
  a+b=-1 \\ -2a+b=-4
  \end{cases}\]
  これを解いて$a=1, b=-2$.よって余りは $\boldsymbol{x-2}$.
  \item[(3)] $P(x)$を$(x-1)^2(x+2)$で割った余りは2次以下の式なので,$r(x)$とおく.
  \[P(x)=(x-1)^2(x+2)q(x)+r(x)\]
  この式より,$P(x)$を$(x-1)^2$で割った余りは,$r(x)$を$(x-1)^2$で割った余りと等しい.
  ①よりこの余りは$4x-5$なので,$r(x)$は定数$a$を用いて
  \[r(x)=a(x-1)^2+4x-5\]
  と書ける.よって,
  \[P(x)=(x-1)^2(x+2)q(x)+a(x-1)^2+4x-5\]
  ここに$x=-2$を代入すると,②より$P(-2)=-4$なので,
  \[-4=P(-2)=9a-13\]
これより$a=1$なので,求める余りは $1\cdot(x-1)^2+4x-5 =\boldsymbol{x^2+2x-4}$.
\end{enumerate}
\textbf{問2の解答例}
\begin{itemize}
    \item [(1)]図形の対称性から,$n$秒後にPがB,C,Dにいる確率はそれぞれ等しいので,これを$q_n$とおく.このとき,$p_n+3q_n=1$に注意しておく.$n+1$秒後にPがAにいるのは,
\begin{itemize}
    \item $n$秒後にAにいて,その場にとどまる.
    \item $n$秒後にB,C,Dのいずれかにいて,$n+1$秒後にAに移る.
\end{itemize}
のいずれかである.確率は順に$p_n(1-a_n),~3q_n\cdot \dfrac a3=aq_n$であるから,求める漸化式は
\[\boldsymbol{p_{n+1}=p_n(1-a_n)+aq_n=\left(1-\frac{4a}3\right)p_n+\frac a3.}\]
    \item [(2)]上で導いた漸化式を解けば,\[\boldsymbol{p_n=\frac14+\frac34\left(1-\frac{4a}{3}\right)^n}\]
\end{itemize}
\end{document}