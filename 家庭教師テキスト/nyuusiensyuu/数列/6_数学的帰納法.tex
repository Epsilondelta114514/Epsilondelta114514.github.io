\documentclass[a4paper,11pt]{ltjsarticle}
\usepackage{base}
\title{}
\author{}
\date{}
\newtcolorbox{rembox}[1][]{enhanced, 
    before skip=2mm,after skip=3mm,fontupper=\gtfamily\sffamily,
    boxrule=0.4pt,left=5mm,right=2mm,top=1mm,bottom=1mm,
    colback=yellow!50,
    colframe=yellow!20!black,
    sharp corners,rounded corners=southeast,arc is angular,arc=3mm,
    underlay={
        \path[fill=tcbcolback!80!black] ([yshift=3mm]interior.south east)--++(-0.4,-0.1)--++(0.1,-0.2);
        \path[draw=tcbcolframe,shorten <=-0.05mm,shorten >=-0.05mm] ([yshift=3mm]interior.south east)--++(-0.4,-0.1)--++(0.1,-0.2);
        \path[fill=yellow!50!black,draw=none] (interior.south west) rectangle node[white]{\Huge\bfseries !} ([xshift=4mm]interior.north west);
    },
drop fuzzy shadow,#1}
\newcommand{\printheader}[2]{
\begin{tikzpicture}[remember picture, overlay]
\node[yshift=-2.5cm, anchor=north] at (current page.north) {
\begin{tikzpicture}
\fill[gray!20] (0,0) rectangle (\textwidth, 2cm);
\fill[gray!80] (0,0) rectangle (0.2cm, 2cm);
\draw[gray!80, thick] (0,0) -- (	\textwidth, 0);
\node[anchor=west, text width=\textwidth-1cm, inner xsep=1cm] at (0, 1.25cm) {
\parbox[b]{\linewidth}{
{\color{gray!50!black}\bfseries #1} \par
\vspace{0.2em}
{\huge\bfseries #2}
}
};
\end{tikzpicture}
};
\end{tikzpicture}
\vspace{0.5cm}
}
\begin{document}
\printheader{単元別演習 数列⑥}{数学的帰納法}
  \\[0.6cm]
「$\boldsymbol{(n+1)回目の試行の確率が,n回目(場合によっては(n-1)回目)の試行の確率のみに依存する}$」ような確率の問題には,漸化式が極めて有効である.
\begin{exque}
    Aの袋には赤玉が1個と黒玉3個が,Bの袋には黒玉が3個入っている.それぞれの袋から同時に1個ずつの玉を取り出して入れ替える操作を繰り返す.この操作を$n$回繰り返したときに,Aの袋に赤玉が入っている確率を$a_n$とおく.
    \begin{itemize}
        \item [(1)]$a_1,~a_2$を求めよ.
        \item [(2)]$a_{n+1}$を$a_n$を用いて表せ.
        \item [(3)]$a_n$を求めよ.
    \end{itemize}
\end{exque}
\ascboxG{\textbf{Point.}}図を書くと漸化式が立てやすい.
\begin{figure}[H]
\centering
\begin{tikzpicture}
 \draw(0,0)node{$n$回目};
  \draw(6,0)node{$(n+1)$回目};
   \draw(0,-1)node{$\boldsymbol{a_n}$};
      \draw(0,-3)node{$\boldsymbol{1-a_n}$};
         \draw(-1,-1)node[left]{Aに赤玉あり};
      \draw(-1,-3)node[left]{Aに赤玉なし};
      \draw[thick,->](0.5,-1)--(5,-1);
            \draw[thick,->](0.6,-3)--(5,-1.3);
               \draw(6,-1)node{$\boldsymbol{a_{n+1}}$};
                  \draw(2.75,-1)node[above]{$\boldsymbol{\dfrac34}$};
                                 \draw(2.75,-3)node{$\boldsymbol{\dfrac13}$};
\end{tikzpicture}
\end{figure}
\ans 
\begin{itemize}
    \item [(1)]初期状態ではAに赤玉が入っているので,$a_1$はAから黒球を取り出す確率と同じである.よって,$\boldsymbol{a_1=\dfrac34}$である.次いで,$a_2$を求めよう.2回目の試行の後にAに赤玉が入っているのは,
    \begin{itemize}
        \item [・]1回目の試行の後にAに赤玉があり,2回目の試行でAから黒玉を取り出す場合
        \item [・]1回目の試行の後にBに赤玉があり,2回目の試行でBから赤玉を取り出す場合\\
    \end{itemize}
    のいずれかである.前者の確率は$a_1\cdot\dfrac34$,後者の確率は$(1-a_1)\cdot\dfrac{1}{3}$であるから,
    \[\boldsymbol{a_2=\frac34a_1+\frac13(1-a_1)=\frac {31}{48}}.\]
    \newpage
\item[(2)]$(n+1)$回目の試行の後にAに赤玉が入っているのは,
    \begin{itemize}
        \item [・]$n$回目の試行の後にAに赤玉があり,$(n+1)$回目の試行でAから黒玉を取り出す場合
        \item [・]$n$回目の試行の後にBに赤玉があり,$(n+1)$回目の試行でBから赤玉を取り出す場合 
    \end{itemize}
    のいずれかである(Pointの図を参照).よって,
    \[\boldsymbol{a_{n+1}=\frac34a_n+\frac13(1-a_n)=\frac{5}{12}a_n+\frac13} \]
    \item[(3)](2)で導いた漸化式を解くと,$\boldsymbol{a_n=\dfrac{3}{7}\left(\dfrac{5}{12}\right)^n+\dfrac47}.$
\end{itemize}
\begin{toi}
正三角形ABCの頂点を点Pが次のルールに従って移動する:
\begin{itemize}
    \item 時刻0にPはAにいる.
    \item 1秒ごとにPは$\dfrac15$の確率で今いる頂点にとどまり,それぞれ$\dfrac25$の確率で他の2頂点のいずれかに移動する.
\end{itemize}
このとき,$n$秒後にPがAにいる確率を$p_n$を求めよ.
\end{toi}
\begin{toi}
正四面体ABCDの頂点を移動する点Pがある.点Pは1秒ごとに隣の3頂点のいずれかに等しい確率$\dfrac a3$で移るか,もとの頂点に確率$1-a$でとどまる.はじめ頂点Aにいた点Pが,$n$秒後に頂点Aにいる確率を$p_n$とする.ただし,$0<a<1$とし,$n$は自然数とする.
\begin{itemize}
    \item [(1)]数列$\{p_n\}$の漸化式を求めよ.
    \item [(2)]確率$p_n$を求めよ.
\end{itemize}
\hfill(北海道大)
\end{toi}
\end{document}