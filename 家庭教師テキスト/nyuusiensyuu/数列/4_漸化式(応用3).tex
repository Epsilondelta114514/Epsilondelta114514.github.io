\documentclass[a4paper,11pt]{ltjsarticle}
\usepackage{base}
\title{}
\author{}
\date{}
\newtcolorbox{rembox}[1][]{enhanced,
    before skip=2mm,after skip=3mm,fontupper=\gtfamily\sffamily,
    boxrule=0.4pt,left=5mm,right=2mm,top=1mm,bottom=1mm,
    colback=yellow!50,
    colframe=yellow!20!black,
    sharp corners,rounded corners=southeast,arc is angular,arc=3mm,
    underlay={
        \path[fill=tcbcolback!80!black] ([yshift=3mm]interior.south east)--++(-0.4,-0.1)--++(0.1,-0.2);
        \path[draw=tcbcolframe,shorten <=-0.05mm,shorten >=-0.05mm] ([yshift=3mm]interior.south east)--++(-0.4,-0.1)--++(0.1,-0.2);
        \path[fill=yellow!50!black,draw=none] (interior.south west) rectangle node[white]{\Huge\bfseries !} ([xshift=4mm]interior.north west);
    },
drop fuzzy shadow,#1}
\newcommand{\printheader}[2]{
\begin{tikzpicture}[remember picture, overlay]
\node[yshift=-2.5cm, anchor=north] at (current page.north) {
\begin{tikzpicture}
\fill[gray!20] (0,0) rectangle (\textwidth, 2cm);
\fill[gray!80] (0,0) rectangle (0.2cm, 2cm);
\draw[gray!80, thick] (0,0) -- (	\textwidth, 0);
\node[anchor=west, text width=\textwidth-1cm, inner xsep=1cm] at (0, 1.25cm) {
\parbox[b]{\linewidth}{
{\color{gray!50!black}\bfseries #1} \par
\vspace{0.2em}
{\huge\bfseries #2}
}
};
\end{tikzpicture}
};
\end{tikzpicture}
\vspace{0.5cm}
}
\begin{document}
\printheader{単元別演習 数列④}{漸化式(応用③)}
\ascboxA{\textbf{$n$の式が入っているやつ}}
\begin{exque}
    $a_1=6,~a_{n+1}=2a_n+4n+3$で定まる数列$\{a_n\}$の一般項を求めよ.
\end{exque}
\ascboxG{\textbf{Point.}}
$\boldsymbol{a_{n+1}=pa_n+qn+r}$は,
\[\boldsymbol{a_{n+1}+\alpha(n+1)+\beta=p(a_n+\alpha n+\beta)}\]となる$\alpha,\beta$を求めて等比数列に帰着させる\footnote{
    $a_{n+1}=pa_n+qn^2+rn+s$なら,2次式を用いて
    \[a_{n+1}+\alpha(n+1)^2+\beta(n+1)+\gamma=p(a_n+\alpha n^2+\beta n+\gamma)\]
    と変形する.}.
\ans 
$a_{n+1}+\alpha(n+1)+\beta=2(a_n+\alpha n+\beta)$が成り立つとするとき,これを整理すると
\[a_{n+1}=2a_n+\alpha n-\alpha+\beta\]
となるので,与えられた漸化式と係数を比較すると,
\[\begin{cases}
    \alpha&=4\\
    -\alpha+\beta&=3
\end{cases}\]
より,$\alpha=4,\beta=7$である.よって,
\[a_{n+1}+4(n+1)+7=2(a_n+4n+7)\]
と変形できるので,数列$\{a_n+4n+7\}$は初項17,公比2の等比数列である.したがって, $a_n+4n+7=17\cdot2^{n-1}$より,
\[\boldsymbol{a_n=17\cdot2^{n-1}-4n-7}\]
\begin{toi}
次の漸化式で定まる数列$\{a_n\}$の一般項を求めよ.\\
\begin{minipage}{0.5\linewidth}
\begin{itemize}
    \item [(1)]$a_1=1,~a_{n+1}=2a_n+n-1$
\end{itemize}
\end{minipage}
\begin{minipage}{0.5\linewidth}
\begin{itemize}
    \item [(2)]$a_1=0,~a_{n+1}=\frac12 a_n +n$
\end{itemize}
\end{minipage}

\end{toi}
\begin{toi}
$a_1=1,~a_{n+1}=2a_n-n^2+2n$で定まる数列$\{a_n\}$の一般項を求めよ.
\end{toi}
\end{document}