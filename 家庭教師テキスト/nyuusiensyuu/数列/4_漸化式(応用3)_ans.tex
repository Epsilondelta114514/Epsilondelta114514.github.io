\documentclass[a4paper,11pt]{ltjsarticle}
\usepackage{base}
\title{}
\author{}
\date{}
\newtcolorbox{rembox}[1][]{enhanced,
    before skip=2mm,after skip=3mm,fontupper=\gtfamily\sffamily,
    boxrule=0.4pt,left=5mm,right=2mm,top=1mm,bottom=1mm,
    colback=yellow!50,
    colframe=yellow!20!black,
    sharp corners,rounded corners=southeast,arc is angular,arc=3mm,
    underlay={
        \path[fill=tcbcolback!80!black] ([yshift=3mm]interior.south east)--++(-0.4,-0.1)--++(0.1,-0.2);
        \path[draw=tcbcolframe,shorten <=-0.05mm,shorten >=-0.05mm] ([yshift=3mm]interior.south east)--++(-0.4,-0.1)--++(0.1,-0.2);
        \path[fill=yellow!50!black,draw=none] (interior.south west) rectangle node[white]{\Huge\bfseries !} ([xshift=4mm]interior.north west);
    },
drop fuzzy shadow,#1}
\newcommand{\printheader}[2]{
\begin{tikzpicture}[remember picture, overlay]
\node[yshift=-2.5cm, anchor=north] at (current page.north) {
\begin{tikzpicture}
\fill[gray!20] (0,0) rectangle (\textwidth, 2cm);
\fill[gray!80] (0,0) rectangle (0.2cm, 2cm);
\draw[gray!80, thick] (0,0) -- (	\textwidth, 0);
\node[anchor=west, text width=\textwidth-1cm, inner xsep=1cm] at (0, 1.25cm) {
\parbox[b]{\linewidth}{
{\color{gray!50!black}\bfseries #1} \par
\vspace{0.2em}
{\huge\bfseries #2}
}
};
\end{tikzpicture}
};
\end{tikzpicture}
\vspace{0.5cm}
}
\begin{document}
\printheader{単元別演習 数列④}{漸化式(応用③) (解答)}
\begin{toi}
次の漸化式で定まる数列$\{a_n\}$の一般項を求めよ.\\
\begin{minipage}{0.5\linewidth}
\begin{itemize}
    \item [(1)]$a_1=1,~a_{n+1}=2a_n+n-1$
\end{itemize}
\end{minipage}
\begin{minipage}{0.5\linewidth}
\begin{itemize}
    \item [(2)]$a_1=0,~a_{n+1}=\frac12 a_n +n$
\end{itemize}
\end{minipage}
\end{toi}
\ans 
\begin{minipage}{0.5\linewidth}
\begin{itemize}
    \item [(1)]$\boldsymbol{a_n = 2^n - n}$
\end{itemize}
\end{minipage}
\begin{minipage}{0.5\linewidth}
\begin{itemize}
    \item [(2)]$\boldsymbol{a_n = 2^{2-n} + 2n - 4}$
\end{itemize}
\end{minipage}
\begin{toi}
$a_1=1,~a_{n+1}=2a_n-n^2+2n$で定まる数列$\{a_n\}$の一般項を求めよ.
\end{toi}
\ans 
まず,$a_{n+1}+p(n+1)^2+q(n+1)+r=2(a_n+pn^2+qn+r)$を満たす$p,a,r$を求める.これを整理すると
\[a_{n+1}= 2a_n + pn^2 + (2q-2p)n + (r-p-q)\]
となるので,与えられた漸化式と係数を比較して$p=-1,~q=0,~r=-1$を得る.よって,
\[a_{n+1}-(n+1)^2-1=2(a_n-n^2-1)\]
より,$\{a_n-n^2-1\}$は初項$-1$,公比2の等比数列であるから,$a_n-n^2-1=-2^{n-1}$.\\したがって,
$$\boldsymbol{a_n = n^2 + 1 - 2^{n-1}}$$

\end{document}