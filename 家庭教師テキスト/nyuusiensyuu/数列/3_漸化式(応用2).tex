\documentclass[a4paper,11pt]{ltjsarticle}
\usepackage{base}
\title{}
\author{}
\date{}
\newtcolorbox{rembox}[1][]{enhanced,
    before skip=2mm,after skip=3mm,fontupper=\gtfamily\sffamily,
    boxrule=0.4pt,left=5mm,right=2mm,top=1mm,bottom=1mm,
    colback=yellow!50,
    colframe=yellow!20!black,
    sharp corners,rounded corners=southeast,arc is angular,arc=3mm,
    underlay={
        \path[fill=tcbcolback!80!black] ([yshift=3mm]interior.south east)--++(-0.4,-0.1)--++(0.1,-0.2);
        \path[draw=tcbcolframe,shorten <=-0.05mm,shorten >=-0.05mm] ([yshift=3mm]interior.south east)--++(-0.4,-0.1)--++(0.1,-0.2);
        \path[fill=yellow!50!black,draw=none] (interior.south west) rectangle node[white]{\Huge\bfseries !} ([xshift=4mm]interior.north west);
    },
drop fuzzy shadow,#1}
\newcommand{\printheader}[2]{
\begin{tikzpicture}[remember picture, overlay]
\node[yshift=-2.5cm, anchor=north] at (current page.north) {
\begin{tikzpicture}
\fill[gray!20] (0,0) rectangle (\textwidth, 2cm);
\fill[gray!80] (0,0) rectangle (0.2cm, 2cm);
\draw[gray!80, thick] (0,0) -- (	\textwidth, 0);
\node[anchor=west, text width=\textwidth-1cm, inner xsep=1cm] at (0, 1.25cm) {
\parbox[b]{\linewidth}{
{\color{gray!50!black}\bfseries #1} \par
\vspace{0.2em}
{\huge\bfseries #2}
}
};
\end{tikzpicture}
};
\end{tikzpicture}
\vspace{0.5cm}
}
\begin{document}
\printheader{単元別演習 数列③}{漸化式(応用②)}
\ascboxA{\textbf{対数型}}
\begin{exque}
     $a_1=10,~a_{n+1}=a_n^2$で定まる数列$\{a_n\}$の一般項を求めよ.
\end{exque}
\ascboxG{\textbf{Point.}}$\boldsymbol{a_{n+1}=p(a_n)^q}$は両辺の対数を取る(底は何でもいいが,初項や$p$の値を参考にして対数が綺麗になるものを選ぶ).
\ans 
$a_1=10$なので,両辺の常用対数をとると,
\[\log_{10}a_{n+1}=2\log_{10}a_n\]
より,数列$\{\log_{10}a_n\}$は初項1,公比2の等比数列である.よって,$\log_{10}a_n=2^{n-1}$なので,
\[\boldsymbol{a_n=10^{\log_{10}a_n}=10^{2^{n-1}}}\]
 \begin{toi}
次の漸化式で定まる数列$\{a_n\}$の一般項を求めよ.\\
\begin{minipage}{0.5\linewidth}
\begin{itemize}
    \item [(1)]     $a_1=2,~a_{n+1}=16a_n^5$
\end{itemize}
\end{minipage}
\begin{minipage}{0.5\linewidth}
\begin{itemize}
    \item [(2)] $a_1=3,~a_{n+1}=9\sqrt{a_n}$
\end{itemize}
\end{minipage}
\end{toi}
\ascboxA{\textbf{和を含む漸化式}}
\begin{exque}
    数列$\{a_n\}$に対して,$S_n=\displaystyle{\sum_{k=1}^n a_k}$とすると,
    $S_n=\dfrac{3}{2}a_n+3-4n$
    が成り立つとする.
    \begin{itemize}
        \item [(1)]$a_1$を求めよ.
        \item [(2)]$a_{n+1}$と$a_n$の漸化式を作れ.
        \item [(3)]$a_n$を求めよ.
    \end{itemize}
\end{exque}
\ans 
\begin{itemize}
    \item [(1)]$a_1=S_1=\dfrac{3}2a_1-1$より,$\boldsymbol{a_1=2.}$
    \item [(2)]$a_{n+1}=S_{n+1}-S_n=\dfrac{3}{2}a_{n+1}-\dfrac{3}{2}a_n-4$より,$\boldsymbol{a_{n+1}=3a_n+8.}$
    \item [(3)](1),(2)より,$\{a_n\}$は$a_1=2,~a_{n+1}=3a_n+8$で定まる数列なので,
    $\boldsymbol{a_n=2\cdot 3^n-4.}$
\end{itemize}
\begin{toi}
    $S_n=3a_n+2n+1$で定まる数列$\{a_n\}$の一般項を求めよ.
\end{toi}
\ascboxA{\textbf{復習用問題}}
\begin{toi}
    $a_1=10,~a_{n+1}=\sqrt{\sqrt{10a_n}}$で定まる数列$\{a_n\}$の一般項を求めよ.
\end{toi}
\begin{toi}
    $S_n=-2a_n-2n+5$で定まる数列$\{a_n\}$の一般項を求めよ.
\end{toi}
\end{document}