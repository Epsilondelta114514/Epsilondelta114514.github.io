\documentclass[a4paper,11pt]{ltjsarticle}
\usepackage{base}
\title{}
\author{}
\date{}
\newtcolorbox{rembox}[1][]{enhanced,
    before skip=2mm,after skip=3mm,fontupper=\gtfamily\sffamily,
    boxrule=0.4pt,left=5mm,right=2mm,top=1mm,bottom=1mm,
    colback=yellow!50,
    colframe=yellow!20!black,
    sharp corners,rounded corners=southeast,arc is angular,arc=3mm,
    underlay={
        \path[fill=tcbcolback!80!black] ([yshift=3mm]interior.south east)--++(-0.4,-0.1)--++(0.1,-0.2);
        \path[draw=tcbcolframe,shorten <=-0.05mm,shorten >=-0.05mm] ([yshift=3mm]interior.south east)--++(-0.4,-0.1)--++(0.1,-0.2);
        \path[fill=yellow!50!black,draw=none] (interior.south west) rectangle node[white]{\Huge\bfseries !} ([xshift=4mm]interior.north west);
    },
drop fuzzy shadow,#1}
\newcommand{\printheader}[2]{
\begin{tikzpicture}[remember picture, overlay]
\node[yshift=-2.5cm, anchor=north] at (current page.north) {
\begin{tikzpicture}
\fill[gray!20] (0,0) rectangle (\textwidth, 2cm);
\fill[gray!80] (0,0) rectangle (0.2cm, 2cm);
\draw[gray!80, thick] (0,0) -- (	\textwidth, 0);
\node[anchor=west, text width=\textwidth-1cm, inner xsep=1cm] at (0, 1.25cm) {
\parbox[b]{\linewidth}{
{\color{gray!50!black}\bfseries #1} \par
\vspace{0.2em}
{\huge\bfseries #2}
}
};
\end{tikzpicture}
};
\end{tikzpicture}
\vspace{0.5cm}
}
\begin{document}
\printheader{単元別演習 数列①}{漸化式(基本)}
\begin{exque}
$a_1=5,~a_{n+1}=\dfrac23a_n+1$で定まる数列$\{a_n\}$の一般項を求めよ.
\end{exque}
\ascboxG{\textbf{Point.}}特殊解型「$\boldsymbol{a_{n+1}=pa_n+q}$」は,特性方程式$\boldsymbol{x=px+q}$の解$\alpha$を使って$\boldsymbol{a_{n+1}-\alpha=p(a_n-\alpha)}$と変形する.
\ans 
特性方程式$\alpha=\dfrac23\alpha+1$を解くと,$\alpha=3$である.よって,与えられた漸化式は
    \[a_{n+1}-3=\frac{2}{3}(a_n-3)\]
    と変形できる.これより,数列$\{a_n-3\}$は初項$a_1-3=2$,公比$\dfrac23$の等比数列であるから,
    \[a_n-3=2\cdot\left(\frac23\right)^{n-1}\Longleftrightarrow \boldsymbol{a_n=2\cdot\left(\frac23\right)^{n-1}+3}\]
\begin{toi}
次の漸化式で定まる数列$\{a_n\}$の一般項を求めよ.\\
\begin{minipage}{0.5\linewidth}
\begin{itemize}
    \item [(1)]$a_1=1,~a_{n+1}=\dfrac13a_n+3$
\end{itemize}
\end{minipage}
\begin{minipage}{0.5\linewidth}
\begin{itemize}
    \item [(2)]$a_1=\dfrac32,~2a_{n+1}=5a_n+3$
\end{itemize}
\end{minipage}

\end{toi}
\begin{exque}
$a_1=5,~a_{n+1}=2a_n+3^n$で定まる数列$\{a_n\}$の一般項を求めよ.
\end{exque}
\ascboxG{\textbf{Point.}}指数関数型「$\boldsymbol{a_{n+1}=pa_n+q^n}$」は,\uwave{両辺を$\boldsymbol{q^{n+1}}$で割って特殊解型に帰着させる.}
\ans 両辺を$3^{n+1}$で割ると,
 \[\frac{a_{n+1}}{3^{n+1}}=\frac23\cdot \frac{a_n}{3^n}+\frac13\]
 であるから,$b_n=\dfrac{a_n}{3^n}$とおくと,数列$\{b_n\}$は$b_1=\dfrac{5}{3},~b_{n+1}=\dfrac23 b_n+\dfrac13$で定まる数列である.例題1と同じようにして解くと,$b_n=\left(\frac23\right)^n+1$なので,
 \[\boldsymbol{a_n=3^nb_n=2^n+3^n}\]
 \begin{toi}
次の漸化式で定まる数列$\{a_n\}$の一般項を求めよ.\\
\begin{minipage}{0.5\linewidth}
\begin{itemize}
    \item [(1)]$a_1=2,~a_{n+1}=3a_n+2^n$
\end{itemize}
\end{minipage}
\begin{minipage}{0.5\linewidth}
\begin{itemize}
    \item [(2)]$a_1=5,~a_{n+1}=3a_n+5\cdot3^n$
\end{itemize}
\end{minipage}
\end{toi}
\begin{exque}
         $a_1=0,~a_2=1,~a_{n+2}=a_{n+1}+6a_n$で定まる数列$\{a_n\}$の一般項を求めよ.
\end{exque}
\ascboxG{\textbf{Point.}}隣接3項間漸化式「$\boldsymbol{a_{n+2}=pa_{n+1}+qa_n}$」は,特性方程式$\boldsymbol{x^2=px+q}$の解$\alpha,\beta$を用いて
  \[
    \boldsymbol{a_{n+2}-\alpha a_{n+1}=\beta(a_{n+1}-\alpha a_n)}
\]
  に変形する\footnote{特性方程式が2つの異なる解を持つときは,
  \begin{align*}
  a_{n+2}-\alpha a_{n+1}&=\beta(a_{n+1}-\alpha a_n)\\
  a_{n+2}-\beta a_{n+1}&=\alpha(a_{n+1}-\beta a_n)
  \end{align*}
  と2通りの変形をし,$\{a_{n+1}-\alpha a_n\},~\{a_{n+1}-\beta a_n\}$を求めることで解く方法もある.今回扱う方法は,特性方程式が重解を持つ場合でも使える汎用的な方法である.}.
  \ans 
  特性方程式$x^2=x+6$の解は$x=3,-2$であるから,
  \[a_{n+2}-(-2)a_{n+1}=3(a_{n+1}-(-2)a_n)\]
  つまり,\[a_{n+2}+2a_{n+1}=3(a_{n+1}+2a_n)\]
  と変形できる.これより,数列$\{a_{n+1}+2a_n\}$は初項$1$,公比3の等比数列なので,$a_{n+1}+2a_n=3^{n-1}$である.これは指数関数型の漸化式なので,例題2と同じようにして解くと,
  \[ \boldsymbol{a_n=\frac15\left(3^{n-1}-(-2)^{n-1}\right)}\]
   \begin{toi}
次の漸化式で定まる数列$\{a_n\}$の一般項を求めよ.\\
\begin{minipage}{0.5\linewidth}
\begin{itemize}
    \item [(1)]    $a_1=2,~a_2=1,~a_{n+2}=a_{n+1}+2a_n$
\end{itemize}
\end{minipage}
\begin{minipage}{0.5\linewidth}
\begin{itemize}
    \item [(2)]$a_1=0,~a_2=2,~a_{n+2}=4a_{n+1}-4a_n$
\end{itemize}
\end{minipage}
\end{toi}
\newpage
\ascboxA{\textbf{復習用問題}}
\begin{toi}
次の漸化式で定まる数列$\{a_n\}$の一般項を求めよ.\\
\begin{minipage}{0.5\linewidth}
\begin{itemize}
    \item [(1)]$a_1=6,~a_{n+1}=3a_n-8$
\end{itemize}
\end{minipage}
\begin{minipage}{0.5\linewidth}
\begin{itemize}
    \item [(2)]$a_1=2,~a_{n+1}=6a_n-15$
\end{itemize}
\end{minipage}

\end{toi}

\begin{toi}
次の漸化式で定まる数列$\{a_n\}$の一般項を求めよ.\\
\begin{minipage}{0.5\linewidth}
\begin{itemize}
    \item [(1)]$a_1=2,~a_{n+1}=3a_n+2^n$
\end{itemize}
\end{minipage}
\begin{minipage}{0.5\linewidth}
\begin{itemize}
    \item [(2)]$a_1=-30,~a_{n+1}=a_n+\dfrac{4}{3^n}$
\end{itemize}
\end{minipage}
\end{toi}
   \begin{toi}
次の漸化式で定まる数列$\{a_n\}$の一般項を求めよ.\\
\begin{minipage}{0.5\linewidth}
\begin{itemize}
    \item [(1)]    $a_1=1,~a_2=2,~a_{n+2}=a_{n+1}+6a_n$
\end{itemize}
\end{minipage}
\begin{minipage}{0.5\linewidth}
\begin{itemize}
    \item [(2)]$a_1=1,~a_2=1,~a_{n+2}=a_{n+1}+a_n$
\end{itemize}
\end{minipage}
\end{toi}
\end{document}