\documentclass[a4paper,11pt]{ltjsarticle}
\usepackage{base}
\title{}
\author{}
\date{}
\newcommand{\printheader}[2]{
\begin{tikzpicture}[remember picture, overlay]
\node[yshift=-2.5cm, anchor=north] at (current page.north) {
\begin{tikzpicture}
\fill[gray!20] (0,0) rectangle (\textwidth, 2cm);
\fill[gray!80] (0,0) rectangle (0.2cm, 2cm);
\draw[gray!80, thick] (0,0) -- (	\textwidth, 0);
\node[anchor=west, text width=\textwidth-1cm, inner xsep=1cm] at (0, 1.25cm) {
\parbox[b]{\linewidth}{
{\color{gray!50!black}\bfseries #1} \par
\vspace{0.2em}
{\huge\bfseries #2}
}
};
\end{tikzpicture}
};
\end{tikzpicture}
\vspace{0.5cm}
}
\begin{document}
\printheader{単元別演習 2次関数③}{総合演習}
\begin{toi}
$f(x)=-x^2+2ax-a^2-1~(-1\leqq x\leqq 1)$の最大値を求めよ.
\end{toi}
\begin{toi}
$x, y$ を実数とし, $x^2 - xy + y^2 = 1$ を満たすとする。$t = x + y$ とおくとき, 次の問いに答えよ。
\begin{enumerate}
    \item[(1)] $xy$ を $t$ を用いて表せ。
    \item[(2)] $t$ の値の範囲を求めよ。
    \item[(3)] $2x + 3xy + 2y$ の最大値および最小値と, そのときの $x, y$ の値を求めよ。
\end{enumerate}
\hfill (22 滋賀大)
\end{toi}
\begin{toi}
$a, b$ を正の定数とする。$x, y$ を $\displaystyle \frac{x^2}{a^2} + \frac{y^2}{b^2} = 1$ を満たす実数とするとき,
$z = \left(\frac{x}{a}\right)^4 + \left(\frac{y}{b}\right)^4$ のとりうる値の範囲は $\fbox{ ② } \le z \le \fbox{ ③ }$ である。
\hfill (20 関西大)
\end{toi}
\begin{toi}
2次関数 $f(x) = ax^2 - 2ax + b$ ($a, b$ は定数) は区間 $0 \le x \le 3$ における最大値が $3$, 最小値が $-5$ である。このとき, $a, b$ の値の組をすべて求めよ。
\hfill (名城大)
\end{toi}
\begin{toi}
 $a$ を定数とするとき, 2次関数 $y = x^2 - 2ax + 2a^2$ について
\begin{enumerate}
    \item[(1)] 区間 $0 \le x \le 2$ におけるこの関数の最大値と最小値を求めよ.
    \item[(2)] 区間 $0 \le x \le 2$ におけるこの関数の最小値が $20$ であるとき, $a$ の値を求めよ.
\end{enumerate}
\hfill (宇都宮大)
\end{toi}
\begin{toi}
 2次方程式 $mx^2 - x - 2 = 0$ の2つの実数解が, それぞれ以下のようになるための $m$ の条件を求めよ.
\begin{enumerate}
    \item[(1)] 2つの解がともに $-1$ より大きい.
    \item[(2)] 1つの解は $1$ より大きく, 他の解は $1$ より小さい.
    \item[(3)] 2つの解の絶対値がともに $1$ より小さい.
\end{enumerate}
\hfill (岐阜大)
\end{toi}
\newpage
\begin{toi}
\begin{enumerate}
    \item[(1)] $a$ は実数の定数とする.2次関数 $f(x) = 2x^2 - 4ax + a + 1$ が $x \ge 0$ においてつねに $f(x) > 0$ を満たすような, $a$ の値の範囲を求めよ.
    \item[(2)] $0 \le x \le 2$ を満たすすべての実数 $x$ に対して, $x^2 - 2ax + a - 3 \le 0$ が成り立つような定数 $a$ の値の範囲を求めよ.
\end{enumerate}
\hfill (秋田大, 千葉工業大)
\end{toi}

\end{document}