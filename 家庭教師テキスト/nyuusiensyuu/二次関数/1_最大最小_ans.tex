\documentclass[a4paper,11pt]{ltjsarticle}
\usepackage{base}
\title{}
\author{}
\date{}
\newcommand{\printheader}[2]{
\begin{tikzpicture}[remember picture, overlay]
\node[yshift=-2.5cm, anchor=north] at (current page.north) {
\begin{tikzpicture}
\fill[gray!20] (0,0) rectangle (\textwidth, 2cm);
\fill[gray!80] (0,0) rectangle (0.2cm, 2cm);
\draw[gray!80, thick] (0,0) -- (	\textwidth, 0);
\node[anchor=west, text width=\textwidth-1cm, inner xsep=1cm] at (0, 1.25cm) {
\parbox[b]{\linewidth}{
{\color{gray!50!black}\bfseries #1} \par
\vspace{0.2em}
{\huge\bfseries #2}
}
};
\end{tikzpicture}
};
\end{tikzpicture}
\vspace{0.5cm}
}
\begin{document}
\printheader{単元別演習 2次関数①}{最大・最小(解答)}
        \begin{toi}
        $a$ を負の定数とする.2次関数 $f(x) = ax^2 - 2ax + b$ の $-2 \le x \le 2$ における最大値が $12$, 最小値が $-6$ のとき、$a, b$ の値を求めよ.
\hfill (04 同志社女子大)
        \end{toi}

        \ans 
        与えられた関数を平方完成すると\[f(x) = a(x^2 - 2x) + b = a(x - 1)^2 - a + b\]であるから,$a<0$ よりグラフは$(1,-a+b)$を頂点とする上に凸の放物線である.
 軸 $x=1$ は定義域$-2 \le x \le 2$ に含まれるので, $f(x)$ は$x=1$ で最大値 $f(1)=-a+b$,$x=-2$ で最小値 $f(-2)=8a+b$をとる.
最大値が $12$, 最小値が $-6$ であるから,
\begin{align*}
    -a+b &= 12 \quad \cdots ① \\
    8a+b &= -6 \quad \cdots ②
\end{align*}
を得る.これを解いて,$\boldsymbol{a=-2, b=10}$.\\

\begin{toi}
関数 $y = (x^2 -3x)^2 -9(x^2 -3x) ~(1\leqq x\leqq4)$ の最大値と最小値を求めよ.\hfill (05 慶應義塾大)
\end{toi}
\ans 
$t = x^2 - 3x$ とおくと, $y = t^2 - 9t$ と表せる.まず,  $t$ の値の範囲を求めよう.
\[t = x^2 - 3x = \left(x - \frac{3}{2}\right)^2 - \frac{9}{4}\]
であるから,$1\leqq x\leqq 4$のとき, $-\dfrac{9}{4} \leqq t \leqq 4$ である.
よって, $y = t^2 - 9t~\left(-\dfrac{9}{4} \leqq t \leqq 4\right)$ の最大値と最小値を考えればよい.平方完成すると
\[y = \left(t - \frac{9}{2}\right)^2 - \frac{81}{4}\]
であるから, $t = -\dfrac{9}{4}\left(x=\dfrac32\right)$ で最大値$\boldsymbol{\dfrac{405}{16}}$, $t=4(x=4)$で最小値は$\boldsymbol{-20}$をとる.
 \\
\newpage
\begin{toi}
関数 $y =-2\sin^2x + 5\sin x +3~(0\leqq x\leqq 2\pi)$ の最小値を求めよ.
\end{toi}
\ans $t = \sin x$ とおくと,$y = -2t^2+5t+3$である.$0 \le x \le 2\pi$ より, $-1 \le t \le 1$ に注意する.平方完成すると
\begin{align*}
    y = -2t^2+5t+3=-2\left(t - \frac{5}{4}\right)^2 + \frac{49}{8}
\end{align*}
であるから,グラフは $\left( \dfrac{5}{4},\dfrac{49}{8}\right)$を頂点とする上に凸の放物線である.よって,
$t=-1$で最小値$ -4$をとる.$t=-1$のとき$x=\dfrac32\pi$であるから,結局,$\boldsymbol{x=\dfrac32\pi}$で最小値$\boldsymbol{-4}$をとる.
 \\
\begin{toi}
$0 \le x \le 3$ のとき、関数 $f(x) = 2x^2 - 4ax + a + a^2$ の最小値 $m$ が $0$ となるような定数 $a$ の値をすべて求めよ.
\hfill (86 東京大)
\end{toi}
\ans 
関数 $f(x)$ を平方完成すると,
\[f(x) = 2(x^2 - 2ax) + a + a^2 = 2(x - a)^2 - 2a^2 + a + a^2 = 2(x - a)^2 - a^2 + a\]
より,グラフは$(a, - a^2 + a)$ を頂点とする下に凸の放物線である.
\begin{itemize}
    \item [(1)]$a < 0$ のとき\\
グラフは定義域内で単調増加となるので,$m=f(0)=a^2+a$.
$m=0$なら $a^2+a=a(a+1)=0$なので,$a<0$ より$\boldsymbol{a=-1}$.
\item[(2)]${0 \le a \le 3}$ のとき
頂点が定義域に含まれるので,$m=f(a)=-a^2+a$.
$m=0$ なら $-a^2+a= -a(a-1)=0$なので,$0 \le a \le 3$ より$\boldsymbol{a=0, 1}$.
\item[(3)] ${a > 3}$ のとき
グラフは定義域内で単調減少となるので,$m=f(3)= a^2 - 11a + 18$.$m=0$ なら$a^2 - 11a + 18  (a-2)(a-9)=0$なので,$a>3$ より$\boldsymbol{a=9}$.
\end{itemize}
以上より, 求める $a$ の値は $\boldsymbol{a = -1, 0, 1, 9}$.\\
\newpage
\begin{toi}
2次関数$f(x)=ax^2-6ax+b$は,区間$1\leqq x\leqq4$において最大値11,最小値8をとる.このとき,$a>0$ならば,$b=~\fbox{ ア }$であり,$a<0$ならば,$b=\fbox{ イ }$である.\hfill{(06 愛知工業大)}
\end{toi}
\ans 
$f(x)$を平方完成すると
\[f(x) = ax^2-6ax+b = a(x-3)^2-9a+b\]
より,グラフは$(3,-9a+b)$を頂点とする放物線である.
\begin{itemize}
    \item [(1)] ${a>0}$ のとき \\
グラフは下に凸なので,$x=3$で最小値 $-9a+b=8$,$x=1$で最大値$ -5a+b=11$をとる.これより,$a=\dfrac{3}{4},\boldsymbol{b =\dfrac{59}{4}\cdots~\fbox{ ア }~
}$.これは$a>0$を満たしているのでOK.\\
\item[(2)]${a<0}$ のとき
グラフは上に凸なので,$x=1$で最小値 $-5a+b=8$,$x=3$で最大値$-9a+b=11$をとる.これより,$a=-\dfrac{3}{4},\boldsymbol{b =\dfrac{17}{4}\cdots~\fbox{ イ }~
}$.これは$a<0$を満たしているのでOK.\\
\end{itemize}

\end{document}