\documentclass[a4paper,11pt]{ltjsarticle}
\usepackage{base}
\title{}
\author{}
\date{}
\newcommand{\printheader}[2]{
\begin{tikzpicture}[remember picture, overlay]
\node[yshift=-2.5cm, anchor=north] at (current page.north) {
\begin{tikzpicture}
\fill[gray!20] (0,0) rectangle (\textwidth, 2cm);
\fill[gray!80] (0,0) rectangle (0.2cm, 2cm);
\draw[gray!80, thick] (0,0) -- (	\textwidth, 0);
\node[anchor=west, text width=\textwidth-1cm, inner xsep=1cm] at (0, 1.25cm) {
\parbox[b]{\linewidth}{
{\color{gray!50!black}\bfseries #1} \par
\vspace{0.2em}
{\huge\bfseries #2}
}
};
\end{tikzpicture}
};
\end{tikzpicture}
\vspace{0.5cm}
}
\begin{document}
\printheader{単元別演習 2次関数②}{2変数関数の最大・最小}
\begin{exque}
$x>0,~y>0,~x+y=1$のとき,$f(x,y)=\displaystyle{\left(2+\frac1x\right)\left(2+\frac1y\right)}$の最小値を求めよ.
\end{exque}
\ascboxG{\textbf{Point.}}2変数関数の基本は,文字を消して1変数関数に帰着させることである.難関大ではたまに出るので経験があると安心.
\ans
$x+y=1$に注意して展開すると, 
\[f(x,y)=\displaystyle{\left(2+\frac1x\right)\left(2+\frac1y\right)=4+\frac2x+\frac2y+\frac1{xy}}=4+\frac{2(x+y)+1}{xy}=4+\frac{2\cdot 1+1}{xy}=4+\frac{3}{xy}\]
である.$y=1-x$より$y$を消去すると,
\[f(x,y)=4+\frac3{x(1-x)}\]
$f(x,y)$が最小になるのは,$x(1-x)$が最大になるときである.$\displaystyle{x(1-x)=-x^2+x=-\left(x-\frac12\right)^2+\dfrac14}$なので,$x=\dfrac12$のときに$f(x,y)$は$\boldsymbol{最大値16}$をとる.
\begin{toi}
$x+y=1,~0\leqq x\leqq 2$のとき,$x-2y^2$の最大値と最小値を求めよ.\hfill[07 関西大]
\end{toi}
\begin{exque}
    $x^2-2xy+2y^2-2y+4x+6$の最小値とそのときの$x,y$を求めよ.\hfill[14 摂南大]
\end{exque}
\ascboxG{\textbf{Point.}}
文字を消すための式が足りないときは,ひとまず平方完成をしてみるとよい.
\ans 
まずは$x$の関数として平方完成をして,次に残りを$y$の関数として平方完成すると.
\begin{align*}
x^2+2(2-y)x+2y^2-2y+6&=(x+(2-y))^2-(y^2-4y+4)+2y^2-2y+6\\
&=(x-y+2)^2+y^2+2y+2\\
&=(x-y+2)^2+(y+1)^2+1\\
\end{align*} 
と変形できるので,$x-y+2=0,~y+1=0$のときに最小値1をとる.つまり,$\boldsymbol{x=-3,y=-1}$\textbf{で最小値1}.
\begin{toi}
    $x^2-8xy+17y^2+6x-30y+10$の最小値とそのときの$x,y$を求めよ.\hfill[15 北海学園大]
\end{toi}
 \\
\ascboxA{\textbf{復習問題}}
\noindent
\begin{toi}
 $x+y=4,~x\geqq0,~y\geqq0$を満たすとき,$x^2y^2+x^2+y^2+xy$の最大値と最小値を求めよ.
\end{toi}
\begin{toi}
実数$x,y$は$2x+y=2,~x\geqq0,~y\geqq0$を満たすとする.
\begin{itemize}
    \item [(1)]$xy$の最大値と最小値を求めよ.
    \item [(2)]$x^2y^2+4x^2+y^2+2xy$の最大値と最小値を求めよ.
\end{itemize}
\end{toi}
\begin{toi}
 $x^2-4xy+5y^2+2x-2y+7$の最小値とそのときの$x,y$を求めよ.
\end{toi}
\end{document}