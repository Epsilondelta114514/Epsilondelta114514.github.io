\documentclass[a4paper,11pt]{ltjsarticle}
\usepackage{base}
\title{}
\author{}
\date{}
\newcommand{\printheader}[2]{
\begin{tikzpicture}[remember picture, overlay]
\node[yshift=-2.5cm, anchor=north] at (current page.north) {
\begin{tikzpicture}
\fill[gray!20] (0,0) rectangle (\textwidth, 2cm);
\fill[gray!80] (0,0) rectangle (0.2cm, 2cm);
\draw[gray!80, thick] (0,0) -- (	\textwidth, 0);
\node[anchor=west, text width=\textwidth-1cm, inner xsep=1cm] at (0, 1.25cm) {
\parbox[b]{\linewidth}{
{\color{gray!50!black}\bfseries #1} \par
\vspace{0.2em}
{\huge\bfseries #2}
}
};
\end{tikzpicture}
};
\end{tikzpicture}
\vspace{0.5cm}
}
\begin{document}
\printheader{単元別演習 2次関数②}{2変数関数の最大・最小(解答)}
\begin{toi}
$x+y=1,~0\leqq x\leqq 2$のとき,$x-2y^2$の最大値と最小値を求めよ.\hfill(07 関西大)
\end{toi}
\ans 
$x+y=1$ より $y=1-x$を用いて$y$ を消去すると,
\begin{align*}
    x-2y^2 =x - 2(1-x)^2 = -2x^2 + 5x - 2
\end{align*}
である.これを平方完成すると
\[-2x^2 + 5x - 2 = -2\left(x - \frac{5}{4}\right)^2 + \frac{9}{8}\]
であるから,グラフは$\left(\dfrac{5}{4}, \dfrac{9}{8}\right)$を頂点とする上に凸の放物線である.
$0 \leqq x \leqq 2$に注意すると,$x=\dfrac{5}{4}$で最大値$\boldsymbol{\dfrac{9}{8}}$, $x=0$ で最小値$\boldsymbol{-2}$をとることがわかる.\\
 
\begin{toi}
    $x^2-8xy+17y^2+6x-30y+10$の最小値とそのときの$x,y$を求めよ.\hfill(15 北海学園大)
\end{toi}
\ans 
与えられた式を $x$ の2次式と思って平方完成したのち,残りをさらに平方完成すると,
\begin{align*}
    x^2 - 8xy + 17y^2 + 6x - 30y + 10  &= x^2 + (6-8y)x + 17y^2 - 30y + 10 \\
    &= (x-4y+3)^2 + y^2 - 6y + 1\\
    &= (x-4y+3)^2 + (y-3)^2 - 8
\end{align*}
である.これより,$x-4y+3=0, y-3= 0$ ,すなわち、$\boldsymbol{x=9, y=3}$ で最小値 $\boldsymbol{-8}$をとる.\\
 
\newpage
\begin{toi}
 $x+y=4,~x\geqq0,~y\geqq0$を満たすとき,$x^2y^2+x^2+y^2+xy$の最大値と最小値を求めよ.
\end{toi}
\ans 
$x^2y^2+x^2+y^2+xy$は$xとy$の対称式であるから,基本対称式$x+y,xy$で表せる.実際,
\[x^2y^2+x^2+y^2+xy=(xy)^2+(x+y)^2-2xy+xy=(xy)^2-xy+(x+y)^2\]
$x+y=4$であるから,
\[x^2y^2+x^2+y^2+xy=(xy)^2-xy+16.\]
 これは$xy$の2次関数であるから,$t=xy$とおく. $t=xy=x(4-x)$であり,$x \geqq 0$と$y=4-x\geqq 0$ より$0 \leqq x \leqq 4$なので,
\[x(4-x)=-x^2+4x = -(x-2)^2+4\]
より$0 \leqq t \leqq 4$とわかる.
$0\leqq t\leqq 4$のもとで$t^2-t+16$の最大値と最小値を求めよう.
\[t^2-t+16=\left(t-\frac{1}{2}\right)^2 + \frac{63}{4}\]
なので,グラフは$\left(\dfrac{1}{2}, \dfrac{63}{4}\right)$ を頂点とする下に凸の放物線.よって,$t=\dfrac{1}{2}$ で最小値 $\boldsymbol{\dfrac{63}{4}}$,$t=4$ で最大値 $\boldsymbol{28}$をとる.\\
 
\begin{toi}
実数$x,y$は$2x+y=2,~x\geqq0,~y\geqq0$を満たすとする.
\begin{itemize}
    \item [(1)]$xy$の最大値と最小値を求めよ.
    \item [(2)]$x^2y^2+4x^2+y^2+2xy$の最大値と最小値を求めよ.
\end{itemize}
\end{toi}
\ans 
$x \geqq 0$, $y =2-2x\geqq 0$より$0 \leqq x \leqq 1$であることに注意する.
\begin{itemize}
    \item [(1)]$xy = x(2-2x) = -2\left(x-\dfrac{1}{2}\right)^2+\dfrac{1}{2}$なので,$x=\dfrac{1}{2}$で最大値 $\boldsymbol{\dfrac{1}{2}}$, $x=0, 1$ で最小値 $\boldsymbol{0}$.
\item[(2)] $x^2y^2+4x^2+y^2+2xy$を$2x+y=2$が使えるように変形すると,
\[x^2y^2+4x^2+y^2+2xy=(xy)^2+(2x+y)^2-4xy+2xy=(xy)^2-2xy+4.\]
これは$xy$の2次関数であるから,$t=xy$とおくと,(1)より$0\leqq t\leqq \dfrac12$.平方完成すると
\[t^2-2t+4=(t-1)^2+3\]
であるから,$t=0$で最大値 $\boldsymbol{4}$,$t=\dfrac12$で最小値 $\boldsymbol{\dfrac{13}{4}}$をとる.
\end{itemize}
\begin{toi}
 $x^2-4xy+5y^2+2x-2y+7$の最小値とそのときの$x,y$を求めよ.
\end{toi}
\ans 与えられた式を $x$ の2次式と思って平方完成したのち,残りをさらに平方完成すると,
\begin{align*}
    x^2 - 4xy + 5y^2 + 2x - 2y + 7 &= (x-2y+1)^2 + y^2 + 2y + 6\\
    &= (x-2y+1)^2 + (y+1)^2 + 5 
\end{align*}
これより$x-2y+1=0, y+1=0$,すなわち$\boldsymbol{x=-3, y=-1}$ で最小値 $\boldsymbol{5}$をとる.
\end{document}