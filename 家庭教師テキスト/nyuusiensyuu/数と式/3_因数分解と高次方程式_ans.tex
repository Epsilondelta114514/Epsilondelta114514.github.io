\documentclass[a4paper,11pt]{ltjsarticle}
\usepackage{base}
\title{}
\author{}
\date{}
\newtcolorbox{rembox}[1][]{enhanced,
    before skip=2mm,after skip=3mm,fontupper=\gtfamily\sffamily,
    boxrule=0.4pt,left=5mm,right=2mm,top=1mm,bottom=1mm,
    colback=yellow!50,
    colframe=yellow!20!black,
    sharp corners,rounded corners=southeast,arc is angular,arc=3mm,
    underlay={
        \path[fill=tcbcolback!80!black] ([yshift=3mm]interior.south east)--++(-0.4,-0.1)--++(0.1,-0.2);
        \path[draw=tcbcolframe,shorten <=-0.05mm,shorten >=-0.05mm] ([yshift=3mm]interior.south east)--++(-0.4,-0.1)--++(0.1,-0.2);
        \path[fill=yellow!50!black,draw=none] (interior.south west) rectangle node[white]{\Huge\bfseries !} ([xshift=4mm]interior.north west);
    },
drop fuzzy shadow,#1}
\newcommand{\printheader}[2]{
\begin{tikzpicture}[remember picture, overlay]
\node[yshift=-2.5cm, anchor=north] at (current page.north) {
\begin{tikzpicture}
\fill[gray!20] (0,0) rectangle (\textwidth, 2cm);
\fill[gray!80] (0,0) rectangle (0.2cm, 2cm);
\draw[gray!80, thick] (0,0) -- (	\textwidth, 0);
\node[anchor=west, text width=\textwidth-1cm, inner xsep=1cm] at (0, 1.25cm) {
\parbox[b]{\linewidth}{
{\color{gray!50!black}\bfseries #1} \par
\vspace{0.2em}
{\huge\bfseries #2}
}
};
\end{tikzpicture}
};
\end{tikzpicture}
\vspace{0.5cm}
}
\begin{document}
\printheader{単元別演習 数と式③}{因数分解・高次方程式(解答)}
\begin{toi}
次の式を因数分解せよ.\\
\begin{minipage}{0.5\linewidth}
\begin{itemize}
    \item [(1)]$x^3-4x^2-7x+10$
\end{itemize}
\end{minipage}
\begin{minipage}{0.5\linewidth}
\begin{itemize}
    \item [(2)]$x^3+2x^2-2x-1$
\end{itemize}
\end{minipage}
\end{toi}
\ans 
\begin{itemize}
    \item[(1)] $x^3-4x^2-7x+10 =\boldsymbol{(x-1)(x+2)(x-5)}.$
    \item[(2)] $x^3+2x^2-2x-1 = \boldsymbol{(x-1)(x^2+3x+1)}.$
\end{itemize}
\begin{toi}
次の方程式を解け.\\[5pt]
\begin{minipage}{0.5\linewidth}
\begin{itemize}
    \item [(1)]$x^3+2x^2-2x-1=0$
    \item [(3)]$x^4+3x^2-4=0$
\end{itemize}
\end{minipage}
\begin{minipage}{0.5\linewidth}
\begin{itemize}
    \item [(2)]$x^3-5x^2+6x=0$
    \item [(4)]$x^3+3x^2+3x+1=0$
\end{itemize}
\end{minipage}
\end{toi}
\ans 
\begin{itemize}
    \item [(1)]$x^3+2x^2-2x-1=(x-1)(x^2+3x+1)=0$ より $\boldsymbol{x=1, \dfrac{-3\pm\sqrt{5}}{2}}$.\\
    \item [(2)]$x^3-5x^2+6x=x(x^2-5x+6)= x(x-2)(x-3)=0$ より $\boldsymbol{x=0, 2, 3}$.\\
    \item [(3)]$x^4+3x^2-4=(x^2+4)(x^2-1)= (x^2+4)(x-1)(x+1)=0$ より $\boldsymbol{x=\pm 1, \pm 2i}.$\\
    \fbox{\textbf{別解}}~\\
    $X=x^2$とおくと,$x^4+3x^2-4=X^2+3x-4=(X+4)(X-1)$より$X=-4,1$.\\よって,$x^2=-4,1$より,$\boldsymbol{x=\pm 1, \pm 2i}.$\\
    \item [(4)]$x^3+3x^2+3x+1=(x+1)^3=0$より, $\boldsymbol{x=-1}$ .
\end{itemize}
\begin{toi}
次の式を因数分解せよ.
\begin{itemize}
    \item [(1)]$(x^2+2x-30)(x^2+2x-8)-135$\hfill(北海学園大)
    \item [(2)]$(x-4)(x-2)(x+1)(x+3)+24$\hfill(東洋大)
    \item [(3)]$x(x+1)(x+2)(x+3)+1$\hfill(松山大)
    \item [(4)]$(x+1)(x+2)(x+3)(x+4)-3$\hfill(九州東海大)
\end{itemize}
\end{toi}
\ans 
\begin{itemize}
      \item[(1)] $x^2+2x=X$ とおくと$(X-30)(X-8)-135 = X^2-38X+105 = (X-3)X(A-35)$.\\
よって,$(与式)=(x^2+2x-3)(x^2+2x-35) = \boldsymbol{(x-1)(x+3)(x-5)(x+7)}$\\
    \item[(2)] $(与式)=(x+3)(x-4)(x+1)(x-2)+24=(x^2-x-12)(x^2-x-2)+24$であるから,$x^2-x=X$ とおくと,$(x-12)(x-2)+24 = X^2-14X+48=(X-6)(X-8)$.\\
    よって,$(与式)=(x^2-x-6)(x^2-x-8) = \boldsymbol{(x+2)(x-3)(x^2-x-8)}$\\
    \item [(3)]$(与式)=(x^2+3x)(x^2+3x+2)+1$であるから,$x^2+3x=A$ とおくと,$A(A+2)+1 = (A+1)^2$.よって,$(与式)=\boldsymbol{(x^2+3x+1)^2}$\\
    \item [(4)]$(与式)=(x+1)(x+4)(x+3)(x+2)-3=(x^2+5x+4)(x^2+5x+6)-3$であるから,$x^2+5x=X$ とおくと,$(A+4)(A+6)-3 = A^2+10A+21=(A+3)(A+7)$.\\
    よって,$(与式)=\boldsymbol{(x^2+5x+3)(x^2+5x+7)}$
\end{itemize}
\begin{toi}
次の式を因数分解せよ.
\begin{itemize}
    \item [(1)]$2x^2+5xy+3y^2-3x-5y-2$\hfill(京都産業大)
    \item [(2)]$2x^2+3xy-2y^2+5y-2$\hfill(京都産業大)
        \item [(3)]$a^3+a^2-2a-a^2b-ab+2b$\hfill(摂南大)
\end{itemize}
\end{toi}
\ans 文字を1つ選び,降べきの順で整理するとよい. (3)は共通因数$a^2+a-2$でくくる方がはやい.
\begin{align*}
    &(1)~& 2x^2+(5y-3)x+(3y^2-5y-2) &= 2x^2+(5y-3)x+(y-2)(3y+1)\\
     &&&=\{x+(y-2)\}\{2x+(3y+1)\} \\
     &&&= \boldsymbol{(x+y-2)(2x+3y+1)}\\
        &(2)~& 2x^2+3xy-2y^2+5y-2 &= 2x^2+3yx-(2y-1)(y-2)\\
     &&&=\{x+(2y-1)\}\{2x-(y-2)\}\\
     &&&= \boldsymbol{(x+2y-1)(2x-y+2)}\\
        &(3)~& a^3+a^2-2a-a^2b-ab+2b&=a(a^2+a-2)-b(a^2+a-2) \\
        &&& = (a-b)(a^2+a-2) \\
        &&&= \boldsymbol{(a-b)(a-1)(a+2)}
     \end{align*}

\end{document}