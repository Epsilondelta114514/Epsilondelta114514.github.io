\documentclass[a4paper,11pt]{ltjsarticle}
\usepackage{base}
\title{}
\author{}
\date{}
\newcommand{\printheader}[2]{
\begin{tikzpicture}[remember picture, overlay]
\node[yshift=-2.5cm, anchor=north] at (current page.north) {
\begin{tikzpicture}
\fill[gray!20] (0,0) rectangle (\textwidth, 2cm);
\fill[gray!80] (0,0) rectangle (0.2cm, 2cm);
\draw[gray!80, thick] (0,0) -- (	\textwidth, 0);
\node[anchor=west, text width=\textwidth-1cm, inner xsep=1cm] at (0, 1.25cm) {
\parbox[b]{\linewidth}{
{\color{gray!50!black}\bfseries #1} \par
\vspace{0.2em}
{\huge\bfseries #2}
}
};
\end{tikzpicture}
};
\end{tikzpicture}
\vspace{0.5cm}
}
\begin{document}
\printheader{単元別演習 数と式②}{整式の除法}
\begin{exque}
    整式$f(x)$を$x+2$で割った余りが3,$x-3$で割った余りが$-1$のとき,$f(x)$を$x^2-x-6$で割った余りを求めよ.\hfill[05~立教大]
\end{exque}
\ascboxG{\textbf{Point.}}商を$q(x)$,余りを$r(x)$として$\boldsymbol{f(x)=g(x)q(x)+r(x)}$の形で表す.
\ans 
$f(x)$を$x+2$で割った余りが$3$であるから,商を$q_1(x)$とすると
\[f(x)=(x+2)q_1(x)+3~\cdots ①\]
また,$f(x)$を$x-3$で割った余りは$-1$であるから,商を$q_2(x)$とすると
\[f(x)=(x-3)q_2(x)-1~\cdots ②\]
と書ける.$f(x)$を$x^2-x-6$で割った余りは一次式であるから,これを$ax+b$とおく.商を$q(x)$とすると
\[f(x)=(x^2-x-6)q(x)+ax+b=(x-3)(x+2)q(x)+ax+b~\cdots ③\]
と表せる.①より$f(-2)=3$,②より$f(3)=-1$なので,③より連立方程式
\[\begin{dcases}
-2a+b&=3\\
3a+b&=-1
\end{dcases}\]
を得る.これを解いて,$a=-\dfrac{4}{5},~b=\dfrac75$となるので,求める余りは$\boldsymbol{-\dfrac45 x+\dfrac75}$である.
\ascboxG{\textbf{補足.}}①,②を書かずに\textbf{因数定理}を用いてもよい.問題文より直ちに$f(-2)=3,~f(3)=-1$がわかる.
\begin{ascolorbox17}{\textbf{因数定理}}
整式$f(x)$を$x-a$で割った余りは$f(a)$である.
\end{ascolorbox17}
\noindent \textbf{証明.}\\
$f(x)$を$x-a$で割ったときの商を$q(x)$,余りを$r$とすると,$f(x)=(x-a)q(x)+r$と書ける.これより,$r=f(a)$がしたがう.  (証明終)\\
\newpage
\begin{toi}
   整式$f(x)$を$x+1$で割ったあまりが1,$x-2$で割った余りが$4$のとき,$f(x)$を$x^2-x-2$で割った余りを求めよ.
\end{toi}
\begin{toi}
$a$ を定数,$n$ を正の整数とする.$x$ の整式$f(x) = x^n + 2x^{n-1} - a$が $x+1$ で割り切れるとき,次の問いに答えよ.

\begin{enumerate}
  \item[(1)] $a$ の値を求めよ.
  \item[(2)] $f(x)$ を $x^2 - 1$ で割ったときの余りを求めよ.
\end{enumerate}
\hfill[00~佐賀大]
\end{toi}
 
\begin{exque}
    $ax^3+bx^2-2$が$(x-1)^2$で割り切れるとき,$a,b$の値を求めよ.\hfill[06~早稲田大 改]
\end{exque}
\ascboxG{\textbf{Point.}}
$(x-a)^n$で割る問題の多くは,例題1と同じようにやると条件が足りずに手詰まりとなる.この系統の問題は,\textbf{微分を使って条件式を増やす}と簡単に解ける.
\ans     $ax^3+bx^2-2$が$(x-1)^2$で割り切れるから,商を$q(x)$とすると
\[ax^3+bx^2-2=(x-1)^2q(x)~\cdots (\text{A})\]
と書ける.ひとまず(A)に$x=1$を代入すると,$a+b-2=0~\cdots ①$がわかる.\\
 ①だけだと$a,b$が求まらないので,式がもう1本必要である.そこで,\textbf{(A)の両辺を$x$で微分する}と,
\[3ax^2+2bx=2(x-1)q(x)+(x-1)^2q'(x)~\cdots(\text{B})\]
である.(B)に$x=1$を代入すれば,$3a+2b=0~\cdots ②$がわかる.①と②より,$\boldsymbol{a=-4,~b=6}$.
\begin{toi}
$x^{10}$を$(x-1)^2$で割った余りを求めよ.
\end{toi}
\begin{toi}
$x^{n}$を$(x-1)^2$で割った余りを求めよ.
\end{toi}
\newpage
\ascboxA{\textbf{復習問題}}
\begin{toi}
因数定理の主張を述べ,それを証明せよ.
\end{toi}
\begin{toi}
$(x+1)^{12}$を$x^2-1$で割った余りを求めよ.\hfill[08~日本歯科大]
\end{toi}
\begin{toi}

$n$ は $3$ 以上の奇数として,多項式$P(x) = x^n - ax^2 - bx + 2$
を考える.$P(x)$ が $x^2 - 4$ で割り切れるときは$a = \fbox{{あ}}~, \quad
b = \fbox{い}$
であり,$(x+1)^2$ で割り切れるときは
$a = \fbox{{う}}~, \quad
b = \fbox{え}$
である.\hfill[11~慶應義塾大]
\end{toi}
\begin{toi}
整式 $P(x)$ を $(x-1)^2$ で割ったときの余りが $4x-5$ で,$x+2$ で割ったときの余りが $-4$ である.
\begin{enumerate}
  \item[(1)] $P(x)$ を $x-1$ で割ったときの余りを求めよ.
  \item [(2)]$P(x)$ を $(x-1)(x+2)$ で割ったときの余りを求めよ.
  \item [(3)]$P(x)$ を $(x-1)^2(x+2)$ で割ったときの余りを求めよ.
\end{enumerate}
\hfill[山形大]
\end{toi}
\end{document}