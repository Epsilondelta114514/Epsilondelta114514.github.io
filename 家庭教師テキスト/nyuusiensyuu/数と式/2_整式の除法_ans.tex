\documentclass[a4paper,11pt]{ltjsarticle}
\usepackage{base}
\title{}
\author{}
\date{}
\newcommand{\printheader}[2]{
\begin{tikzpicture}[remember picture, overlay]
\node[yshift=-2.5cm, anchor=north] at (current page.north) {
\begin{tikzpicture}
\fill[gray!20] (0,0) rectangle (\textwidth, 2cm);
\fill[gray!80] (0,0) rectangle (0.2cm, 2cm);
\draw[gray!80, thick] (0,0) -- (	\textwidth, 0);
\node[anchor=west, text width=\textwidth-1cm, inner xsep=1cm] at (0, 1.25cm) {
\parbox[b]{\linewidth}{
{\color{gray!50!black}\bfseries #1} \par
\vspace{0.2em}
{\huge\bfseries #2}
}
};
\end{tikzpicture}
};
\end{tikzpicture}
\vspace{0.5cm}
}
\begin{document}
\printheader{単元別演習 数と式②}{整式の除法(解答)}

\begin{toi}
   整式$f(x)$を$x+1$で割った余りが1,$x-2$で割った余りが$4$のとき,$f(x)$を$x^2-x-2$で割った余りを求めよ.
\end{toi}
\ans 
求める余りは1次以下の整式なので,$ax+b$とおく.商を$q(x)$とすると,
\[f(x)=(x^2-x-2)q(x)+ax+b=(x-2)(x+1)q(x)+ax+b\]
と書ける.因数定理より$f(-1)=1, f(2)=4$であるから,
\[\begin{cases}
f(-1) = -a+b = 1 \\
f(2) = 2a+b=4
\end{cases}\]
この連立方程式を解くと,$a=1, b=2$を得る.よって求める余りは $\boldsymbol{x+2}$.
\begin{toi}
$a$ を定数,$n$ を正の整数とする.$x$ の整式$f(x) = x^n + 2x^{n-1} - a$が $x+1$ で割り切れるとき,次の問いに答えよ.

\begin{enumerate}
  \item[(1)] $a$ の値を求めよ.
  \item[(2)] $f(x)$ を $x^2 - 1$ で割ったときの余りを求めよ.
\end{enumerate}
\hfill[00~佐賀大]
\end{toi}
\ans 
\begin{itemize}
    \item [(1)] $f(x)$が$x+1$で割り切れるので,因数定理より$f(-1)=0$である.
    \[f(-1)=(-1)^n+2(-1)^{n-1}-a = (-1)^{n-1}(-1+2)-a = (-1)^{n-1}-a=0\]
    よって,$\boldsymbol{a=(-1)^{n-1}}$.
    \item [(2)] $f(x)$を$x^2-1=(x-1)(x+1)$で割った余りを$rx+s$とおく.
    \[f(x)=(x^2-1)q(x)+rx+s\]
    $f(-1)=0$より,$-r+s=0~\cdots ①$である.また,
    $f(1)=1^n+2\cdot1^{n-1}-a=3-a$より,$r+s=3-a~\cdots ②$.よって,①と②より\[r=s=\dfrac{3-a}{2}\]である.
    (1)より$a=(-1)^{n-1}$であったから,$r=s=\dfrac{3-(-1)^{n-1}}{2}$.\\
したがって,求める余りは $\boldsymbol{\dfrac{3-(-1)^{n-1}}{2}x + \dfrac{3-(-1)^{n-1}}{2}}$.
\end{itemize}

\newpage
\begin{toi}
$x^{10}$を$(x-1)^2$で割った余りを求めよ.
\end{toi}
\ans 
$x^{10}$を$(x-1)^2$で割った商を$q(x)$,余りを$ax+b$とおくと,
\[x^{10}=(x-1)^2q(x)+ax+b \cdots (\text{A})\]
(A)に$x=1$を代入すると,$1=a+b \cdots ①$.次に,(A)の両辺を$x$で微分すると,
\[10x^9=2(x-1)q(x)+(x-1)^2q'(x)+a\]
この式に$x=1$を代入すると,$10=a\cdots ②$.これを①に代入して$b=-9$.\\
よって,求める余りは $\boldsymbol{10x-9}$.
\begin{toi}
$x^{n}$を$(x-1)^2$で割った余りを求めよ.
\end{toi}
\ans 
$x^n$を$(x-1)^2$で割った商を$q(x)$,余りを$ax+b$とおくと,
\[x^n=(x-1)^2q(x)+ax+b \cdots (\text A)\]
(A)に$x=1$を代入すると,$1=a+b \cdots ①$.次に,(A)の両辺を$x$で微分すると,
\[nx^{n-1}=2(x-1)q(x)+(x-1)^2q'(x)+a\]
この式に$x=1$を代入すると,$a=n~\cdots ②$.これを①に代入して$b=1-n$.\\
よって,求める余りは $\boldsymbol{nx+1-n}$.
\begin{toi}
因数定理の主張を述べ,それを証明せよ.
\end{toi}
\ans
\begin{itemize}
\item[\textbf{主張:}]整式$f(x)$を$x-a$で割った余りは$f(a)$である.
\item[\textbf{証明:}]$f(x)$を$x-a$で割ったときの商を$q(x)$,余りを$r$とすると,$f(x)=(x-a)q(x)+r$と書ける.これより,$r=f(a)$がしたがう.\hfill(証明終)
\end{itemize}
\newpage
\begin{toi}
$(x+1)^{12}$を$x^2-1$で割った余りを求めよ.\hfill[08~日本歯科大]
\end{toi}
\ans 
$f(x)=(x+1)^{12}$を$x^2-1=(x-1)(x+1)$で割った余りを$ax+b$とおく.
\[(x+1)^{12}=(x-1)(x+1)q(x)+ax+b\]
因数定理より$2^{12}=f(1)=a+b, 0=f(-1)=-a+b$なので$a=b=2^{11}=2048$.よって,求める余りは $\boldsymbol{2048x+2048}$.
\begin{toi}

$n$ は $3$ 以上の奇数として,多項式$P(x) = x^n - ax^2 - bx + 2$
を考える.$P(x)$ が $x^2 - 4$ で割り切れるときは$a = \fbox{{あ}}~, \quad
b = \fbox{い}$
であり,$(x+1)^2$ で割り切れるときは
$a = \fbox{{う}}~, \quad
b = \fbox{え}$
である.\hfill[11~慶應義塾大]
\end{toi}
\ans 
\begin{itemize}
    \item $P(x)$が$x^2-4=(x-2)(x+2)$で割り切れるとき\\
    因数定理より$P(2)=0, P(-2)=0$なので,
    \[\begin{cases}
    P(2) = 2^n - 4a - 2b + 2 = 0\\
    P(-2) = (-2)^n - 4a + 2b + 2 =0
    \end{cases}\]
    これより,
        \[\begin{cases}
     4a+2b=2^n + 2 \\
    4a-2b=2-2^{n}
    \end{cases}\]
    連立方程式を解くと,$\boldsymbol{a = \dfrac12\cdots (あ)~, \quad b = 2^{n-1}\cdots (い)}$
    \item $P(x)$が$(x+1)^2$で割り切れるとき\\
    $P(x)=(x+1)^2q(x)$と書けるので,$P(-1)=0$.また,この両辺を$x$で微分すると$P'(x)=2(x+1)q(x)+(x+1)^2q'(x)$なので$ P'(-1)=0$.\\
    $P(-1)=0$より,$-a+b+1=0 \cdots ①$,$P'(-1)=0$より,$n+2a-b=0\cdots ②$である.①と②より, $\boldsymbol{a = -n-1\cdots (う)~, \quad b = -n-2\cdots (え)}$
\end{itemize}
\newpage
\begin{toi}
整式 $P(x)$ を $(x-1)^2$ で割ったときの余りが $4x-5$ で,$x+2$ で割ったときの余りが $-4$ である.
\begin{enumerate}
  \item[(1)] $P(x)$ を $x-1$ で割ったときの余りを求めよ.
  \item [(2)]$P(x)$ を $(x-1)(x+2)$ で割ったときの余りを求めよ.
  \item [(3)]$P(x)$ を $(x-1)^2(x+2)$ で割ったときの余りを求めよ.
\end{enumerate}
\hfill[山形大]
\end{toi}
\ans 
$P(x)=(x-1)^2q_1(x)+4x-5 \cdots$①,$P(x)=(x+2)q_2(x)-4 \cdots$②とおく.
\begin{enumerate}
  \item[(1)] 因数定理より$x-1$で割った余りは$P(1)$である.よって,求める余りは①より $P(1)=4\cdot1-5=\boldsymbol{-1}$.
  \item [(2)] $P(x)$を$(x-1)(x+2)$で割った余りを$ax+b$とおく.
  $P(1)=-1, P(-2)=-4$なので,
  \[\begin{cases}
  a+b=-1 \\ -2a+b=-4
  \end{cases}\]
  これを解いて$a=1, b=-2$.よって余りは $\boldsymbol{x-2}$.
  \item[(3)] $P(x)$を$(x-1)^2(x+2)$で割った余りは2次以下の式なので,$r(x)$とおく.
  \[P(x)=(x-1)^2(x+2)q(x)+r(x)\]
  この式より,$P(x)$を$(x-1)^2$で割った余りは,$r(x)$を$(x-1)^2$で割った余りと等しい.
  ①よりこの余りは$4x-5$なので,$r(x)$は定数$a$を用いて
  \[r(x)=a(x-1)^2+4x-5\]
  と書ける.よって,
  \[P(x)=(x-1)^2(x+2)q(x)+a(x-1)^2+4x-5\]
  ここに$x=-2$を代入すると,②より$P(-2)=-4$なので,
  \[-4=P(-2)=9a-13\]
これより$a=1$なので,求める余りは $1\cdot(x-1)^2+4x-5 =\boldsymbol{x^2+2x-4}$.
\end{enumerate}
\end{document}