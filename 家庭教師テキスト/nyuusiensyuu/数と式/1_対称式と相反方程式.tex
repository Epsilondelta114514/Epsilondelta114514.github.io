\documentclass[a4paper,11pt]{ltjsarticle}
\usepackage{base}
\title{}
\author{}
\date{}
\newtcolorbox{rembox}[1][]{enhanced,
    before skip=2mm,after skip=3mm,fontupper=\gtfamily\sffamily,
    boxrule=0.4pt,left=5mm,right=2mm,top=1mm,bottom=1mm,
    colback=yellow!50,
    colframe=yellow!20!black,
    sharp corners,rounded corners=southeast,arc is angular,arc=3mm,
    underlay={
        \path[fill=tcbcolback!80!black] ([yshift=3mm]interior.south east)--++(-0.4,-0.1)--++(0.1,-0.2);
        \path[draw=tcbcolframe,shorten <=-0.05mm,shorten >=-0.05mm] ([yshift=3mm]interior.south east)--++(-0.4,-0.1)--++(0.1,-0.2);
        \path[fill=yellow!50!black,draw=none] (interior.south west) rectangle node[white]{\Huge\bfseries !} ([xshift=4mm]interior.north west);
    },
drop fuzzy shadow,#1}
\newcommand{\printheader}[2]{
\begin{tikzpicture}[remember picture, overlay]
\node[yshift=-2.5cm, anchor=north] at (current page.north) {
\begin{tikzpicture}
\fill[gray!20] (0,0) rectangle (\textwidth, 2cm);
\fill[gray!80] (0,0) rectangle (0.2cm, 2cm);
\draw[gray!80, thick] (0,0) -- (	\textwidth, 0);
\node[anchor=west, text width=\textwidth-1cm, inner xsep=1cm] at (0, 1.25cm) {
\parbox[b]{\linewidth}{
{\color{gray!50!black}\bfseries #1} \par
\vspace{0.2em}
{\huge\bfseries #2}
}
};
\end{tikzpicture}
};
\end{tikzpicture}
\vspace{0.5cm}
}
\begin{document}
\printheader{単元別演習 数と式①}{対称式と相反方程式}
\ascboxA{\textbf{対称式}}
\noindent $x$と$y$の多項式で,$x$と$y$を入れ替えても同じ式になるものを\textbf{対称式}という:\\[3pt]
\[x^3+3xy+y^3,~~~x+y,~~~xy,~~~x^2+2xy+y^2\]
特に,$x+y,~xy$を\textbf{基本対称式}という.\uwave{対称式は,必ず基本対称式の和・差・積で表せる.}
\begin{exque}
    \begin{itemize}
        \item [(1)]$x^3+y^3$を基本対称式$x+y,~xy$で表せ.
        \item [(2)]$x+y=3,~xy=2$のとき,$x^3+y^3$の値を求めよ.
    \end{itemize}
\end{exque}
\ans 
\begin{itemize}
    \item [(1)]$x^3,y^3$を作りたいので,まずは$(x+y)^3$を考えよう.
    \[(x+y)^3=x^3+3x^2y+3xy^2+y^3=x^3+3xy(x+y)+y^3\]
    なので,$\boldsymbol{x^3+y^3=(x+y)^3-3xy(x+y)}$と表せる.
    \item[(2)](1)より$x^3+y^3=(x+y)^3-3xy(x+y)=3^3-3\cdot 2\cdot3=\underline{\boldsymbol{9}}$
\end{itemize}
 
\begin{toi}
\end{toi}$x=\displaystyle{\frac{\sqrt3+1}{\sqrt3-1}},~~y=\displaystyle{\frac{\sqrt3-1}{\sqrt3+1}}$のとき,次の式の値を求めよ.\\[5pt]
\begin{minipage}{0.25\linewidth}
\begin{itemize}
    \item [(1)]$x+y,~xy$
\end{itemize}
\end{minipage}
\begin{minipage}{0.25\linewidth}
\begin{itemize}
    \item [(2)]$x^2+y^2$
\end{itemize}
\end{minipage}
\begin{minipage}{0.25\linewidth}
\begin{itemize}
    \item [(3)]$x^3+y^3$
\end{itemize}
\end{minipage}
\begin{minipage}{0.25\linewidth}
\begin{itemize}
    \item [(4)]$x^4+y^4$
\end{itemize}
\end{minipage}
\begin{exque}
$\displaystyle{x+\frac{1}{x}=\sqrt7}$のとき,$\displaystyle{x^2+\frac{1}{x^2}}$の値を求めよ.
\end{exque}
\ans 
$x,~\dfrac1x$の対称式と見ればいい.$\boldsymbol{\displaystyle{x^2+\frac1{x^2}=\left(x+\frac1x\right)^2-2x\cdot\frac1x}=7-2=5}$
\begin{toi}
$\displaystyle{x+\frac{1}{x}=\sqrt7}$のとき,次の式の値を求めよ.\\[5pt]
\begin{minipage}{0.33\linewidth}
\begin{itemize}
    \item [(1)]$\displaystyle{x^3+\frac{1}{x^3}}$
\end{itemize}
\end{minipage}
\begin{minipage}{0.33\linewidth}
\begin{itemize}
    \item [(2)]$\displaystyle{x^4+\frac{1}{x^4}}$
\end{itemize}
\end{minipage}
\begin{minipage}{0.33\linewidth}
\begin{itemize}
    \item [(3)]$x^5+\dfrac{1}{x^5}$
\end{itemize}
\end{minipage}
\end{toi}
\ascboxA{\textbf{相反方程式}}
 
\begin{exque}
    4次方程式$x^4+3x^3+2x^2+3x+1=0$を解け.
\end{exque}
\ascboxG{\textbf{Point}}\noindent 
この問題のように,係数が左右対称になっている$n$次方程式を\textbf{相反方程式}という($1,3,2,3,1$).正攻法で解いてもいいが,対称式の考え方を使うとうまく解ける.
\ans 
\noindent $x=0$は解ではないので,両辺を$x^2$で割ると,
\[x^2+3x+2+3\cdot\frac1x+\frac{1}{x^2}=x^2+\frac{1}{x^2}+3\left(x+\frac1x\right)+2=0\]
これは$x$と$\dfrac1x$の対称式なので,前の問題のように$X=x+\dfrac1x$で表そう.
\[x^2+\frac{1}{x^2}+3\left(x+\frac1x\right)+2=\left(x+\dfrac1x\right)^2-2+3\left(x+\dfrac1x\right)+2=X^2+3X=0\]
これより$X=0,-3$であるから,結局,$x+\dfrac1x=0,-3$,すなわち$x^2+1=0,-3x$を解けばいい.
\[x^2+1=0~\Longleftrightarrow~x=\pm i\]
\[x^2+1=-3x~\Longleftrightarrow x^2+3x+1=0~\Longleftrightarrow~x=\frac{-3\pm\sqrt5}{2}\]
以上より,求める$x$は,
\[\boldsymbol{x=\pm i ,~\frac{-3\pm\sqrt5}{2}}\]
\begin{toi}
    4次方程式$x^4+5x^3+2x^2+5x+1=0$を解け.
\end{toi}
 \\
 \\
\begin{rembox}
$\boldsymbol{x^4+2x^2+2x^2+1=0}$は相反方程式ではない.$\boldsymbol{x^4+0x^3+2x^2+2x^2+1=0}$と考える.
\end{rembox}
\newpage
\ascboxA{\textbf{復習問題}}
\begin{toi}
$x=\displaystyle{\frac{1}{2-\sqrt3}},~~y=\displaystyle{2-\sqrt3}$のとき,次の式の値を求めよ.
\[(1)~x^2+y^2      (2)~\frac{y}{x}+\frac{x}{y}\]
\end{toi}
\begin{toi}
$\sqrt 3$の整数部分を$a$,小数部分を$b$とするとき,$\dfrac{a}{b}+\dfrac{b}{a}$の値を求めよ.
\end{toi}
\begin{toi}
$a^2+3b=b^2+3a=8$のとき,次の式の値を求めよ.ただし,$a\neq b$とする.\\
\begin{minipage}{0.25\linewidth}
\begin{itemize}
    \item [(1)]$ab$
\end{itemize}
\end{minipage}
\begin{minipage}{0.25\linewidth}
\begin{itemize}
    \item [(2)]$a+b$
\end{itemize}
\end{minipage}
\begin{minipage}{0.25\linewidth}
\begin{itemize}
    \item [(3)]$a^2+b^2$
\end{itemize}
\end{minipage}
\begin{minipage}{0.25\linewidth}
\begin{itemize}
    \item [(4)]$\dfrac{a}{b}+\dfrac{b}{a}$
\end{itemize}
\end{minipage}
\end{toi}
\begin{toi}
4次方程式$x^4-8x^3+17x^2-8x+1=0$を解け.\rightline{[2020~横浜市大 医]}
\end{toi}
\end{document}