\documentclass[a4paper,11pt]{ltjsarticle}
\usepackage{base}
\title{}
\author{}
\date{}
\newtcolorbox{rembox}[1][]{enhanced,
    before skip=2mm,after skip=3mm,fontupper=\gtfamily\sffamily,
    boxrule=0.4pt,left=5mm,right=2mm,top=1mm,bottom=1mm,
    colback=yellow!50,
    colframe=yellow!20!black,
    sharp corners,rounded corners=southeast,arc is angular,arc=3mm,
    underlay={
        \path[fill=tcbcolback!80!black] ([yshift=3mm]interior.south east)--++(-0.4,-0.1)--++(0.1,-0.2);
        \path[draw=tcbcolframe,shorten <=-0.05mm,shorten >=-0.05mm] ([yshift=3mm]interior.south east)--++(-0.4,-0.1)--++(0.1,-0.2);
        \path[fill=yellow!50!black,draw=none] (interior.south west) rectangle node[white]{\Huge\bfseries !} ([xshift=4mm]interior.north west);
    },
drop fuzzy shadow,#1}
\newcommand{\printheader}[2]{
\begin{tikzpicture}[remember picture, overlay]
\node[yshift=-2.5cm, anchor=north] at (current page.north) {
\begin{tikzpicture}
\fill[gray!20] (0,0) rectangle (\textwidth, 2cm);
\fill[gray!80] (0,0) rectangle (0.2cm, 2cm);
\draw[gray!80, thick] (0,0) -- (	\textwidth, 0);
\node[anchor=west, text width=\textwidth-1cm, inner xsep=1cm] at (0, 1.25cm) {
\parbox[b]{\linewidth}{
{\color{gray!50!black}\bfseries #1} \par
\vspace{0.2em}
{\huge\bfseries #2}
}
};
\end{tikzpicture}
};
\end{tikzpicture}
\vspace{0.5cm}
}
\begin{document}
\printheader{単元別演習 数と式④}{証明問題(解答)}

\begin{toi}
$a,b,c$ を整数とするとき,次の問に答えよ.
\begin{itemize}
    \item [(1)]$a^2$ を $4$ で割ると余りは $0$ または $1$ であることを示せ.
    \item [(2)]$a^2+b^2$ が $4$ の倍数ならば,$a,b$ はともに偶数であることを示せ.
\end{itemize}
\end{toi}
\ans 
\begin{itemize}
    \item[(1)] 4で割った余りを考えるので,$a$を4で割った余りで分類する.
    \begin{itemize}
        \item [①]$a=4k$のとき ($k$は整数)\\
        $a^2=(4k)^2=16k^2=4(4k^2)$より,4で割った余りは0.
        \item [②]$a=4k+1$のとき ($k$は整数)\\
        $a^2=(4k+1)^2=16k^2+8k+1=4(4k^2+2k)+1$より,4で割った余りは1.
        \item [③]$a=4k+2$のとき ($k$は整数)\\
        $a^2=(4k+2)^2=16k^2+16k+4=4(4k^2+4k+1)$より,4で割った余りは0.
        \item [④]$a=4k+3$のとき ($k$は整数)\\
        $a^2=(4k+3)^2=16k^2+24k+9=4(4k^2+6k+2)+1$より,4で割った余りは1.
    \end{itemize}
    以上より,$a^2$を4で割った余りは0または1である.
    \item[(2)]与えられた命題の対偶
    \[\boldsymbol{「a,bのうち少なくとも一方が奇数~\Rightarrow~a^2+b^2~は4の倍数ではない」}\]
    を示す.\\
     $a$が奇数であると仮定しても一般性は失われない.このとき,(1)より$a^2$を4で割った余りは1である.
    \begin{itemize}
        \item[①] $b$が偶数のとき,$b^2$を4で割った余りは0.$a^2+b^2$の余りは$1+0=1$.
        \item[②] $b$が奇数のとき,$b^2$を4で割った余りは1.$a^2+b^2$の余りは$1+1=2$.
    \end{itemize}
    いずれの場合も$a^2+b^2$を4で割った余りは0にならない.よって,$a^2+b^2$は4の倍数ではない.
    対偶命題が示されたので,元の命題も示された.\hfill(証明終)
\end{itemize}
\newpage
\begin{toi}
$n$ を奇数とするとき,次の問に答えよ.
\begin{itemize}
    \item [(1)]$n^2-1$は8の倍数であることを示せ.
    \item [(2)]$n^5-n$は3の倍数であることを示せ.
\end{itemize}
\hfill[千葉大]
\end{toi}
\ans 

\begin{itemize}
    \item [(1)]$n$ は奇数なので,$n=2k+1$ ($k$ は整数) と表せる.
この式を $n^2-1$ に代入すると,
\[n^2-1 = (2k+1)^2-1 = (4k^2+4k+1)-1 = 4k^2+4k = 4k(k+1)\]
ここで,$k$ と $k+1$ は連続する2つの整数なので, $k(k+1)$ は 偶数.よって,$k(k+1)=2m$ ($m$ は整数) と表せる.
したがって,
\[n^2-1 = 4k(k+1) = 4\cdot2m= 8m\]
$m$ は整数なので,$8m$ は $8$ の倍数である.
よって,$n$ が奇数のとき $n^2-1$ は $8$ の倍数.

    \item [(2)]まず,$n^5-n$ を因数分解すると,
\[n^5-n = n(n^4-1) = n(n^2-1)(n^2+1) = (n-1)n(n+1)(n^2+1)\]
$(n-1)n(n+1)$連続する3つの数の積なので,3の倍数である(実際は6の倍数まで言える).
よって,$n^5-n$は3の倍数.
\begin{rembox}
 (2)は$n$が偶数でも成り立つ($n$が奇数であることは使っていない).
\end{rembox}
\end{itemize}
\end{document}