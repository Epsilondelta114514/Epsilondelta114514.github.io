\documentclass[a4paper,11pt]{ltjsarticle}
\usepackage{base}
\title{}
\author{}
\date{}
\newcommand{\printheader}[2]{
\begin{tikzpicture}[remember picture, overlay]
\node[yshift=-2.5cm, anchor=north] at (current page.north) {
\begin{tikzpicture}
\fill[gray!20] (0,0) rectangle (\textwidth, 2cm);
\fill[gray!80] (0,0) rectangle (0.2cm, 2cm);
\draw[gray!80, thick] (0,0) -- (	\textwidth, 0);
\node[anchor=west, text width=\textwidth-1cm, inner xsep=1cm] at (0, 1.25cm) {
\parbox[b]{\linewidth}{
{\color{gray!50!black}\bfseries #1} \par
\vspace{0.2em}
{\huge\bfseries #2}
}
};
\end{tikzpicture}
};
\end{tikzpicture}
\vspace{0.5cm}
}
\begin{document}
\printheader{単元別演習 数と式④}{証明問題}
\begin{exque}
$a,b,c$ を整数とする.このとき,次のことを示せ.

\begin{enumerate}
  \item[(1)] $a^2$ を $3$ で割ると余りは $0$ または $1$ である.
  \item[(2)] $a^2 + b^2$ が $3$ の倍数ならば,$a,b$ はともに $3$ の倍数である.
\end{enumerate}
\end{exque}
\ans 
\begin{itemize}
    \item [(1)]3で割った余りを考えるので,はじめから$a$も3で割った余りで分類しておけばいい.\begin{itemize}
        \item [①]$a=3k$のとき\\
        $a^2=9k^2$より,3で割った余りは0.
        \item [②]$a=3k+1$のとき\\$a^2=9k^2+6k+1=3(3k^2+2k)+1$より,3で割った余りは1.
        \item [③]$a=3k+2$のとき\\$a^2=9k^2+12k+4=3(3k^2+4k+1)+1$より,3で割った余りは1.
    \end{itemize}
    以上より,$a^2$を3で割った余りは0または1である.(\textbf{この結果はよく使うので覚える.})\\
    \item[(2)]$a^2+b^2$の情報から$a,b$を復元するのは難しいので,対偶命題
    \[\boldsymbol{「a,bのうち少なくとも一方が3の倍数でない~\Rightarrow~a^2+b^2~は3の倍数ではない」}\]
    を示す.\\
     $a$が3の倍数でないと仮定しても一般性は失わない(条件は$a,b$に関して対称だから).このとき,(1)より$a^2$を3で割った余りは1である.$b^2$を3で割った余りは0または1であるから,$a^2+b^2$を3で割った余りは1または2である.よって,$a^2+b^2$は3の倍数ではない.対偶命題が示されたので,元の命題も示された.\hfill(証明終)
\end{itemize}
\begin{toi}
$a,b,c$ を整数とするとき,次の問に答えよ.
\begin{itemize}
    \item [(1)]$a^2$ を $4$ で割ると余りは $0$ または $1$ であることを示せ.
    \item [(2)]$a^2+b^2$ が $4$ の倍数ならば,$a,b$ はともに偶数であることを示せ.
\end{itemize}
\end{toi}
\begin{toi}
$n$ を奇数とするとき,次の問に答えよ.
\begin{itemize}
    \item [(1)]$n^2-1$は8の倍数であることを示せ.
    \item [(2)]$n^5-n$は3の倍数であることを示せ.
\end{itemize}
\hfill[千葉大]
\end{toi}
% \newpage
% \ascboxA{\textbf{自習用問題}}
% \begin{toi}
% 3辺の長さが整数である直角三角形の,少なくとも1つの辺の長さは3の倍数であることを示せ.
% \end{toi}
% \begin{toi}
% \begin{itemize}
%     \item [(1)]$n$を自然数とする,$n,~n+2,~n+4$がすべて素数であるのは$n=3$の場合だけであることを示せ.\hfill[早稲田大]
%     \item [(2)]$n$を2以上の自然数とするとき,$n^4+4$は素数にならないことを示せ.\hfill[宮崎大]
% \end{itemize}
% \end{toi}
\end{document}