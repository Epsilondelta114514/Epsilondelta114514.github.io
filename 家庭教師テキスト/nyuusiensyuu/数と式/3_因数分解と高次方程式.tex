\documentclass[a4paper,11pt]{ltjsarticle}
\usepackage{base}
\title{}
\author{}
\date{}
\newtcolorbox{rembox}[1][]{enhanced,
    before skip=2mm,after skip=3mm,fontupper=\gtfamily\sffamily,
    boxrule=0.4pt,left=5mm,right=2mm,top=1mm,bottom=1mm,
    colback=yellow!50,
    colframe=yellow!20!black,
    sharp corners,rounded corners=southeast,arc is angular,arc=3mm,
    underlay={
        \path[fill=tcbcolback!80!black] ([yshift=3mm]interior.south east)--++(-0.4,-0.1)--++(0.1,-0.2);
        \path[draw=tcbcolframe,shorten <=-0.05mm,shorten >=-0.05mm] ([yshift=3mm]interior.south east)--++(-0.4,-0.1)--++(0.1,-0.2);
        \path[fill=yellow!50!black,draw=none] (interior.south west) rectangle node[white]{\Huge\bfseries !} ([xshift=4mm]interior.north west);
    },
drop fuzzy shadow,#1}
\newcommand{\printheader}[2]{
\begin{tikzpicture}[remember picture, overlay]
\node[yshift=-2.5cm, anchor=north] at (current page.north) {
\begin{tikzpicture}
\fill[gray!20] (0,0) rectangle (\textwidth, 2cm);
\fill[gray!80] (0,0) rectangle (0.2cm, 2cm);
\draw[gray!80, thick] (0,0) -- (	\textwidth, 0);
\node[anchor=west, text width=\textwidth-1cm, inner xsep=1cm] at (0, 1.25cm) {
\parbox[b]{\linewidth}{
{\color{gray!50!black}\bfseries #1} \par
\vspace{0.2em}
{\huge\bfseries #2}
}
};
\end{tikzpicture}
};
\end{tikzpicture}
\vspace{0.5cm}
}
\begin{document}
\printheader{単元別演習 数と式③}{因数分解・高次方程式}
\begin{exque}
$x^4-2x^3-x+2$を因数分解せよ.
\end{exque}
\ascboxG{\textbf{Point.}}3次以上の因数分解→強引に「$=0$の解」(根という)を見つけて因数定理を使う.有理数解の候補は
\[\boldsymbol{\pm\frac{(定数項の約数)}{( 最高次数の係数の約数)}}\]
の形のみなので,ここから攻めるとよい.
\ans 
$x=1$を代入すると$0$になるので,因数定理により,$(x-1)$で割った余りは0である.つまり,$x^4-2x^3-x+2$は$x-1$で割り切れる.実際に割り算をしてみると,
\[x^4-2x^3-x+2=(x-1)(x^3-x^2-x-2)\]
である.\\
 次に,$x^3-x^2-x-2$を因数分解する.$x=2$を代入すると0になるので,上と同様に$x-2$で割り切れる:
\[x^3-x^2-x-2=(x-2)(x^2+x+1).\]
$x^2+x+1$は(有理数の範囲では)これ以上因数分解できないので,以上の結果を合わせて
\[\boldsymbol{x^4-2x^3-x+2=(x-1)(x-2)(x^2+x+1)}.\]
 
\begin{rembox}
2次式が有理数の範囲でこれ以上因数分解できないことは,$=0$の解を探せばすぐに分かる.$\boldsymbol{x^2+x+1=0}$の解は$\boldsymbol{x=\dfrac{-1\pm\sqrt3 i}{2}}$なので,これ以上因数分解するなら
\[\boldsymbol{x^2+x+1=\left(x-\dfrac{-1+\sqrt3 i}{2}\right)\left(x-\dfrac{-1-\sqrt3 i}{2}\right)}\]
しかありえない.よって,有理数の範囲ではこれ以上因数分解できないことがわかる.\\

3次以上の場合はそう簡単にいかないので注意.$\boldsymbol{x^4+4=0}$は実数解すら持たないが,
\[\boldsymbol{x^4+4=(x^2+2)^2-4x^2=(x^2+2x+2)(x^2-2x+2)}\]
と因数分解できる.
\end{rembox}
\begin{toi}
次の式を因数分解せよ.\\
\begin{minipage}{0.5\linewidth}
\begin{itemize}
    \item [(1)]$x^3-4x^2-7x+10$
\end{itemize}
\end{minipage}
\begin{minipage}{0.5\linewidth}
\begin{itemize}
    \item [(2)]$x^3+2x^2-2x-1$
\end{itemize}
\end{minipage}
\end{toi}
\begin{exque}
3次方程式$x^3-4x^2-7x+10=0$を解け.
\end{exque}
\ascboxG{\textbf{Point.}}まずは2次以下が出てくるまで因数分解しよう.2次式が更に因数分解するなり解の公式を使うなり自由にやってOK.
\ans 
左辺を因数分解すると,$x^3-4x^2-7x+10=(x-1)(x+2)(x-5)$なので,$\boldsymbol{x=1,-2,5}$
\begin{toi}
次の方程式を解け.\\[5pt]
\begin{minipage}{0.5\linewidth}
\begin{itemize}
    \item [(1)]$x^3+2x^2-2x-1=0$
    \item [(3)]$x^4+3x^2-4=0$
\end{itemize}
\end{minipage}
\begin{minipage}{0.5\linewidth}
\begin{itemize}
    \item [(2)]$x^3-5x^2+6x=0$
    \item [(4)]$x^3+3x^2+3x+1=0$
\end{itemize}
\end{minipage}
\end{toi}
 \\
\ascboxA{\textbf{因数分解いろいろ(自習用)}}
計算の工夫が必要な因数分解の入試問題を集めたので,自習に使ってください.
\begin{toi}
次の式を因数分解せよ.
\begin{itemize}
    \item [(1)]$(x^2+2x-30)(x^2+2x-8)-135$\hfill(北海学園大)
    \item [(2)]$(x-4)(x-2)(x+1)(x+3)+24$\hfill(東洋大)
    \item [(3)]$x(x+1)(x+2)(x+3)+1$\hfill(松山大)
    \item [(4)]$(x+1)(x+2)(x+3)(x+4)-3$\hfill(九州東海大)
\end{itemize}
\end{toi}
\begin{toi}
次の式を因数分解せよ.
\begin{itemize}
    \item [(1)]$2x^2+5xy+3y^2-3x-5y-2$\hfill(京都産業大)
    \item [(2)]$2x^2+3xy-2y^2+5y-2$\hfill(京都産業大)
        \item [(3)]$a^3+a^2-2a-a^2b-ab+2b$\hfill(摂南大)
\end{itemize}
\end{toi}
\end{document}