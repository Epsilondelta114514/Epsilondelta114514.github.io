\documentclass[a4paper,11pt]{ltjsarticle}
\usepackage{base}
\title{}
\author{}
\date{}
\begin{document}
\thispagestyle{empty}
\begin{ascolorbox17}{2016年 PM問3}
$\zeta = \exp(2\pi i/9)$ を1の原始9乗根とするとき, $\Q(\zeta)$ の部分体をすべて求めよ.
\end{ascolorbox17}
\ans
素体$\Q$を含むので,$\Q(\zeta)/\Q$の中間体をすべて求めればよい.$\zeta$の$\Q$上の最小多項式は
\[f(X)=\frac{X^9-1}{X(X-\zeta^3)(X-\zeta^6)}=\frac{X^8+\cdots X+1}{X^2+X+1}=X^6+X^3+1\]
である.実際,$f(X+1)$は$p=3$のアイゼンシュタイン多項式で既約.よって,拡大次数は6である.\\
 あとは適当に計算すれば, $\mathrm{Gal}(\Q(\zeta)/\Q)$は$\sigma:\zeta\mapsto \zeta^2$が生成する巡回群になることがわかる.
\[ \mathrm{Gal}(\Q(\zeta)/\Q) \simeq \Z/6\Z \]
である.$(\Z/9\Z)^\times = \{1, 2, 4, 5, 7, 8\}$ であり,この群は $2$ を生成元とする位数 $6$ の巡回群である.$\Z/6Z$の部分群は
\[\{1\},~\langle\sigma^3\rangle,~\langle \sigma^2\rangle,~\Z/6\Z\]
の4つであるから,ガロアの基本定理によって$\Q(\zeta)$ の部分体は $\Q$ と $\Q(\zeta)$ 自身を含めて全部で4つ存在する.\\
 次いで,対応する部分体を決定する.まず,$\{1\}$に対応するのは$\Q(\zeta)$で,$\Z/6\Z$に対応するのは$\Q$である.
\begin{itemize}
\item[(1)]$\langle\sigma^3\rangle$の固定体\\
$\sigma^3:\zeta\mapsto \zeta^8=\overline{\zeta}$が複素共役写像であることから,$\sigma^3$は$\re\zeta=\cos2\pi/9$を固定する.3倍角の公式 $\cos3\theta = 4\cos^3\theta - 3\cos\theta$で$\theta=2\pi/9$とすることにより,$\cos2\pi/9$は
\[ -\frac{1}{2} = 4X^3 - 3X \implies 8X^3 - 6X + 1 = 0 \]
の根であることがわかる.$X\mapsto X+1$としてアイゼンシュタインの判定法を使えば既約とわかるので,これが最小多項式.$\langle\sigma^3\rangle$の固定体は3次拡大なので,対応する体は$\Q(\cos 2\pi/9)$である.
\item[(2)]$\langle \sigma^2\rangle$の固定体\\
$\sigma^2:\zeta\mapsto \zeta^4$なので,$\sigma^2$は$\zeta^3=\exp(2\pi i/3)$を固定する.よって,対応する体は$\Q(i\sin 2\pi/3)=\Q(\sqrt{-3})$である.
\end{itemize}
以上より,求める部分体は$\Q(\zeta),~\Q(\cos 2\pi/9),~\Q(\sqrt{-3}),~\Q$である.
\end{document}