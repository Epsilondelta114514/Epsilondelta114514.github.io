\documentclass[a4paper,11pt]{ltjsarticle}
\usepackage{base}
\title{}
\author{}
\date{}
\begin{document}
\begin{ascolorbox17}{2021年PM問3}
体の拡大 $\Q(\sqrt[6]{2})/\Q$ の中間体をすべて求めよ.
\end{ascolorbox17}
\ascboxF{\textbf{方針}}
体 $K = \Q(\sqrt[6]{2})$ は $\Q$ 上のガロア拡大ではない,なぜなら,$\alpha = \sqrt[6]{2}$ の最小多項式 $x^6-2$ は,$K$ に含まれない虚数根を持つため,正規拡大ではないからである.

そこで,$\alpha$ の最小分解体(ガロア閉包)$L = \Q(\sqrt[6]{2}, \zeta_6)$に埋め込んで$G = \text{Gal}(L/\Q)$ に対するガロア対応を利用する方針を取る.

$\Q(\sqrt[6]{2})$ の中間体 $F$ ($\Q \subseteq F \subseteq \Q(\sqrt[6]{2})$) は,$L$ の中間体でもある.ガロア理論の基本定理より,体の包含関係は群の包含関係に逆転して対応する.
\[
F \subseteq \Q(\sqrt[6]{2}) \quad \Longleftrightarrow \quad \text{Gal}(L/F) \supseteq \text{Gal}(L/\Q(\sqrt[6]{2}))
\]
したがって,$\text{Gal}(L/\Q(\sqrt[6]{2}))$ を部分群として含むような $\text{Gal}(L/\Q)$ の部分群をすべて求め,それらに対応する固定体を決定することが目標となる.\\[5pt]
\ans
\begin{itemize}
\item[(1)]ガロア群 $\mathrm{Gal}(L/\Q)$ の構造\\
$L = \Q(\sqrt[6]{2}, \zeta_6)$ とする.
体の拡大次数を計算すると,
\[
[L:\Q] = [L:\Q(\zeta_6)][\Q(\zeta_6):Q] = 6 \cdot 2 = 12
\]
となるので,$|G|=12$ である.

$G$ の自己同型写像は生成元の行き先で決まるから,
\begin{align*}
\sigma &: \sqrt[6]{2} \mapsto \zeta_6 \sqrt[6]{2}, \quad \zeta_6 \mapsto \zeta_6 \\
\tau &: \sqrt[6]{2} \mapsto \sqrt[6]{2}, \quad \zeta_6 \mapsto \zeta_6^{-1} \quad (\text{複素共役})
\end{align*}
を考える.これらの生成元は $\sigma^6 = \tau^2 = \text{id}$, $\tau\sigma = \sigma^{-1}\tau$ の関係を満たすから,$\langle \tau,\sigma\rangle$は位数12の二面体群 $D_6$ である.よって $\mathrm{Gal}(L/\Q) \cong D_6$.
\item[(2)]$H = \mathrm{Gal}(L/\Q(\sqrt[6]{2}))$ の特定\\
$\Q(\sqrt[6]{2})$ を固定する部分群は,$H = \{\text{id}, \tau\}$ である.
\item[(3)]$H$ を含む部分群のリストアップ\\
$\mathrm{Gal}(L/\Q) \cong D_6$ の部分群 $S$ で,$H$ を含むものを探す.$|S|$ は12の約数かつ2の倍数なので,候補は位数 2, 4, 6, 12.
\begin{itemize}
    \item 位数 2: $S_1 = H = \{\text{id}, \tau\}$
    \item 位数 4: $S_2 = \langle \sigma^3, \tau \rangle = \{\text{id}, \sigma^3, \tau, \sigma^3\tau\}$
    \item 位数 6: $S_3 = \langle \sigma^2, \tau \rangle = \{\text{id}, \sigma^2, \sigma^4, \tau, \sigma^2\tau, \sigma^4\tau\}$
    \item 位数 12: $S_4 = G$
\end{itemize}
$H$ を含む部分群は以上の4つのみである.
\item[(4)]対応する固定体\\
\item[] ガロア対応により,これら4つの部分群が求める中間体のすべてに対応する.
\begin{description}
    \item[$S_1=H$] 固定体は $L^H = \Q(\sqrt[6]{2})$.次数は $12/2=6$.
    \item[$S_4=G$] 固定体は $L^G = \Q$.次数は $12/12=1$.
    \item[$S_2=\langle \sigma^3, \tau \rangle$] 対応する固定体の次数は $12/4=3$.
    この体の元は $\tau$ と $\sigma^3$ の両方で固定される.
    $x = (\sqrt[6]{2})^2 = \sqrt[3]{2}$ を考えると,
    $\tau(x) = x$ (実数なので).
    $\sigma^3(x) = \sigma^3((\sqrt[6]{2})^2) = (\sigma^3(\sqrt[6]{2}))^2 = (-\sqrt[6]{2})^2 = \sqrt[3]{2}$.
    よって $\Q(\sqrt[3]{2})$ はこの固定体に含まれる.次数が一致するため,固定体は $\Q(\sqrt[3]{2})$.

    \item[$S_3=\langle \sigma^2, \tau \rangle$] 対応する固定体の次数は $12/6=2$.
    この体の元は $\tau$ と $\sigma^2$ の両方で固定される.
    $y = (\sqrt[6]{2})^3 = \sqrt{2}$ を考えると,
    $\tau(y) = y$ (実数なので).
    $\sigma^2(y) = \sigma^2((\sqrt[6]{2})^3) = (\sigma^2(\sqrt[6]{2}))^3 = (\zeta_6^2 \sqrt[6]{2})^3 = \zeta_6^6 (\sqrt[6]{2})^3 = \sqrt{2}$.
    よって $\Q(\sqrt{2})$ はこの固定体に含まれる.次数が一致するため,固定体は $\Q(\sqrt{2})$.
\end{description}
\end{itemize}
以上より,$\Q(\sqrt[6]{2})$ の中間体は,$\Q, \Q(\sqrt{2}), \Q(\sqrt[3]{2}), \Q(\sqrt[6]{2})$ の4つである\footnote{固定体を求めるのは面倒なので,個数がわかったら後は具体的に構成してしまってもOK.}.

\end{document}