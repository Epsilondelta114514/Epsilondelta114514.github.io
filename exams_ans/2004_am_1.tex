\documentclass[a4paper,11pt]{ltjsarticle}
\usepackage{base}
\title{}
\author{}
\date{}
\begin{document}
\begin{ascolorbox17}{2004年AM問1} 
 $n$は2以上の整数とし,$A\in M_n(\C)$とする.いま,次の条件$(\ast)$を満たす$A$を決定することを考える:
            \begin{equation}
          \forall X,Y\in M_n(\C)~に対し,~XY=A\Rightarrow YX=A~\tag{$\ast$}
        \end{equation}
    以下,$n$次単位行列を$E_n$と表す.
    \begin{itemize}
        \item [(1)]任意の$a\in\C\backslash\{0\}$に対し,$A=aE_n$は条件$(\ast)$を満たすことを示せ.
        \item [(2)]$M\in M_n(\C)$とする.$MX=XM$が任意の$GL_n(\C)$に対して成立するなら,ある$z\in\C$があり$M=zE_n$となることを示せ.
        \item [(3)]条件$(\ast)$を満たす行列$A$は$(1)$の形に限ることを示せ.
    \end{itemize}
  \end{ascolorbox17}
\noindent (2):眺めていてもよくわからないので,適当に成分表示してみるとよい.\\
(3):(2)を使えるように,任意の正則行列$X$に対して$AX=XA$を示す.
\ans
\begin{itemize}
    \item [(1)]$XY=aE_n$と仮定すると,$\dfrac{X}{a}Y=E_n$である.よって,$Y=\left(\dfrac{X}{a}\right)^{-1}$であるから,$YX=aE_n$を得る.\\[3pt]
\item[(2)]$M=(a_{ij}),X=(x_{ij})$とおく.$MX=XM$を行列の成分で表すと$\sum_{k=1}^na_{ik}x_{kj}=\sum_{k=1}^nx_{ik}a_{kj}$であるから,左辺にすべて寄せて次の等式を得る.
\begin{equation}\label{2004}
    a_{i1}x_{1j}+a_{i2}x_{2j}+\cdots+a_{in}x_{nj}-a_{1j}x_{i1}-a_{2j}x_{i2}\cdots-a_{nj}x_{in}=0 \tag{$\dag$}
\end{equation}
以下,等式\eqref{2004}を用いて$M$の各成分を決定する.\\
 まず,$i=j$の場合を考えると,\eqref{2004}は次のように整理できる:
\begin{equation}\label{2004-1}
    (a_{i1}x_{1i}-a_{1i}x_{i1})+(a_{i2}x_{2i}-a_{2i}x_{i2})+\cdots+(a_{in}x_{ni}-a_{ni}x_{in})=0 \tag{$\dag\dag$}
\end{equation}$k\neq i$を1つ固定し,$X$を$E_n$の$i$行目と$k$行目を入れ替えた行列とすると,\eqref{2004-1}から$a_{ik}=a_{ki}$がわかる.次に,$E_n$の$i$行目と$k$行目を入れ替えた行列において$x_{ik}=-1$に置き換えたものを$X$とすると,\eqref{2004-1}から$a_{ik}+a_{ki}=0$がわかる.これら2つの式から$a_{ik}=a_{ki}=0$を得るので,$i=1,2,\ldots,n$と$k\neq i$を動かすことにより$M$の対角成分以外はすべて$0$であることがわかる.\\[3pt]
次に,対角成分を決定する.$M$の対角成分以外は0になっているので,\eqref{2004}において$i=1$として\[a_{11}x_{1j}-a_{1j}x_{1j}=(a_{11}-a_{jj})x_{1j}=0\]
を得る.$X$は正則行列なので,適当に行基本変形をして$x_{1j}\neq0$なるものを取ってこれる.よって,$a_{11}=a_{jj}~;~j=2,\ldots,n$である.以上より,$A=a_{11}E_n$である.\\[3pt]
\item[(3)]任意の正則行列$X$に対して$(AX)X^{-1}=A$であるので,$A$が条件($\ast$)を満たすことから$X^{-1}(AX)=A$,つまり$AX=XA$である.よって,$(2)$の結果より$A=aE_n~;~a\in\C$と表せる.あとは$a\neq0$を示せばよいが,$X,Y$を
\[
X=
\begin{pmatrix}
    1 & -1&0 &\cdots&0\\
    0 & 0&0 &\cdots&0\\
    0&0&0 &\cdots&0\\
    \vdots&\vdots&\vdots &\ddots&0\\
    0&0&0 &\cdots&0\\
    \end{pmatrix} ,~Y=
\begin{pmatrix}
1 & 0&0 &\cdots&0\\
1 & 0&0 &\cdots&0\\
0&0&0 &\cdots&0\\
\vdots&\vdots&\vdots &\ddots&0\\
0&0&0 &\cdots&0\\
\end{pmatrix} 
\]
で定めると$XY=O$だが$YX\neq O$なので$a=0$のときは条件$(\ast)$を満たさない.\\
 結局,条件($\ast$)を満たすならば$A=aE_n~;~a\in\C\backslash\{0\}$である.(1)と合わせて,条件$(\ast)$を満たす$A$はこの形に限ることがわかった.
\end{itemize}
\end{document}

