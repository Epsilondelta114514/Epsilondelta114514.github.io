\documentclass[a4paper,11pt]{ltjsarticle}
\usepackage{base}
\title{}
\author{}
\date{}
\begin{document}
\begin{ascolorbox17}{2016年PM問1}
$f(x) = x^4-1$ とする.
\begin{itemize}
    \item[(1)] 剰余環 $\mathbb{C}[x]/(f(x))$ の素イデアルをすべて求めよ.
    \item[(2)] 剰余環 $\mathbb{R}[x]/(f(x))$ の素イデアルと極大イデアルをすべて求めよ.
    \item[(3)] 剰余環 $\mathbb{Z}[x]/(5, f(x))$ の素イデアルをすべて求めよ.
\end{itemize}
\end{ascolorbox17}
\ans
\begin{itemize}
    \item [(1)]イデアルの対応定理により,$\C[x]/(f(x))$の素イデアルは$\C[x]$の素イデアルで$(f(x))$を含むものと対応する.$\mathbb{C}[x]$ は体 $\mathbb{C}$ 上1変数多項式環なので単項イデアル整域(PID)である.よって,素イデアルと極大イデアルは一致し,既約多項式で生成されるイデアルである.\\
 $\mathbb{C}$ は代数的閉体なので、0でない既約多項式は1次式に限られる.$f(x) = x^4-1$ を含む素イデアルは、$f(x) = (x-1)(x+1)(x-i)(x+i)$ の既約因子で生成されるものであるので,$(f(x))$ を含む $\mathbb{C}[x]$ の素イデアルは以下の4つである.
\[ (x-1), \quad (x+1), \quad (x-i), \quad (x+i) \]
したがって、$\mathbb{C}[x]/(f(x))$ の素イデアルはこれらのイデアルの像ですべてである:
\[ (x-1 + (f(x))), \quad (x+1 + (f(x))), \quad (x-i + (f(x))), \quad (x+i + (f(x))) \]

\item[(2)]
イデアルの対応定理より,$\R[x]/(f(x))$の素イデアルは,$(f(x))$ を含む $\R[x]$ の素イデアルと対応する.$\R[x]$ は体 $\R$ 上のPIDであり,そのゼロでない素イデアルは極大イデアルと一致する.
$\R$ 上での $f(x)$ の既約多項式分解は,
\[ f(x) = (x-1)(x+1)(x^2+1) \]
となる.各因子は $\R[x]$ 上で既約である.
よって,$(f(x))$ を含む $\R[x]$ の素イデアルは以下の3つである.
\[ (x-1), \quad (x+1), \quad (x^2+1) \]
これらは $\R[x]$ の極大イデアルでもある.
したがって,$\R[x]/(f(x))$ の素イデアルはこれらの像であり,それらは同時に極大イデアルでもある:
\[ (x-1 + (f(x))), \quad (x+1 + (f(x))), \quad (x^2+1 + (f(x))) \]
\item[(3)]第三同型定理により,
\[\Z[x]/(5, f(x)) \cong(\Z[x]/(5))/((\overline{5}, \overline{f(x)})/(\overline{5})) \cong(\Z/5\Z)[x]/(\overline{f(x)})\]
である.ここで $\Z/5\Z $ は体であり,$\overline{f(x)}$ は $f(x)$ の係数を $\Z/5\Z$ で考えたものである.
よって問題は,$\Z/5\Z[x]$ において $(\overline{f(x)})$ を含む素イデアルを求めることに帰着する.$\Z/5\Z[x]$ は体上の1変数多項式環ゆえPIDなので,その素イデアルは $\overline{f(x)}$ の既約因子で生成される単項イデアルに限る.\\
 $\sqrt{-1}=2$ なので$\overline{f(x)}$ は次のように因数分解される.
\[ \overline{f(x)} = x^4-1 = (x-1)(x-2)(x-3)(x-4) \]
各因子は1次式であり,$\Z/5\Z[x]$ で既約である.
したがって,$\Z/5\Z[x]$ で $(\overline{f(x)})$ を含む素イデアルは以下の4つである.
\[ (x-1), \quad (x-2), \quad (x-3), \quad (x-4) \]
環同型を通じて,これらが求める素イデアルに対応する.
よって $\Z[x]/(5, f(x))$ の素イデアルは,これらのイデアルの生成元の像で生成されるものですべてである:
\[ (x-1 + (5, f(x))), \ (x-2 + (5, f(x))), \ (x-3 + (5, f(x))), \ (x-4 + (5, f(x))) \]

\end{itemize}
\end{document}
